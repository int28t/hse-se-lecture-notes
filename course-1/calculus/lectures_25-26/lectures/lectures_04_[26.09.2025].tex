\begin{example}
    б.б. + б.б.

    Мы точно не можем сказать, чем будет являться эта сумма. Например $n + (-n) = 0$ и $n + n = 2n$
\end{example}

\begin{example}
    б.б. + огр. = б.б.

    Покажем, что это так

    Хотим: $b_n + c_n = u_n$ соответственно. Тогда:
    $$ \forall M > 0\ \exists N(M) \in \N\ \forall n > N(M)\ |b_n| > M $$
    $$ \exists C > 0\ \forall n \in \N\ |c_n| \leq C $$

    Хотим:
    $$ \forall K > 0\ \exists N_2(K) \in \N\ \forall n > N_2(K)\ |b_n + c_n| > K $$
    Так, как $ |x + y| \geq |x| - |y| $:
    $$ |b_n| - |c_n| > K $$
    $$ |b_n| - C > K $$
    $$ |b_n| > K + C $$
    Возьмем $N_2(K) = N(K + C)$
\end{example}

\begin{theorem}
    $$ \Lim{n}{\infty} a_n = A \Leftrightarrow (a_n - A) = \alpha_n \text{ - бесконечно малая} $$
\end{theorem}

\begin{Proof}
    $$ \forall \eps > 0\ \exists N_1(\eps) \in \N\ \forall n > N_1(\eps)\ |a_n - A| < \eps $$
    Хотим
    $$ \forall \eps > 0\ \exists N_2(\eps) \in \N\ \forall n > N_2(\eps)\ |(a_n - A) - 0| < \eps $$
    $$ N_2(\eps) = N_1(\eps) $$
\end{Proof}

\begin{remark}
    Теперь можно спокойно доказать свойство 2) из арифметики пределов
\end{remark}

\begin{remark}
    Произведение бесконечно малых - бесконечно малая последовательность. Так как можно представить одну из них как ограниченную и использовать теорему выше
\end{remark}

\begin{definition}
    Назовем последовательность $d_n$ \textbf{отделимой от нуля}, если:
    $$ \exists \delta > 0\ \forall n \in \N\ |d_n| \geq \delta $$
\end{definition}

\begin{example}
    $(-1)^n$
\end{example}

\begin{theorem}
    $$ \frac{1}{\text{огр}} = \text{отделима от нуля},\ \frac{1}{\text{отделима от нуля}} = \text{огр} $$
\end{theorem}

\begin{Proof}
    $$ \exists \delta > 0\ \forall n \in \N\ |d_n| \geq \delta $$
    $$ u_n = \frac{1}{d_n},\ \exists c > 0\ \forall n \in \N\ |u_n| \leq c $$
    $$ \Updownarrow $$
    $$ \l|\frac{1}{d_n}\r| \leq c $$
    $$ \Updownarrow $$
    $$ |d_n| \geq \frac{1}{c} $$
    Возьмем $ c = \frac{1}{\delta} $. Тогда $ |d_n| \geq \delta $ верно $ \forall n \in \N $
\end{Proof}

\begin{theorem}
    Если $ \Lim{n}{\infty} u_n = U;\ u_n, U \neq 0$,\ то $u_n$ отделима от нуля
\end{theorem}

\begin{Proof}
    $$ \forall \eps > 0\ \exists N(\eps) \in \N\ \forall n > N(\eps)\ |u_n - U| < \eps $$
    Хотим
    $$ \exists \delta > 0\ \forall n \in \N\ |u_n| \geq \delta $$
    Возьмем $ \eps_0 = \frac{|U|}{2} $
    $$ \exists N_0 = N\l(\frac{|U|}{2}\r) = N(\eps_0) $$
    $$ \forall n > N_0\ u_n \in U_{\frac{|U|}{2}}(U) $$
    Очевидно, что существует конечное число $u_k,\ k \leq N(\eps_0)$, причем $u_k \neq 0$

    Возьмем $\delta = min\{|u_1|, |u_2|, \ldots, \frac{|U|}{2}\}$
\end{Proof}

\subsection{Предельный переход в неравенствах}

\begin{theorem}
    Если $\exists N_0\ \forall n > N_0\ a_n > b_n$ и $a_n \underset{n \to \infty}{\to} A$ и $b_n \underset{n \to \infty}{\to} B$, то 
    $ A \geq B $
\end{theorem}

\begin{Proof}
    Пп $A < B$. Возьмем $\eps_0 = \frac{B - A}{2} $
    $$ \exists N_1(\eps_0)\ \forall n > N_1(\eps_0)\ a_n \in U_{\eps_0}(A) $$
    $$ \exists N_2(\eps_0)\ \forall n > N_2(\eps_0)\ b_n \in U_{\eps_0}(B) $$
    Возьмем $n_0 = N_1(\eps_0) + N_2(\eps_0) + N_0$
    Тогда по условию $a_{n_0} > b_{n_0}$, но также $a_{n_0} < b_{n_0}$ (так как окрестность $a$ по предположению левее окрестности $b$). Получили противоречие
\end{Proof}

\subsection{Теорема о зажатой последовательности}
\begin{theorem}[Теорема о зажатой последовательности]
    Если $\exists N_0\ \forall n > N_0\ a_n \leq c_n \leq d_n$, а также $a_n \underset{n \to \infty}{\to} A$ и $d_n \underset{n \to \infty}{\to} A$, то $c_n \underset{n \to \infty}{\to} A$
\end{theorem}

\begin{Proof}
    $$a_n \underset{n \to \infty}{\to} A;\ \forall \eps > 0\ \exists N_1(\eps)\ \forall n > N_1(\eps)\ A-\eps < a_n < A+\eps$$
    $$d_n \underset{n \to \infty}{\to} A;\ \forall \eps > 0\ \exists N_2(\eps)\ \forall n > N_2(\eps)\ A-\eps < d_n < A+\eps$$
    Хотим
    $$c_n \underset{n \to \infty}{\to} A;\ \forall \eps > 0\ \exists N_3(\eps)\ \forall n > N_3(\eps)\ A-\eps < c_n < A+\eps$$
    Получаем
    $$ \underbrace{A-\eps <}_{\forall n > N_1(\eps)} a_n \underbrace{\leq c_n \leq}_{\forall n > N_0} d_n \underbrace{< A+\eps}_{\forall n > N_2(\eps)} $$
    Возьмем $N_3(\eps) = max\{N_1(\eps),\ N_2(\eps),\ N_0\}$
\end{Proof}

\subsection{Список хороших пределов}
\begin{enumerate}
    \item $\Lim{n}{\infty} q^n = 0,\ |q| < 1$\\
    $\Lim{n}{\infty} q^n = +\infty,\ |q| > 1$
    \item $\Lim{n}{\infty} \sqrt[n]{a} = 1,\ a > 0$
    \item $\Lim{n}{\infty} \sqrt[n]{n} = 1$
    \item $\Lim{n}{\infty} \frac{n^a}{b^n} = 0$
    \item $\Lim{n}{\infty} \frac{b^n}{n!} = 0$
    \item $\Lim{n}{\infty} \frac{\ln{n}}{n^a} = 0,\ a > 0$
\end{enumerate}

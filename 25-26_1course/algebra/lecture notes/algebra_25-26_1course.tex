\documentclass[a4paper,12pt]{article}
\usepackage{header}

\title{Лекции по алгебре (1 курс, 25-26)}
\author{github.com/int28t/hse-se-lecture-notes}
\date{}

\begin{document}
    \pagestyle{empty}
	\maketitle
	\begin{flushright}
		Михайлец Екатерина Викторовна

		emihaylets@hse.ru

		ТГК: \url{t.me/alg_pi_25_26}
	\end{flushright}
	\tableofcontents{}
    \newpage
    \pagestyle{plain}
	\section {Матрицы}
	Матрицей размера/типа/порядка $n$x$m$ называется упорядоченная таблица с $m$ строками и $n$ столбцами.

	{\bf Обозначение: }
	\[
		A = 
		\begin{pmatrix}
			a_{11} & \ldots & a_{1n} \\
			\vdots & \ddots & \vdots \\
			a_{m1} & \ldots & a_{mn} 
		\end{pmatrix}
		_{mxn}
	\]
	$i$ - строка, $j$ - столбец

	$M_{mn}(\mathbb{R})$ - множество всех матриц размера $m$x$n$ с вещественными элементами

	1) При $m = n$ матрица называется квадратной порядка $n$

	2) При $m = 1$ матрица $1$x$n$ - $n$-мерная строка

	При $n = 1$ матрица $m$x$1$ - $m$-мерный столбец

	Матрица $1$x$1$ - число

	3) Матрица состоящая из нулей (т.е. $a_{ij}=0~\forall i = \overline{1, m},~j = \overline{1, n}$) называется нулевой матрицей.
	{\bf Обозначение: $\mathbb{O}$ }

	4) Будем называть единичной квадратную матрицу порядка n, если:

	\[
		a_{ij}=\underbrace{\delta^i_j}_{\text{Символ Кронекера}}=
		\begin{cases}
			1, & \text{при  $i = j$ } \\
			0, & \text{при  $i \neq j$}
		\end{cases}
	\]

	{\bf Обозначение: }
	\[
		E = \begin{pmatrix}
			1 & 0 & 0 & \ldots & 0 \\
			0 & 1 & 0 & \ldots & 0 \\
			0 & 0 & 1 & \ldots & 0 \\
			\hdotsfor{5} \\
			0 & 0 & \ldots & 0 & 1
			\end{pmatrix}
	\]

	{\bf Определение: } Две матрицы $a$ и $b$ называются равными если они одного размера и соответствующие элементы равны (т.е. $a_{ij}=b_{ij} \forall i=\overline{1, m}, j=\overline{1, n}$)

	{\bf Определение: } Матрица $C$ называется суммой матриц A и B если все матрицы A, B и C одинакового размера и $c_{ij}=b_{ij}+a_{ij}, \forall i=\overline{1,m}, j=\overline{1,n}$

	{\bf Обозначение: } $C = A + B$

	{\bf Пример: }

	\[
		\begin{pmatrix} 
			1 & 0 & -1 \\
			0 & 0 & 2
		\end{pmatrix}
		+
		\begin{pmatrix} 
			0 & 0 & 1 \\
			0 & 0 & -2
		\end{pmatrix}
		=
		\begin{pmatrix} 
			1 & 0 & 0 \\
			0 & 0 & 0
		\end{pmatrix}
	\]

	\subsection{Свойства сложения матриц}

	1. Коммутативность ($A + B = B + A$)

	$\square$ Поэлементно: $[A+B]_{ij} \underbrace{=}_{\text{по определению сложения}} [A]_{ij}+[B]_{ij} \underbrace{=}_{\text{Вещественные числа коммутативны}} [B]_{ij}+[A]_{ij} = \underbrace{=}_{\text{по определению сложения}} [B+A]_{ij}$ $\blacksquare$

	2. Ассоциативность $(A + B) + C = A + (B + C)$

	3. $\exists$ нейтральный элемент по сложению т.е. $\exists$ матрица $\mathbb{O} \in M_{mn}(\mathbb R)~\forall A \in M_{mn}(\mathbb R)$ выполняется $\mathbb O + A = A + \mathbb O = A$

	$\square~[\mathbb O]_{ij} = 0$ -- нулевая матрица $\blacksquare$

	4. $\forall$ матрицы $A \in M_{mn}(\mathbb R)~\exists~B \in M_{mn}(\mathbb R): A + B = B + A = \mathbb O$ -- нулевая матрица
	
	$\square~B_{ij} = -A_{ij}~\blacksquare$

	{\bf Обозначение: } $B = -A$ -- Обратная по сложению к $A$ или противоположная

	{\bf Определение: } Матрица $C$ называется произведением числа $\lambda \in \mathbb R$ и матрицы $A$, если матрицы $C$ и $A$ одинакового размера и $[C]_{ij} = \lambda * [A]_{ij}~\forall i = \overline{1,m}, j = \overline{1,n}$

	{\bf Обозначение: } $C = \lambda * A$

	{\bf Пример: }
	\[
		2 * 
		\begin{pmatrix}
			1 & 3 \\
			0 & -1
		\end{pmatrix}
		=
		\begin{pmatrix}
			2 & 6 \\
			0 & -2
		\end{pmatrix}
	\]

	{\bf Определение: } Разностью матриц A и B называется сумма $A$ и $-B$

	\subsection {Свойства умножения на число}

	$\forall \lambda, \mu \in \mathbb R: (\lambda * \mu) * A = \lambda * (\mu * A)$ -- ассоциативность относительно умножения на число

	$(\lambda + \mu) * A = \lambda * A + \mu * A$ -- дистрибутивность относительно умножения чисел

	$\lambda * (A + B) = \lambda * A + \lambda * B$ -- дистрибутивность относительно сложения матриц

	{\bf Это место стоит перепроверить (!!!) } 

	$1 * A = A$ -- унарность

	\subsection{Умножение матриц}

	Рассмотрим $A_{mxn} \in M_{mn}(\mathbb R), B_{nxk} \in M_{nk}(\mathbb R)$:

	{\bf Определение: }
	Произведением матрицы $A_{mxn}$ и $B_{nxk}$ (число столбцов в $A$ равно числу строк в $B$) называется $C_{mk}$ где:
	\[
		C_{ij}=\sum_{q=1}^{n} a_{iq} * b_{qj}
	\]
	
	($C_{ij}=a_{i1} b_{1j}+a_{i2} b_{2j}+\ldots+a_{in} b_{nj}~\forall i=\overline{1,m},~j=\overline{1,k}$)

	{\bf Пример:}

	\[
		\begin{pmatrix}
			1 & 2 \\
			3 & 4 \\
			0 & -1
		\end{pmatrix}_{\text{3x2}}
		*
		\begin{pmatrix}
			1 \\
			-2
		\end{pmatrix}_{\text{2x1}}
		=
		\begin{pmatrix}
			-3 \\
			-5 \\
			2
		\end{pmatrix}_{\text{3x1}}
	\]

	\subsection{Свойства умножения матриц}

	1. Ассоциативность
	Пусть $A_{mxn}$, $B_{nxk}$, $C_{kxl}$. Тогда $(A * B) * C = A * (B * C)$

	$\square~[(A * B) * C]_{ij} = \sum_{r=1}^k~[A * B]_{ir} * [C]_{rj} = \sum_{r=1}^{k}~(\sum_{s=1}^{n}~[A]_{is} * [B]_{sr}) * [C]_{rj} = \sum_{r=1}^{k}~\sum_{s=1}^{n}~[A]_{is} * [B]_{sr} * [C]_{rj} \underbrace{=}_\text{перегруппируем слагаемые} \sum_{s=1}^{n}~\sum_{r=1}^{k}~[A]_{is} * [B]_{sr} * [C]_{rj}=\sum_{s=1}^{n}~[A]_{is} * (\sum_{r=1}^{k}~[B]_{sr} * [C]_{rj})=\sum_{s=1}^{n}~[A]_{is} * [B * C]_{sj} \underbrace{=}_\text{по определению умножения}=A * (B * C)~\blacksquare$

	2. $\exists$ Нейтрального элемента по умножению (для квадратных матриц)
	$\exists$ матрица $E \in M_{n}(\mathbb R): \forall A \in M_{n}(\mathbb R)~E * A = A * E = A$

	$\square$ Предъявим единичную матрицу

	$[E]_{ij} = \delta_j^i$

	$[A * E]_{ij} = \sum_{r=1}^{n}~[A]_{ir} * [E]_{rj} = \sum_{r=1}^{n}~[A]_{ir} * \delta_j^i = 0 + \ldots + [A]_{ij} + \ldots + 0 = [A]_{ij}~\forall i, j = \overline{1,n}$

	$E * A$ аналогично $\blacksquare$

	3. $A * \mathbb O = \mathbb O * A = \mathbb O$
	Для квадратных $A$ и $\mathbb O$ порядка $n$

	Рассмотрим: $(A + B) * C = A * C + B * C$ -- дистрибутивность умножения матриц
	{\bf Замечание: } Вообще говоря умножение матриц некоммутативно (даже для квадратных)
	{\bf Пример: }
	\[
		\begin{pmatrix}
			1 & 1 \\
			0 & 0
		\end{pmatrix}
		*
		\begin{pmatrix}
			0 & 0 \\
			1 & 1
		\end{pmatrix}
		=
		\begin{pmatrix}
			1 & 1 \\
			0 & 0
		\end{pmatrix}
	\]

	\[
		\begin{pmatrix}
			0 & 0 \\
			1 & 1
		\end{pmatrix}
		*
		\begin{pmatrix}
			1 & 1 \\
			0 & 0
		\end{pmatrix}
		=
		\begin{pmatrix}
			0 & 0 \\
			1 & 1
		\end{pmatrix}
	\]

	\section{Транспонирование}

{\bf Определение:}
Транспонирование --- операция, переводящая все строки в столбцы с сохранением порядка, то есть:
$$[A^T]_{ij} = [A]_{ji}\ \forall i = \overline{1, m}, j = \overline{1, n}$$ 

{\bf Обозначение:} $A^T$

$A_{m\times n} \xrightarrow[]{T} (A^T)_{n \times m}$ 

\subsection{Свойства транспонирования}

\begin{itemize}
    \item[1.] $(A^T)^T = A$
    \item[2.] $(A + B)^T = A^T + B^T$
    \item[3.] $(\lambda \cdot A)^T = \lambda \cdot A^T$
    \item[4.] $(A \cdot B)^T = B^T \cdot A^T$
    
    $\square$ 
    Пусть матрицы $A \in M_{mn}(\R), B \in M_{nk}(\R)$

    $[(A \cdot B)^T]_{ij} \overbrace{=}^{\text{по определению Т}} [A \cdot B]_{ji} \overbrace{=}^{\text{по определению произведения матриц}} \Sum{r=1}{n} A_{jr} \cdot B_{ri} =$ 
    
    $\overbrace{=}^{\text{по коммутативности произведения}} \Sum{r=1}{n} B_{ri} \cdot A_{jr} = \Sum{r=1}{n} B^T_{ir} \cdot A^T_{rj} = [B^T \cdot A^T]_{ij}\ \forall i = \overline{1, k}, j = \overline{1, m}$
    $\blacksquare$
\end{itemize}

\subsection{Элементарное преобразование строк(столбцов)}
{\bf Определение:} --- это 3 следующие операции:

\begin{itemize}
    \item[1.] Перестановка двух строк в матрице
    
    $(i) \Leftrightarrow (k) \text{(i-ая строка меняется местами с k-ой)}$
    \item[2.] Умножение строки на ненулевое число
    
    $\lambda \in \R,\ \lambda \neq 0, (i) \Rightarrow \lambda \cdot (i), \lambda \neq 0$
    \item[3.] Прибавление к i-ой строке другой k-ой строки с коэффициентом $\lambda$, $\lambda \in \R$
    
    $(i) \Rightarrow (i) \cdot \lambda (k),\ i \neq k$
\end{itemize}

{\bf Замечание 1:} Все элементарные преобразования обратимы

{\bf Замечание 2:} Каждое элементарное преобразование строк матрицы можно трактовать как умножение на матрицу слева специального вида (квадратную). Эта матрица получается применением к единичной матрице того же самого элементарного преобразования

{\bf Пример:} (1) элементарное преобразование к $E$

\[
    \begin{pmatrix}
        1 &  0 \\
        0 &  1 \\
    \end{pmatrix}
    \xrightarrow[]{(2) - 2(1)}
    \begin{pmatrix}
        1 &  0 \\
        -2 &  1 \\
    \end{pmatrix}
    \text{--- матрица элементарного преобразования}
\]

\[
    \begin{pmatrix}
        1 &  0 \\
        -2 &  1 \\
    \end{pmatrix}
    A
    = 
    \begin{pmatrix}
        1 &  0 \\
        -2 &  1 \\
    \end{pmatrix}
    \begin{pmatrix}
        1 & 2 &  3 \\
        2 & 1 &  3 \\
    \end{pmatrix}
    = 
    \begin{pmatrix}
        1 & 2 &  3 \\
        0 & -3 &  -3 \\
    \end{pmatrix}
\]

{\bf Замечание 3:} Преобразования можно применять и к столбцам. Это будет соответствовать умножению справа на матрицу специального вида

{\bf Определение:} Матрица имеет:
\begin{itemize}
    \item[1)] ступенчатый вид, если номера первых ненулевых элементов всех строк последовательно возрастают, такие элементы называются ведущими (нижние ведущие элементы находятся правее, чем верхние), а все нулевые строки стоят внизу
    \item[2)] канонический вид (улучшенные ступенчатый), если матрица имеет ступенчатый вид, в которой все ведущие элементы равны 1, и в любом столбце с ведущими элементами все остальные равны 0
\end{itemize}

{\bf Теорема о методе Гаусса:} Любую конечную матрицу можно привести к ступенчатому и каноническому виду элементарными преобразованиями строк.

$\square$
Предъявим алгоритм. Берем матрицу $m \times n$ и двигаемся из левого верхнего угла. 
\[
\begin{pmatrix}
    a_{11} & a_{12} & a_{13} & \ldots & a_{1n} \\
    a_{21} & a_{22} &  &  &   \\
    a_{31} &  & \ldots &  &   \\
    \ldots &  &  & \ldots &   \\
    a_{m1} &  &  &  & a_{mn}  \\
\end{pmatrix}
\]

\fbox{п. 1} 
Если текущий элемент равен 0, тогда переходим к \fbox{п. 2}, иначе объявляем теущий элемент ведущим. Прибавляем текущую строку к остальным так, чтобы элементы ниже ведущего (и выше в случае канонического вида) обратились в 0. Пусть $a_{ij}$ --- текущий элемент. Тогда для $k \neq i$ строки берем $\lambda = -\frac{a_{kj}}{a_{ij}}$, где $a_{ij} \neq 0$ и $(k) \rightarrow (k) + \lambda (i)$ (для канонического вида $(i) \rightarrow \frac{1}{a_{ij}}(i)$, чтобы получить 1 на месте ведущего элемента). 

Выбираем новый текущий элемент, смещаясь в матрице на один столбец вправо и на одну строку вниз $\Rightarrow$ следующий шаг, повторяем \fbox{п. 1}, если это невозможно \text{stop}

\fbox{п. 2}
Если текущий элемент равен нулю, то просматриваем все элементы под ним. Если среди них нет не равных 0, то переходим к \fbox{п. 3}. Иначе если в k-ой строке ненулевой элемент нашелся (под текущим), то $(i) \leftrightarrow (k)$

\fbox{п. 3}
Если текущий элемент и все под ним равны 0, то меняем текущий столбец смещаясь на один столбец вправо. Если возможно, то \fbox{п. 1}, иначе \text{stop}

Так как матрица имеет конечные размеры, а за 1 итерацию смещается вправо на 1 столбец --- процесс преобразования к ступенчатому (каноническому) виду закончится не более чем за $n$ шагов

\section{Система линейных алгебраических уравнений (СЛАУ)}

Покоординатная запись СЛАУ:
\[
    \begin{dcases}
        a_{11} x_{1} + a_{12} x_2 + \ldots + a_{1n} x_n = b_1\\
        a_{21} x_{1} + a_{22} x_2 + \ldots + a_{2n} x_n = b_2\\
        \ldots\\
        a_{m1} x_{1} + a_{m2} x_2 + \ldots + a_{mn} x_n = b_m
    \end{dcases}
    \Rightarrow
    Ax = b,
    \text{где }
    A =
    \begin{pmatrix}
        a_{11} & \ldots & a_{1n}   \\
        \ldots &  & \ldots  \\
        a_{m1}  & \ldots &  a_{mn}  \\
    \end{pmatrix}
    \text{, }
\]

\[
    x = 
    \begin{pmatrix}
        x_1 \\
        \ldots \\
        x_n \\
    \end{pmatrix}
    \text{--- столбец неизвестных, }
    b = 
    \begin{pmatrix}
        b_1 \\
        \ldots \\
        b_m \\
    \end{pmatrix}
    \text{--- столбец свободных членов}
\]

\[
    (A|B)_{m \times (n+1)} =
    \begin{pmatrix}
        a_{11} & \ldots & a_{1n} & \vline & b_1   \\
        \ldots & \ldots & \ldots & \vline & \ldots   \\
        a_{m1} & \ldots & a_{mn} & \vline & b_m   \\
    \end{pmatrix}
    \text{--- элементарными преобр. к каноническому виду}
\]

{\bf Замечание:}
Элементарные преобразования матрицы не меняют множество решений СЛАУ

	\section{Перестановки (Подстановки)}
	{\bf Определение: } Перестановкой чисел $1, \ldots, n$ называется расположение их в определённом порядке.

	\[
		\alpha =
		\begin{pmatrix}
			\alpha_1, & \alpha_2, & \ldots, & \alpha_n
		\end{pmatrix}
	\]

	{\bf Пример: } 
	\[
		\alpha =
		\begin{pmatrix}
			5, & 3, & 4, & 1, & 2
		\end{pmatrix}
	\]

	{\bf Определение: } $\alpha_i~\text{и}~\alpha_j$ образуют инверсию в перестановке если $\alpha_i > \alpha_j$, но $i < j$ 

	{\bf Определение: } Знак перестановки это $(-1)^n$, где n - число инверсий в перестановке

	{\bf Обозначение: } $sgn(\alpha)$


	{\bf Пример: }
	\[
		\alpha =
		\underbrace{
			\begin{pmatrix}
				4 & 5 & 1 & 3 & 6 & 2
			\end{pmatrix}
		}_{3+3+0+1+1+0}
	\]

	$$sgn(\alpha)=1$$

	{\bf Определение: } Если $sgn(\alpha)=1$ то $\alpha$ - чётная перестановка, если $sqn(\alpha)=-1$ то $\alpha$ - нечётная перестановка

	{\bf Определение: } Транспозицией называется преобразование при котором в $\alpha$ меняются местами только $\alpha_i$ и $\alpha_j$, а остальные элементы не меняются

	{\bf Утверждение: } Каждая транспозиция меняет чётность перестановки

	$\square$ a) Транспозиция соседних элементов:

	$\alpha_1, \ldots, \alpha_i, \alpha_{i+1}, \ldots, \alpha_n$

	$\downarrow$

	$\alpha_1, \ldots, \alpha_{i+1}, \alpha_i, \ldots, \alpha_n$

	$\Rightarrow$ Число инверсий изменилось на 1 $\Rightarrow$ знак перестановки поменялся

	б) 

	$\alpha_1, \ldots, \alpha_i, \ldots, \alpha_{i+k}, \ldots, \alpha_n$

	Меняем $\alpha_{i} \leftrightarrow \alpha_{i+k}$ с $k-1$ соседних транспозиций

	$\alpha_1, \ldots, \alpha_{i+k}, \ldots, \alpha_{i}, \ldots, \alpha_n$

	$\Rightarrow k + k - 1 = 2k - 1$ шагов (соседних транспозиций) 

	$\Rightarrow$ Знак перестановки поменяется $\blacksquare$

	{\bf Определение: } Подстановка

	\[
		\delta=
		\begin{pmatrix}
			1 & 2 & \ldots & n \\
			\delta(1) & \delta(2) & \ldots & \delta(n)
		\end{pmatrix}
	\]

	-- отображение чисел в себя являющееся взаимо-однозначным

	Нижняя строка -- перестановка

	{\bf Пример: }
	\[
		\delta =
		\begin{pmatrix}
			1 & 2 & 3 & 4 \\
			4 & 2 & 1 & 3
		\end{pmatrix}
	\]

	$sgn(\delta)=1$
	$\delta(2)=2$ - стационарный элемент

	{\bf Определение: } Знаком подстановки называется знак перестановки в её нижней строке

	{\bf Обозначение: } $S_n$ - множество подстановок длины $n$

	$|S_n|=n!$ число элементов и мощность множества

	{\bf Замечание: } Транспозиция - нечётная подстановка (в которой $\alpha_i$ и $\alpha_j$ переходят друг в друга, а остальные элементы неподвижны)

	{\bf Замечание: } Часто используют запись подстановок <<в циклах>>, и каждый элементы выписывается справа

	\[
		\begin{pmatrix}
			1 & 2 & 3 & 4 \\
			4 & 2 & 1 & 3
		\end{pmatrix}
		=
		\begin{pmatrix}
			1 & 4 & 3
		\end{pmatrix}
		*
		\underbrace{
			\begin{pmatrix}
				2
			\end{pmatrix}
		}_{\text{можно не писать}}
		=
		\begin{pmatrix}
			1 & 4 & 3
		\end{pmatrix}
	\]

	Циклическая запись транспозиции: ($\alpha_i, \alpha_j$)

	{\bf Определение: }
	\[
		\delta =
		\begin{pmatrix}
			1 & 2 & \ldots & n \\
			1 & 2 & \ldots & n
		\end{pmatrix}
		= (1)(2)(3)\ldots(n)=Id
	\]
	называется тождественной подстановкой

	{\bf Обозначение: } $Id$ ($id$)

	{\bf Определение: } Умножением подстановок называется их последовательное применение (т.е. композиция отображений)

	{\bf Пример: }
	\[
		\begin{pmatrix}
			1 & 2 & 3 \\
			2 & 1 & 3
		\end{pmatrix}
		* 
		\begin{pmatrix}
			1 & 2 & 3 \\
			3 & 1 & 2
		\end{pmatrix}
		=
		\begin{pmatrix}
			1 & 2 & 3 \\
			1 & 3 & 2
		\end{pmatrix}
	\]

	(умножая слева направо)

	В циклах: $(12) * (132) = (1)(23) = (23)$

	{\bf Замечание: } Умножение подстановок некоммутативно

	{\bf Пример: } $(132)(12) = (13) \neq (23)$

	{\bf Замечание: } $Id$ -- нейтральный элемент по умножению

	{\bf Замечание: } Подстановку обратную к
	\[
		\delta = 
		\begin{pmatrix}
			1 & 2 & 3 & \ldots & n \\
			\delta(1) & \delta(2) & \delta(3) & \ldots & \delta(n)
		\end{pmatrix}
	\]
	можно получить как
	\[
		\delta^{-1} = 
		\begin{pmatrix}
			\delta(1) & \delta(2) & \delta(3) & \ldots & \delta(n) \\
			1 & 2 & 3 & \ldots & n \\
		\end{pmatrix}
	\]
	и отсортировав столбцы

	$\Rightarrow \delta * \delta^{-1} = \delta^{-1} * \delta = Id$

	{\bf Пример: }
	\[
		\delta = 
		\begin{pmatrix}
			1 & 2 & 3 \\
			2 & 1 & 3
		\end{pmatrix}
	\]
	\[
		\delta^{-1} = 
		\begin{pmatrix}
			1 & 2 & 3 \\
			2 & 1 & 3
		\end{pmatrix}
	\]

	{\bf Замечание: } $\forall$ Подстановку можно представить как произведение транспозиций

	$\square$ Запишем подстановку в циклах $(\alpha_1, \alpha_2, \ldots, \alpha_k)= (\alpha_k, \alpha_{k-1})(\alpha_{k-1}, \alpha_{k-2})\ldots(\alpha_2, \alpha_1)$

	- произведение $k-1$ транспозиций слева направо
	$\blacksquare$

	{\bf Замечание: } $sgn(\delta_1 * \delta_2)=sgn(\delta_1) * sgn(\delta_2)$

	\section{Определитель}
	
	{\bf Определение} Определителем или детерминантом квадратной матрицы $A$ порядка $n$ называется сумма $n!$ слагаемых следующего вида:

	$\det A = \sum_{\delta \in S_n}~sgn(\delta) * a_{1\delta(1)} * a_{2\delta(2)} * a_{3\delta(3)} * \ldots * a_{n\delta(n)}$, где 

	\[
		A = 
		\begin{pmatrix}
			a_{11} & a_{12} & a_{13} & \ldots & a_{1n} \\
			a_{21} & a_{22} & a_{23} & \ldots & a_{2n} \\
			\hdotsfor{5} \\
			a_{n1} & \hdotsfor{3} & a_{nn}
		\end{pmatrix}
	\]
	и
	\[
		\delta = 
		\begin{pmatrix}
			1 & 2 & 3 & \ldots & n \\
			\delta(1) & \delta(2) & \delta(3) & \ldots & \delta(n)
		\end{pmatrix}
	\]

	{\bf Обозначение: } $\det A$ или $|A|$

	{\bf Замечание: } По сути $\det A$ является суммой произведений элементов матрицы по одному из каждой строки и столбца (стоящие в разных строках и столбцах по всем способам так сделать (с учётом знака))

	{\bf Пример: } $n = 2 \Rightarrow n! = 2! = 2$

	\[
		\delta_1 = 
		\begin{pmatrix}
			1 & 2 \\
			1 & 2
		\end{pmatrix}
		= Id
	\]

	\[
		\delta_2 =
		\begin{pmatrix}
			1 & 2 \\
			2 & 1
		\end{pmatrix}
	\]

	$\Rightarrow$

	\[
		\begin{vmatrix}
			a_{11} & a_{12} \\
			a_{21} & a_{22}
		\end{vmatrix}
		= a_{11} * a_{22} - a_{21} * a_{12}
	\]

	% todo: пример для n = 3...

	{\bf Пример: } $n = 3 \Rightarrow n! = 3! = 6$

	\[
		A =
		\left[
			\begin{matrix}
				a_{11} & a_{12} & a_{13}, \\
				a_{21} & a_{22} & a_{23}, \\
				a_{31} & a_{32} & a_{33}
			\end{matrix}
			\left|
				\,
				\begin{matrix}
					a_{11} & a_{12} \\
					a_{21} & a_{22} \\
					a_{31} & a_{32}
				\end{matrix}
			\right.
		\right]
	\]


	$\det A = a_{11} a_{22} a_{33} + a_{12} a_{23} a_{31} + a_{13} a_{21} a_{32} - a_{31} a_{22} a_{13} - a_{22} a_{23} a_{11} - a_{32} a_{23} a_{11} - a_{33} a_{21} a_{12}$ - Правило Саррюса

	(Диагонали $\searrow$ идут с плюсом, диагонали с $\nearrow$ с минусом)

	\subsection{Свойства определителей}

	\textnumero1
	$\det A = \det A^T$
	($\Rightarrow$ Все свойства $\det$ верные для строк матрицы справедливы и для столбцов)

	$\square~B = A^T, b_{ij}=a_{ji}$
	Переставим $b_{i\delta_i}$ по возрастанию номеров столбцов

	$\det B = \sum_{\delta \in S_n} sgn(\delta) * b_{1\delta(1)} * b_{2\delta(2)} * \ldots * b_{n\delta(n)} = $

	Используем: 
	\[
		\delta = 
		\begin{pmatrix}
			1 & 2 & \ldots & n \\
			\delta(1) & \delta(2) & \ldots & \delta(n)
		\end{pmatrix}
		= 
		\begin{pmatrix}
			\delta(1)^{-1} & \delta(2)^{-1} & \ldots & \delta(n)^{-1} \\
			1 & 2 & \ldots & n
		\end{pmatrix}
	\]

	$sgn(\delta^{-1})=sgn(\delta)$ так как $sgn(\delta * \delta^{-1}) = sgn(\delta) * sgn(\delta^{-1}) = 1$

	$= \sum_{\delta \in S_n} sgn(\delta^{-1}) b_{\delta^{-1}_{}(1)1} b_{\delta^{-1}_{}(2)2} * \ldots * b_{\delta^{-1}_{}(n)n} = \sum_{\delta \in S_n} sgn(\delta^{-1}) * a_{1\delta^{-1}(1)} * a_{2\delta^{-1}(2)} * \ldots * a_{n\delta^{-1}(n)}$

	Переименуем $\tau = \delta^{-1}$

	$=\sum_{\tau \in S_n} sgn(\tau) * a_{1\tau_1} * \ldots * a_{n\tau_n} = \det A \,\blacksquare$


	\textnumero2
	Определитель линеен по строкам и столбцам т.е. пусть $A = (A_1, \ldots, A_n)$ -- матрица как набор строк

	Тогда:

	a) $\det(A_1, \ldots, A_i^\prime + A_i^{\prime\prime}, \ldots, A_n)$
	$=\det(A_1, \ldots, A_i^\prime, \ldots, A_n) + \det(A_1, \ldots, A_i^{\prime\prime}, \ldots, A_n)$

	б) $\det(A_1, \ldots, \alpha * A_i, \ldots, A_n) = \alpha * \det(A_1, \ldots, A_i, \ldots, A_n)$

	{\bf Пример: }
	\[
		\begin{vmatrix}
			1 & 3 + x \\
			2 & 7 + x
		\end{vmatrix}
		=
		\begin{vmatrix}
			1 & 3 \\
			2 & 7
		\end{vmatrix}
		+
		\begin{vmatrix}
			1 & x \\
			2 & x
		\end{vmatrix}
		=
		(1 * 7 - 3 * 2) + x * 
		\begin{vmatrix}
			1 & 1 \\
			2 & 1
		\end{vmatrix}
		= 1-x
	\]

	\input{./lectures/lectures_04_[24.09.2025].tex}
	\input{./lectures/lectures_05_[01.10.2025].tex}

	\input{./lectures/lectures_09_[05.10.2025].tex}

	{\bf Свойства сложения и умножения комплексных чисел}
	\[
		\forall z_1,z_2,z_3,z \in \C
	\]
	$$ \text{№1 } z_1 + z_2 = z_2 + z_1 \text{ - коммутативность} $$ 
	$$ \text{№2 } (z_1 + z_2) + z_3 = z_1 + (z_2 + z_3) \text{ - ассоциативность} $$ 
	$$ \text{№3 Существует нейтральный элемент по сложению - } O = (0, 0) \in \C:$$ 

	$$ z + 0 = 0 + z = z \forall z \in \C \text{ (Ноль)} $$

	$$ \text{№4 } \forall z \in \C \exists -z \in \C: z + (-z) = -z + z = 0 $$

	Т.е. $$ \forall z \exists $$ противопололжный или обратный элемент по сложению.

	$$ \text{№5 } z_1 * z_2 = z_2 * z_1 \text{- коммутативность} $$
	$$ \text{№6 } z_1 * z_2 * z_3 = z_1 * z_2 * z_3 \text{- ассоциативность} $$
	$$ \text{№7 } \exists e \in \C: z * e = e * z = z \forall z \text{ Существует единица - нейтральный элемент по умножению } (e = (1,0) = 1 \in \C )$$
	$$ \text{№8 } \forall z \neq 0 \exists z^{-1} \in \C: z * z^{-1} = z^{-1} * z = e \text{ Существует обратный элемент по умножению })$$
	$$ \text{№9 } z_1 * (z_2 + z_3) = z_1 * z_2 + z_1 * z_3 \text{ Дистрибутивность умножения относительно сложения} $$

	{\bf Замечание: } Эти 9 свойств-аксиом числового поля котороые и позволяют называть комплексные числа числами

	\subsection{Тригонометрическая форма}
	Здесь будет картинка

	Перейдём к полярным координатам (r, $ \phi$)

	\[
		\begin{cases}
			x = r * cos \phi \\
			y = r * sin \phi
		\end{cases}
	\]

	Тогда: $$ z = x + iy = r * cos \phi + ir * sin \phi = r(cos \phi + i sin \phi) = \text{тригонометрическая форма комплексного числа} $$ 

	{\bf Определение: } $$ \text{№1 } r = \sqrt{x^2+y^2} - \text{ модуль комплексного числа (угол между положительным направлением вещественной оси и числом} z = x + iy $$

	{\bf Обозначение: } $$ \phi = Arg z = \text{\{}arg z + 2 \pi k | k \in \Z\text{\}} \text{, где arg z - главное значение аргумента, его выбирают произвольно на полуинтервале длины} 2 \pi $$ 

	\subsection{Умножение и деление в тригонометрической форме}

	$$ \text{№1 Умножение. } z_1 * z_2 = r_1 * r_2(cos(\phi_1 + \phi_2) + i sin (\phi_1 + \phi_2) $$

	$$ \text{№2 Деление. } (z_2 \neq 0):  \frac{z_1}{z_2} = \frac{r_1}{r_2} (cos(\phi_1 - \phi_2) - i sin(\phi_1 - \phi_2)) $$

	$$ \square \text{Умножение: } z_1 * z_2 = r_1 * r_2 * (cos \phi_1 + i sin \phi_1) * (cos \phi_2 + i sin \phi_2) = r_1 * r_2 = (\underbrace{cos \phi_1 cos \phi_2 - sin \phi_1 * sin \phi_2}{cos(\phi_1 + \phi_2)} + i (\underbrace{cos \phi_1 * sin \phi_2 + sin \phi_1 * cos \phi_2}{sin(\phi_1 + \phi_2} = r_1 * r_2 * (cos(\phi_1 + \phi_2) + i sin(\phi_1 + \phi_2) \blacksquare$$ 

	{\bf Теорема (Муавра) } $$ \forall z \in \C \forall n \in \N z^n = r^n * cos(n \phi) + i sin(n * \phi) $$

	$$ \square \text{По индукции: } n = 1 \Rightarrow z = r(cos \phi + i sin \phi) = \text{ база индукции, верно} $$ 

	$$ \text{Пусть верно для} n = k \text{, покажем для} n = k + 1 \text{ (шаг индукции) } $$ 

	$$ z^{k+1} = z^k * z = r^k(cos k \phi + i sin k \phi) * r * (cos \phi + i sin \phi) = r^{k+1} (cos(k+1) \phi + i sin (k+1) \phi) \blacksquare $$ 

	\subsection{Извлечение комплексного корня: } 

	$$ \text{\bf Теорема: }  \forall \text{комплексноe числo } w \neq 0 (w \in \C) \text{ имеет ровно } n \text{ различных корней n-ой степени, т.е. таких чисел } z \in \C \text{, что } z^n = w $$

	$$ \square \text{Как их найти? } $$

	$$ \text{№1 Представим } w \text{ в тригонометрической форме: } w = \rho(cos \psi + i sin \psi) (\rho, \psi \text{ даны по условию}) $$

	$$ \text{№2 Ищем корни тоже в тригонометрической форме: } z = r(cos \phi + i sin \phi) $$

	$$ \text{№3 По условию } w = z^n \Rightarrow \text{ по формуле Муавра } z^n = r^n(cos n \phi + i sin n \phi) = \rho (cos \psi + i sin \psi) $$

	$$ \text{№4 Приравняем модули и аргументы: } $$

	\[
		z^n = w \Leftrightarrow
		\begin{cases}
			r^n = \rho (\rho, r \in \R_1 > 0) \\
			n \phi = \psi + 2 \pi k, k \in \Z
		\end{cases}
		\Leftrightarrow
		\begin{cases}
			r = \sqrt[n]{p} \\
			\psi = \frac{\psi + 2 \pi k}{n}, k \in \Z
		\end{cases}
	\]


	$$ \text{5. Заметим, что достаточно брать } k = \overline{0,n-1} \text{ т.к. начиная с } k = n \text{ корни станут повторяться } \Rightarrow \text{ Существует ровно n различных комплексных корней n-ой степени} $$

	\[
		\Rightarrow
		\sqrt[n]{w} = 
		\text{\{}
			\sqrt[n]{p}(cos(\frac{\psi + 2 \pi k}{n}) + i sin(\frac{\psi + 2 \pi k}{n}) 
			\text{|}
			k = \overline{0,n-1}
		\text{\}}
		\blacksquare
	\]

	\subsection{Геометрическая интерпретация}

	При $$ k = 0, \phi_0 = \frac{\psi}{n} $$ соответствует первому корню, $ \frac{2 pi}{n} $ - угол между соседними корнями - "шаг" 

	Здесь будет картиночка чуть позже

	Корни лежат в вершинах правильного n-угольника, вписанного в окружность радиуса $ \sqrt[n]{\rho} $ При этом 1-ый корень $ z_0 $ имеет аргумент $ \phi_0 = \frac{\psi}{n} $, а каждый следующий корень получен поворотом на угол $ \frac{2 \pi}{n} $ 

	{\bf Пример: }

	$ \sqrt[6]{1} $ - Найти все комлексные корни

	$$ 1 = 1 * (cos 0 + i sin 0) \Rightarrow \rho = 1, \psi = 0 (+ 2 \pi k) $$ 

	$$ \sqrt[6]{1} = \text{\{} \sqrt[6]{1} (cos \frac{0 + 2 \pi k}{6} + i sin \frac{0 + 2 \pi k}{6}, k = \overline{0,5} \text{\}} = \text{\{} cos \frac{\pi k}{3} + sin \frac{\pi k}{3}, k = \overline{0,5} \text{\}} $$

	$$ k = 0 \Rightarrow \phi_0 = 0 \text{, шаг } = \frac{2 \pi}{6} = \frac{\pi}{3} $$


	Здесь будет ещё один рисуночек 

	\subsection{Формула эйлера}

	\fbox{$ e^{i \phi} = cos \phi + i sin \phi $} $ \forall \phi \in \R $

	\[
		\begin{cases}
			cos \phi = Fe * e^{i \phi} \\
			sin \phi = Im * e^{i \phi}
		\end{cases}
	\]

	{\bf Замечание: } Тождество Эйлера: \fbox{$e^{i \pi} + 1 = 0$}

	{\bf Следствие: }
	\[
		\begin{cases}
			e^{i \phi} = cos \phi + i sin \phi \\
			e^{-i \phi} = cos \phi - i sin \phi 
		\end{cases}
		\Rightarrow
		\begin{cases}
			2 cos \phi = e^{i \phi} + e^{-i \phi} \\
			2 sin \phi = e^{i \phi} - e^{-i \phi}
		\end{cases}
		\Rightarrow
	\]

	\fbox{$ cos \phi = \frac{e^{i \phi} + e^{-i \phi}}{2} \text{ и } sin \phi = \frac{e^{i \phi} - e^{-i \phi}}{2} $}


	\subsection{Комплексные многочлены}

	{\bf Теорема: } "Основная" теорема алгебры

	Для любого многочлена с комплексными коэффициентами:

	$$ f(z) = a_n z^n + a_{n-1} z^{n-1} + \ldots + a_1 z + z_0, a_i \in \C, n \in \N, a_n \neq 0 $$

	Существует корень уравнения $ f(z) = 0 $, и этот корень всегда принадлежит $ \C $ 

	{\bf Замечание: } Это утверждение означает, что поле комплексных чисел алгебраически замкнуто. (у $\forall$ многочлена с коэффициентами из $\C$ есть корень из $\C$)

	Это не так для $\Q$ (например $x^2 - 2$ имеет корень $\sqrt{2} \notin Q) $ 

	и не так для $\R: x^2 + 1 $ имеет комплексные корни $\pm i \notin \R$ 

	{\bf Теорема (Безу) } Остаток от деления много члена $f(x)$ на х-с (что?) есть $f(c)$

	$\square$ Разделим $f(x)$ с остатком на х-с: $f(x) = (x-c) * Q(x) + R(x)$, где  deg $R(x) < deg(x-c) = 1 \Rightarrow $ остаток $R(x) = const \Rightarrow f(c) = (c-c) Q(c) + const \Rightarrow R(x) = f(c) \blacksquare$ 

	{\bf Определение: } Разложение многочлена $f(x) = g(x) * h(x)$ будем называть нетривиальным если 

	\[
		\begin{cases}
			\deg g < \deg f \\
			\deg h < \deg f
		\end{cases}
	\]
	,где $def f$ - степень многочлена

	
	{\bf Пример: } $(x^2+1)(x-1)$ - нетривиальный т.к. 
	\[
		\begin{cases}
			2 < 3 \\
			1 < 3
		\end{cases}
	\]

	{\bf Определение: } Многочлен называется приводимым если существует нетривиальное разложение $f(x) = g(x) \cdot h(x)$ и неприводимым в противном случае

	{\bf Утверждение: } Любой многочлен степени $n \in \N$ раскладывается в произведение неприводимых многочленов

	{\bf Частные случаи }

	\textnumero 1
	Многочлен над $\C$ (с коэффициентами из $\C$) степени $n$ всегда разлагается в произведение степеней линейных множителей

	$f(z) = a_n(z-z_1)^{\alpha_1}\ldots(z-z_k)^{\alpha_k}$, где $\alpha_1+\ldots+\alpha_k=n,\alpha_i \in \N$ - кратности корней, $z_i \in \C$ - корни многочлена, $a_n \in \C$ 

	\textnumero 2
	Разложение над $\R$

	{\bf Утверждение: } Если $z_0 \in \C$ является корнем нмогочлена с вещественными корнями, то и $z_0$ тоже является корнем этого многочлена

	$\square$ Пусть $z_0 \in \C$ - корень $f(x) \Rightarrow f(z_0) = 0$, т.е. $f(z_0) = a_n \cdot z_0^n + a_{n-1} z_0^{n-1}+ \ldots + a_1 z_0 + a_0 = 0$ 

	(*)

	Рассмотрим комплексное сопряжение равенства (*)
	$\overline{f(z_0)} = \overline{a_n} \cdot \overline{z_0^n} + \overline{a_{n-1}} \overline{z_0^{n-1}} + \ldots + \overline{a_1} \overline{z_0} + \overline{a_0} = \overline{0} = 0$

	Но $a_i \in \R$ по условию $\Rightarrow \overline{a_i} = a_i \Rightarrow a_n \cdot \overline{z_0^n} + a_{n-1} \overline{z_0^{n-1}} + \ldots + a_1 \overline{z_0} + a_0 \Leftrightarrow f(\overline{z_0}) = 0$, т.е. $\overline{z_0}$ - тоже корень $f(x) \blacksquare$

	{\bf Замечание: } $(z-z_0)(z-\overline{z_0}) = z^2 - z_0 \cdot z - \overline{z_0} \cdot z + z_0 \cdot \overline{z_0} = z^2 - (z_0 + \overline{z_0}) \cdot z + |z_0|^2 =$ (если $z_0 = x+iy, \overline{z_0} = x-iy \Rightarrow z_0 + \overline{z_0} = 2x = 2 Re~z_0 \in \R$)
	
	$ = z^2 - \underbrace{2 Re z_0}_{\in \R} \cdot z + \underbrace{|z_0|^2}_{\in \R} $ многочлен от $z$ с вещественными коэффициентами

	Соответственно, разложение на неприводимые множители над $R$ имеет вид:

	$f(x) = a_n(x-c_1)^{k_1} \ldots (x-c_s)^{k_s} \cdot (x^2 + p_1x+q_1)^{l_1} \ldots (x^2 + p_tx+q_t)^{l_t}$, где 

	$a_n, c_i, p_j, q_j \in \R, k_i, l_j, \in \N$

	Здесь квадратичные сомножители не имеют вещественных корней, а обладают парой комплексных сопряжеённых коренй с ненулевой мнимой частью (дискриминант $D < 0$) 

	{\bf Теорема Виета}

	Пусть $c_1, \ldots, c_n$ - корни многочлена степени $n$

	$f(x) = x^n + a_1 x^{n-1} + a_2 x^{n-2} + \ldots + a_n$ 

	\[
		\begin{cases}
			a_1 = -(c_1+\ldots+c_n) \\
			a_2 = c_1 c_2 + c_1 c_3 + \ldots + c_{n-1} c_n \\
			\vdots \\
			a_n = (-1)^n \cdot c_1 \ldots c_n 
		\end{cases}
	\]

	Т.е. $(-1)^j a_j$ равно сумме всех возможных произведений $j$ корней

	{\bf Пример $(n = 3)$: } 

	$f(x) = x^3 + a x^2 + bx + c$ 

	\[
		\begin{cases}
			c_1 + c_2 + c_3 = -a \\
			c_1 \cdot c_2 + c_1 \cdot  c_3 + c_2 \cdot c_3 = b \\
			c_1 \cdot c_2 \cdot c_3 = -c
		\end{cases}
	\]

	На этом комплексное заканчивается

	\section{Аналитическая геометрия} 

	{\bf Определение: } Угол между векторами $a$ и $b$.

	Отложим $a$ и $b$ из одной точки, назовём углом $\phi$ между векторами $a$ и $b$ наименьший из двух плоских углов которые они образуют, т.е. $\phi \in [0, \pi]$

	{\bf Определение: } Если $\phi = \frac{\pi}{2}$, то векторы называются ортогональными. Нулевой вектор ортогонален любому вектору 

	{\bf Определение:} Ортогональная проекция вектора $\overline{a}$ на нправлении вектора $\overline{b} (\overline{b} \neq 0)$

	\textnumero 1 скалярная - число ${\text{Пр}_{\overline{b}}}^{\overline{a}}$ = $|a| \cdot cos \phi$, где $\phi$ - угол между $\overline{a}$ и $\overline{b}$

	\textnumero 2 векторная - вектор $\overline{{\text{Пр}_{\overline{b}}}^{\overline{a}}}$ = $|\overline{a}| \cdot cos(\phi) \cdot \frac{\overline{b}}{|\overline{b}|}$

	{\bf Определение: } Векторы $\overline{a}$ и $\overline{b}$ называют коллинеарными, если они лежат на одной либо параллельных прямых

	{\bf Замечание: } Нулевой вектор коллинеарен любому вектору

	{\bf Замечание: } Коллинеарные векторы $a$ и $b$ могут быть сонаправленными ($a \uparrow \uparrow b$) или противоположно направленными ($\overline{a} \uparrow \downarrow \overline{b}$)

	{\bf Определение: } Векторы $\overline{a}, \overline{b}, \overline{c}$ называются компланарными, если они лежат на прямых, параллельных фиксированной плоскости (т.е. если их отложить из одной точки, они лежат в одной плоскости)

	{\bf Определение: } Скалярное произведение векторов - это функция, ставящая в соответствие паре векторов число, удовлетворяющее следующим свойствам: Для любых векторов $a,b,c$ и для любых $\alpha, \beta \in \R$ 

	1. $(a,b) = (b,a)$ - симметричность

	2. $(\alpha a + \beta b, c) = \alpha(ac) + \beta(b,c)$ - линейность

	3. $(a,a) \geq 0$ и $(a, a) = 0 \Leftrightarrow a = 0$ - положительная определённость

	{\bf Замечание: } В $V_3$ скалярное произведение задаётся, как $(a,b) = |a| \cdot |b| \cdot cos \widehat{a,b}$

	% {\bf Замечание: } Скалярное произведение $(a,b) = 0 \Leftrigharrow a$ и $b$ ортогональны

	{\bf Определение: } Базис - это упорядоченный набор векторов $a_1,\ldots,a_k$ такой что

	1. $a_1,\ldots, a_k$ - линейно независимы
	2. Любой вектор линейно зависим через $a_1, \ldots, a_k$



	{\bf Определеие: } Матрицей Грама базиса $\overline{e_1}, \overline{e_2}, \overline{e_3}$ в $V_3$ называется матрица, состоящая из попарных скалярных произведений этих векторов

	\[
		\text{Г}
		=
		\begin{pmatrix}
			(e_1, e_1) & (e_1, e_2) & (e_1, e_3) \\
			(e_2, e_1) & (e_2, e_2) & (e_2, e_3) \\
			(e_3, e_1) & (e_3, e_2) & (e_3, e_3) \\
		\end{pmatrix}
	\]

	{\bf Утверждение: } Пусть $\overline{a} = a_1 \overline{e_1} + a_2 \overline{e_2} + a_3 \overline{e_3} $

	$\overline{b} = b_1 \overline{e_1} + b_2 \overline{e_2} + b_3 \overline{e_3}$

	- разложение векторов $a$ и $b$ по базису $e$


	\[
		\text{Тогда \fbox{($\overline{a}, \overline{b}$)}}
		=
		\begin{pmatrix}
			a_1 & a_2 & a_3
		\end{pmatrix}
		\begin{pmatrix}
			(e_1, e_1) & (e_1, e_2) & (e_1, e_3) \\
			(e_2, e_1) & (e_2, e_2) & (e_2, e_3) \\
			(e_3, e_1) & (e_3, e_2) & (e_3, e_3) \\
		\end{pmatrix}
		\begin{pmatrix}
			b_1 \\
			b_2 \\
			b_3
		\end{pmatrix}
		= a_e^T \text{Г} b_e 
	\]
	, где
	\[
		a_e
		=
		\begin{pmatrix}
			a_1 \\
			a_2 \\
			a_3 
		\end{pmatrix}
		\text{,}
		b_e
		=
		\begin{pmatrix}
			b_1 \\
			b_2 \\
			b_3 
		\end{pmatrix}
	\]
	- координаты векторов $a$ и $b$ в базисе $e_1, e_2, e_3$

	$\square$
	$\blacksquare$

	{\bf Определение: } Упорядоченную тройку векторов $\overline{a}, \overline{b}, \overline{c}$ называют правой тройкой, если со стороны вектора $\overline{c}$ (с конца) кратчайший поворот от вектора $\overline{a}$ до вектора $\overline{b}$ происходит против часовой стрелки

	В противном случае тройка векторов называется {\bf левой} (если поворот по часовой стрелке) 

	{\bf Определение: } Вектор $\overline{c}$ называется векторным произведением векторов $a$ и $b$ если выполняются 3 свойства

	1. $|c| = |a| \cdot |b| \cdot sin(\phi)$, где $\phi$ - угол между $\overline{a}$ и $\overline{b}$ 

	2. $\overline{c} \bot \overline{a}, c \bot b$ (ортогонален $a$ и $b$)

	3. $a,b,c$ - правая тройка
	
	\input{./lectures/lectures_13_[03.12.2025].tex}
	
\setlength{\arrayrulewidth}{0.8pt}
\begin{tabular}{|p{5cm}|p{5cm}|p{5cm}|}
\hline
\textbf{Скалярное произв.} & \textbf{Векторное произв.} & \textbf{Смешанное произв.} \\
\hline
(.,.) : $V_3 \times V_3 \rightarrow \mathbb{R}$ (число). \par
Обозн.: $\langle x,y \rangle$, $x \cdot y$. &
$[.,.] : V_3 \times V_3 \rightarrow V$ (вектор). \par
Обозн.: $[x,y]$, $x \times y$. &
\\
\hline
Опр. $(a,b)=|a||b|\cos(\hat{a},\hat{b})$ в $V_3$. &
Опр. $c=[a,b]$, $a,b \in V_3$. \par
1) $|c| = |a||b|\sin(\hat{a},\hat{b})$. &
\\
\hline
Геом. св-ва: \par
$(a,b)=0 \Leftrightarrow a \perp b$ \par
(a ортогонален b). &
 &
\\
\hline
Алгебр. св-ва: \par
1) Линейность \par
2) Симметричность \par
3) Полож.опр-ть \par
$(a,a)\ge 0$ и $(a,a)=0 \Leftrightarrow a=0$. &
 &
\\
\hline
Вычисл. в координатах: \par
В ОНБ: $(a,b)=a_1b_1 + a_2b_2 + a_3b_3$ \par
В произв. б-се: $(a,b)=a^T \Gamma b$. &
 &
\\
\hline
Геом. приложения: \par
$|a| = \sqrt{(a,a)} = \sqrt{a^2}$ \par
$\cos(a^\circ) = \frac{(a,b)}{|a||b|}$ \par
Про a = $\frac{(a,b)}{(b,b)} b$. &
 &
\\
\hline
\end{tabular}

	{\bf Утверждение 1} Векторы $a$ и $b$ коллинеарны ($a \parallel b$)
	$\Leftrightarrow [a,b] = 0$ 

	$\square \Rightarrow$ Пусть $a \parallel b \Rightarrow$ либо $sin \phi = 0 (\phi = 0 или \phi = \pi) либо$

	\[
		\begin{cases}
			a = 0 \\
			b = 0
		\end{cases}
		\text{хотя бы один}
		\Rightarrow
		|[a,b]|=0 \Rightarrow [a,b]=0
	\]

	$\Leftarrow$ Пусть $[a,b]=0$. Если один из них $= 0 \Rightarrow$ коллинеарны по определению.
	Если оба $\neq 0$,
	то из $|[a,b]|=0 \Rightarrow sin \phi = 0 \Rightarrow a \parallel b \blacksquare$

	{\bf Определение: } Базис $e_1,e_2,e_3$ называется правым если тройка векторов $e_1,e_2,e_3$ правая, и левым если тройка левая.

	Далее будем полагать, что рассматривается правый \underline{ортонормированный базис} $\{i,j,k\}$

	{\bf Утверждение: } 

	\[
		[a,b]=
		\begin{vmatrix}
			\overline{i} & \overline{j} & \overline{k} \\
			a_x & a_y & a_z \\
			b_x & b_y & b_z 
		\end{vmatrix}
		= \overline{i}(a_y b_z - a_x b_y) + \overline{j} (a_z b_x - a_x b_z) + \overline{k} (a_x b_y - a_y b_z)
	\]
	, где $\overline{a} = a_x \overline{i} + a_y \overline{j} + a_z \overline{k} $  и $\overline{b} = b_x \overline{i} + b_y \overline{j} + b_z \overline{k} $

	$\square [a,b] = [a_x \overline{i} + a_y \overline{j} + a_z \overline{k}, b_x \overline{i} + b_y \overline{j} + b_z \overline{k}] $ 
	$= a_x b_x \underbrace{[i, i]}_{=0} + a_x b_y \underbrace{[i, j]}_{=k} + a_x b_z + a_x b_z \underbrace{[i, k]}_{=-j} + a_y b_x \underbrace{[j, i]}_{=-k} + a_y b_y \underbrace{[j, j]}_{=0} + a_y b_z \underbrace{[j, k]}_{=i} + a_z b_x \underbrace{[k, i]}_{=j} + a_z b_j \underbrace{[k, j]}_{=-i} + a_z b_z \underbrace{[k, k]}_{=0}$
	$ = i (a_y b_z - a_z b_y) + j (a_z b_x - a_x b_z) + k (a_x b_y - a_y b_x)$
	\[
		=
		\begin{vmatrix}
			\overline{i} & \overline{j} & \overline{k} \\
			a_x & a_y & a_z \\
			b_x & b_y & b_z 
		\end{vmatrix}
	\]

	$\blacksquare$

	\section{Смешанное произведение}

	{\bf Смешанным произведением } \underline{Смешанным произведением} векторов $a,b$ и $c$ называется число ($[a,b],c$)

	{\bf Обозначение: } $<a,b,c>$

	{\bf Утвереждение: } Пусть $V$ - объём параллелепипеда построенного на неколлинеарных векторах $a,b,c$ 

	Тогда:
	\[
		<a,b,c>=
		\begin{cases}
			V \text{если a,b,c - правая тройка} \\
			-V \text{если a,b,c - левая тройка}
		\end{cases}
	\]

	$|[a,b]| = S$ параллелограмма на векторах $a$ и $b$

	Обозначим за $\overline{l}$ единичный
	ортогональный $\overline{l} \uparrow \uparrow [a,b] \Rightarrow [a,b] = S \cdot \overline{l} \Rightarrow <a,b,c> = ([a,b],c) = (S \cdot \overline{l}, \overline{c}) = S * \cos \phi$, где $\phi$ - угол между $c$ 

	и $\phi \in (\frac{\pi}{2}; \pi]$, если $a,b,c$ - левая $\Rightarrow \cos \phi < 0 \Rightarrow |c| * \cos \phi = -h (\phi \neq \frac{\pi}{2} $так как $a,b,c$ не компланарны $)$

	Таким образом

	\[
		<a,b,c> = 
		\begin{cases}
			S \cdot h = V \text{если a,b,c - правая тройка} \\
			-S \cdot h = -V \text{если a,b,c - левая тройка} \\
		\end{cases}
	\]

	$\blacksquare$


	\underline{Следствие 1}
	$a,b,c$ - компанарны $\Leftrightarrow <a,b,c> = 0$ 

	$\square \Rightarrow$ Пусть $a,b,c$ - компланарны. Если $a \parallel b$, то $[a,b] = 0 \Rightarrow <a,b,c> = 0$

	Если $a \nparallel b$, то $[a,b] \perp c$ (так как $c$ компланарен $a$ и $b$) $\Rightarrow <a,b,c> = 0$ 

	\section{Прямоугольная декартова система координат}

	{\bf Определение: } Прямоугольной декартовой системой координат (ПДСК) будем называть пару состоящуюю из 

\end{document}

\section{Транспонирование}

{\bf Определение:}
Транспонирование --- операция, переводящая все строки в столбцы с сохранением порядка, то есть:
$$[A^T]_{ij} = [A]_{ji}\ \forall i = \overline{1, m}, j = \overline{1, n}$$ 

{\bf Обозначение:} $A^T$

$A_{m\times n} \xrightarrow[]{T} (A^T)_{n \times m}$ 

\subsection{Свойства транспонирования}

\begin{itemize}
    \item[1.] $(A^T)^T = A$
    \item[2.] $(A + B)^T = A^T + B^T$
    \item[3.] $(\lambda \cdot A)^T = \lambda \cdot A^T$
    \item[4.] $(A \cdot B)^T = B^T \cdot A^T$
    
    $\square$ 
    Пусть матрицы $A \in M_{mn}(\R), B \in M_{nk}(\R)$

    $[(A \cdot B)^T]_{ij} \overbrace{=}^{\text{по определению Т}} [A \cdot B]_{ji} \overbrace{=}^{\text{по определению произведения матриц}} \Sum{r=1}{n} A_{jr} \cdot B_{ri} =$ 
    
    $\overbrace{=}^{\text{по коммутативности произведения}} \Sum{r=1}{n} B_{ri} \cdot A_{jr} = \Sum{r=1}{n} B^T_{ir} \cdot A^T_{rj} = [B^T \cdot A^T]_{ij}\ \forall i = \overline{1, k}, j = \overline{1, m}$
    $\blacksquare$
\end{itemize}

\subsection{Элементарное преобразование строк(столбцов)}
{\bf Определение:} --- это 3 следующие операции:

\begin{itemize}
    \item[1.] Перестановка двух строк в матрице
    
    $(i) \Leftrightarrow (k) \text{(i-ая строка меняется местами с k-ой)}$
    \item[2.] Умножение строки на ненулевое число
    
    $\lambda \in \R,\ \lambda \neq 0, (i) \Rightarrow \lambda \cdot (i), \lambda \neq 0$
    \item[3.] Прибавление к i-ой строке другой k-ой строки с коэффициентом $\lambda$, $\lambda \in \R$
    
    $(i) \Rightarrow (i) \cdot \lambda (k),\ i \neq k$
\end{itemize}

{\bf Замечание 1:} Все элементарные преобразования обратимы

{\bf Замечание 2:} Каждое элементарное преобразование строк матрицы можно трактовать как умножение на матрицу слева специального вида (квадратную). Эта матрица получается применением к единичной матрице того же самого элементарного преобразования

{\bf Пример:} (1) элементарное преобразование к $E$

\[
    \begin{pmatrix}
        1 &  0 \\
        0 &  1 \\
    \end{pmatrix}
    \xrightarrow[]{(2) - 2(1)}
    \begin{pmatrix}
        1 &  0 \\
        -2 &  1 \\
    \end{pmatrix}
    \text{--- матрица элементарного преобразования}
\]

\[
    \begin{pmatrix}
        1 &  0 \\
        -2 &  1 \\
    \end{pmatrix}
    A
    = 
    \begin{pmatrix}
        1 &  0 \\
        -2 &  1 \\
    \end{pmatrix}
    \begin{pmatrix}
        1 & 2 &  3 \\
        2 & 1 &  3 \\
    \end{pmatrix}
    = 
    \begin{pmatrix}
        1 & 2 &  3 \\
        0 & -3 &  -3 \\
    \end{pmatrix}
\]

{\bf Замечание 3:} Преобразования можно применять и к столбцам. Это будет соответствовать умножению справа на матрицу специального вида

{\bf Определение:} Матрица имеет:
\begin{itemize}
    \item[1)] ступенчатый вид, если номера первых ненулевых элементов всех строк последовательно возрастают, такие элементы называются ведущими (нижние ведущие элементы находятся правее, чем верхние), а все нулевые строки стоят внизу
    \item[2)] канонический вид (улучшенные ступенчатый), если матрица имеет ступенчатый вид, в которой все ведущие элементы равны 1, и в любом столбце с ведущими элементами все остальные равны 0
\end{itemize}

{\bf Теорема о методе Гаусса:} Любую конечную матрицу можно привести к ступенчатому и каноническому виду элементарными преобразованиями строк.

$\square$
Предъявим алгоритм. Берем матрицу $m \times n$ и двигаемся из левого верхнего угла. 
\[
\begin{pmatrix}
    a_{11} & a_{12} & a_{13} & \ldots & a_{1n} \\
    a_{21} & a_{22} &  &  &   \\
    a_{31} &  & \ldots &  &   \\
    \ldots &  &  & \ldots &   \\
    a_{m1} &  &  &  & a_{mn}  \\
\end{pmatrix}
\]

\fbox{п. 1} 
Если текущий элемент равен 0, тогда переходим к \fbox{п. 2}, иначе объявляем теущий элемент ведущим. Прибавляем текущую строку к остальным так, чтобы элементы ниже ведущего (и выше в случае канонического вида) обратились в 0. Пусть $a_{ij}$ --- текущий элемент. Тогда для $k \neq i$ строки берем $\lambda = -\frac{a_{kj}}{a_{ij}}$, где $a_{ij} \neq 0$ и $(k) \rightarrow (k) + \lambda (i)$ (для канонического вида $(i) \rightarrow \frac{1}{a_{ij}}(i)$, чтобы получить 1 на месте ведущего элемента). 

Выбираем новый текущий элемент, смещаясь в матрице на один столбец вправо и на одну строку вниз $\Rightarrow$ следующий шаг, повторяем \fbox{п. 1}, если это невозможно \text{stop}

\fbox{п. 2}
Если текущий элемент равен нулю, то просматриваем все элементы под ним. Если среди них нет не равных 0, то переходим к \fbox{п. 3}. Иначе если в k-ой строке ненулевой элемент нашелся (под текущим), то $(i) \leftrightarrow (k)$

\fbox{п. 3}
Если текущий элемент и все под ним равны 0, то меняем текущий столбец смещаясь на один столбец вправо. Если возможно, то \fbox{п. 1}, иначе \text{stop}

Так как матрица имеет конечные размеры, а за 1 итерацию смещается вправо на 1 столбец --- процесс преобразования к ступенчатому (каноническому) виду закончится не более чем за $n$ шагов

\section{Система линейных алгебраических уравнений (СЛАУ)}

Покоординатная запись СЛАУ:
\[
    \begin{dcases}
        a_{11} x_{1} + a_{12} x_2 + \ldots + a_{1n} x_n = b_1\\
        a_{21} x_{1} + a_{22} x_2 + \ldots + a_{2n} x_n = b_2\\
        \ldots\\
        a_{m1} x_{1} + a_{m2} x_2 + \ldots + a_{mn} x_n = b_m
    \end{dcases}
    \Rightarrow
    Ax = b,
    \text{где }
    A =
    \begin{pmatrix}
        a_{11} & \ldots & a_{1n}   \\
        \ldots &  & \ldots  \\
        a_{m1}  & \ldots &  a_{mn}  \\
    \end{pmatrix}
    \text{, }
\]

\[
    x = 
    \begin{pmatrix}
        x_1 \\
        \ldots \\
        x_n \\
    \end{pmatrix}
    \text{--- столбец неизвестных, }
    b = 
    \begin{pmatrix}
        b_1 \\
        \ldots \\
        b_m \\
    \end{pmatrix}
    \text{--- столбец свободных членов}
\]

\[
    (A|B)_{m \times (n+1)} =
    \begin{pmatrix}
        a_{11} & \ldots & a_{1n} & \vline & b_1   \\
        \ldots & \ldots & \ldots & \vline & \ldots   \\
        a_{m1} & \ldots & a_{mn} & \vline & b_m   \\
    \end{pmatrix}
    \text{--- элементарными преобр. к каноническому виду}
\]

{\bf Замечание:}
Элементарные преобразования матрицы не меняют множество решений СЛАУ

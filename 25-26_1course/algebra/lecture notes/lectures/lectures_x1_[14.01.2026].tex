\section{Теория Групп. Начало общей алгебры}

\subsection{Алгебраические структуры}

\[
    \fbox{множество} \xrightarrow[\text{опер.}]{} \underbrace{\fbox{группоид} \xrightarrow[\text{ассоц. оп.}]{} \fbox{полугруппа} \xrightarrow[\text{нейтр. эл.}]{}  \fbox{моноид} \xrightarrow[\text{любой эл-т обратим}]{} \fbox{группа}}_{\text{алгебраические структуры}}
\]

\begin{definition}[Эквивалентное определение группы]
    Множество $G$ с бинарной операцией $\times$ называется группой, если:
\begin{itemize}
    \item[1)] $\forall a, b, c \in G\ (a \times b) \times c = a \times (b \times c)$ --- ассоциативность
    \item[2)] $\exists e \in G\ \forall x \in G\ e \times x = x \times e = x$ (существует нейтральный элемент)
    \item[3)] $\forall x \in G\ \exists x^{-1} \in G : x \times x^{-1} = x^{-1} \times x = e $ (любой элемент обратим)
\end{itemize}
\end{definition}

\begin{example}
    Множество всех невырожденных матриц порядка $n$ с операциями матричного умножения

    Обозначение: $GL_n(\R)$ --- общая линейная группа
\end{example}

\begin{example}
    Множество подстановок длины $n$ с операцией композиции образует группу $S_n$

    Она называется симметрической группой
\end{example}

\begin{definition}
    Порядком группы $G$ называется число элементов в ней (мощность группы)

    Обозначение: $|G|$
\end{definition}

\begin{example}
\[
    |GL_n(\R)| = \infty,\ |S_n| = n!
\]
\end{example}

\begin{definition}
    Группа с коммутативной бинарной операцией называется абелевой
\end{definition}

\begin{example}
\[
    (\Z, +) \text{ --- абелева группа}
\]
\end{example}

\begin{definition}
    Подмножество $H \subseteq G$ называется подгруппой в группе $G$, если: 
\begin{itemize}
    \item[1)] $e \in H$
    \item[2)] $\forall h_1, h_2 \in H : h_1 \times h_2 \in H$ (замкнутость относительно бинарной операции)
    \item[3)] $\forall h \in H \Rightarrow h^{-1} \in H$ (замкнутость относительно взятия обратного элемента)
\end{itemize}
Другими словами, само $H$ является группой относительно операций в $(G, \times)$, (т.е групповой операции)
\end{definition}

\begin{example}
\begin{itemize}
    \item[1)] $SL_n(\R) = \{A \in GL_n(\R) | \det A = 1\}$ --- специальная линейная подгруппа
    (т.к $\det(E) = 1, \det (A \cdot B) = \det A \cdot \det B = 1 \cdot 1 = 1, \det A^{-1} = \frac{1}{\det A} = \frac{1}{1} = 1 $)

    \item[2)] $A_n = \{\delta \in S_n | sgn\ \delta = 1\}$ --- подгруппа, четных подстановок длины n. Она называется знакопеременной
    
    ${\prod_{i < j}} (x_i - x_j)$ --- знакопеременный многочлен, $\det $ Вандермонда. Подстановки из $A_n$ не меняют его знак 
\end{itemize}
\end{example}

\begin{definition}
    Подгруппа $H$ называется собственной подгруппой в $G$, если $H \neq \{e\}, H \neq G, (SL_n(\R)_nA_n$ --- собств. подгруппы в $GL_n(\R)$ и $S_n$ соотв.)
\end{definition}

\[
    |A_n| = \frac{n!}{2}
\]

\begin{lemma}[Критерий подгруппы]
Пусть $H$ --- подмножество в группе G.

$H$ является подгруппой $\Leftrightarrow \forall h_1, h_2 \in H \Rightarrow h_1 \times h_2^{-1} \in H $
\end{lemma}

$\square$ Необходимость ($\Rightarrow$). Дано, что $H$-подгруппа. Тогда $\forall h_1, h_2 \in H \Rightarrow h_2^{-1} \in H $ по определению, $\Rightarrow$ и произведение двух элементов $h_1$ и $h_2^{-1}$ из $H$ тоже лежит в $H$

Достаточность ($\Leftarrow$) 

Дано: $\forall h_1, h_2 \in H \Rightarrow h_1 \times h_2^{-1} \in H $

Доказать: 
\begin{itemize}
    \item[1)] $e \in H$
    \item[2)] $h_1 \times h_2 \in H\ \forall h_1, h_2 \in H$
    \item[3)] $h^{-1} \in H\ \forall h \in H$
\end{itemize}

Поскольку $h_1$ и $h_2$ произвольны, возьмем $h_2 = h_1 \Rightarrow h_1 \times h_1^{-1} \in H$, то есть $e \in H \Rightarrow$ выполнено 1)

Возьмем $h_1 = e \Rightarrow \forall h_2 \in H \Rightarrow h_2^{-1} \in H$, т.е выполнено 3)

Пусть $h_1, h_2 \in H \Rightarrow h_2^{-1} \in H$. Следовательно, $h_1 \times (h_2^{-1})^{-1} \in H$, то есть $h_1 \times h_2 \in H \Rightarrow$ выполнено 2) $\blacksquare$

\subsection{Гомоморфизмы}

\begin{definition}
    Отображение, $f : G_1 \rightarrow G_2$. (Здесь $(G_1, \times)$ и $(G_2, \circ)$ --- группы) называется гомоморфизмом групп, если \fbox{$\forall a, b \in G \Rightarrow f(a \times b) = f(a) \circ f(b)$}. Говорят, что $f$ \emph{уважает} операцию.
\end{definition}

\begin{example}
\begin{itemize}
    \item[1)] $\det : GL_n (\R) \rightarrow \R^{\times} = (\R\setminus\{0\}, \cdot)$ --- группа 
    
    Т.е $A \xrightarrow[]{\det } \det A \in \R^{\times}\ \forall A \in GL_n(\R)$

    $(\det A \cdot B = \det A \cdot \det B \Rightarrow \det$ --- гомоморфизм групп)

    \item[2)] $G_1 = (\R_+, \cdot) \text{ --- группа} \{x \in \R | x > 0\}, G_2 = (\R, +)$

    Рассмотрим $f = \ln x, f : \R_+ \rightarrow \R$

    $\forall a, b \in G_1 (a > 0, b > 0) \Rightarrow \ln (a \cdot b) = \ln a + \ln b \in G_2$ --- уважает операцию $\Rightarrow$ логарифм --- гомоморфизм групп.
\end{itemize}
\end{example}

\begin{definition}
    Инъективный гомоморфизм называется мономорфизмом.
\end{definition}

\begin{definition}
    Сюръективный гомоморфизм называется эпиморфизмом.
\end{definition}

\begin{definition}
    Биективный гомоморфизм называется изоморфизмом.
\end{definition}

\begin{example}
\begin{itemize}
    \item[1)] $\det $ --- эпиморфизм
    \item[2)] $\ln x$ --- изоморфизм
\end{itemize}
\end{example}

\begin{remark}[Замечание]
    Если две группы изоморфны, то они тождественно неразличимы с точки зрения алгебры.

    Обозначение $G_1 \cong G_2$ ($G_1$ и $G_2$ изоморфны)
\end{remark}

\begin{example}
\begin{itemize}
    \item[1)] $(\R_+, \cdot) \cong (\R, +)$
    \item[2)] $G_1$ --- группа движений (т.е отображений, переводящих фигуру в себя) правильного треугольника на плоскости:
    
    - повороты ($p_0$ -- тождественное отображение, $p_{\frac{2\pi}{3}}, p_{\frac{4\pi}{3}} = p_{-\frac{2\pi}{3}})$

    - отражения относительно осей симметрии $(s_1, s_2, s_3)$

    \emph{Закодируем} движения фигуры подстановками из $S_{3}$)

    $G_2 = S_3 = \{Id, (1 2), (1 3), (2 3), (1 2 3), (1 3 2)\}$

    $p_0 \sim Id$

    $p_{\frac{2\pi}{3}} \sim (1 2 3)$

    $p_{\frac{4\pi}{3}} \sim (1 2 3)^2 = (1 3 2)$

    $s_1 \sim (2 3)$

    $s_2 \sim (1 3)$

    $s_3 \sim (1 2)$

    Построили гомоморфизм, даже изоморфизм.
\end{itemize}
\end{example}

\begin{definition}
    Группа диэдра --- группа движений (симметрий) правильного $n$-угольника на плоскости, включающая вращения и отражения, переводящие фигуру в себя. 

    Обозначение: $D_n$, $|D_n| = 2n$, так как там есть $n$ вращений вокруг центра и $n$ отражений (осевых симметрий)
\end{definition}

\begin{remark}[Замечание]
    $D_3 \cong S_3$ (пр.2)
\end{remark}

\subsection{Таблица Кэли}
\begin{definition}
    Таблица Кэли алгебраической структуры (в частности, группы) называется матрица:
\end{definition}

\begin{figure}[h]
  \centering
  \includegraphics[width=0.5\textwidth]{lectures/files/jkjkjk12-29-56.png}
  \caption{таблица умножения}
  \label{fig:jkjkjk12-29-56.png}
\end{figure}

\begin{remark}
    Если таблица Кэли симметрична, то группа абелева
\end{remark}
 
\begin{remark}
    Если $G$-группа, то в ее таблице Кэли каждый элемент встречается только 1 раз в каждой строке и каждом столбце
\end{remark}

$\square$ Так как если $g_1 \times g_2$

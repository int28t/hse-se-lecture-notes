\section{test}

{\bf Замечание } Для однородной СЛАУ $Ax = 0$ теорема Кронекера-К. выполняется всегда: $Rg A = Rg(A | O)$ - всегда есть решения

{\bf Однородные СЛАУ }

Рассмотрим ОСЛАУ $Ax = 0$ (где $ A \in M_{mn}(\R)$ 

{\bf Определение: } Любые $n-r$ линейно независимых столбцов (где $n-r$ - число неизвестных, $r = Rg A$) является решениями однородных СЛАУ $Ax = 0$, называют фундаментальной системой решений (ФСР) однородной СЛАУ $Ax = 0$

{\bf Теорема: } (о существовании ФСР):
Рассмотрим ОСЛАУ $Ax = 0$ У неё $\exists$ $ k = n-r$ (где $n$ - число неизвестных), $r=RgA$) линейно независимых решений

$\square$ Предположим, что БМ, расположен в левом верхнем углу матрицы $A$. Пусть $Rg A = R$

A = рисунок вставим потом1

По теореме о БМ строки $A_{r+1},\ldots,A_m$ линейно выражаются через базисные строки $A_1,\ldots,A_r$

Сделаем элементарные преобразования: $A_{r+1} -> A_{r+1}-\alpha_{1} A_{1}-\ldots,\alpha_r * A_r$

$A_m -> A_m - \mu_1 A_1 - \ldots - \mu_r A_r$

Получим матрицу $A^{\prime}$, у которой последние $m-r$ строк равны 0

A = рисунок 2

Элементарные преобразования строк матрицы $A$ (А следовательно и строк расширенной матрицы $(A|O)$) соответствуют эквивалентным преобразованиям уравнений исходной ОСЛАУ $Ax = 0 \Rightarrow$ исходная ОСЛАУ $Ax = 0$ эквивалентна СЛАУ $A^{\prime} x = 0$

\[
	\begin{cases}
		a_{11}x_1+\ldots+a_{1r}x_r + a_{1r+1}x{r+1}+\ldots+a_{11}x_{n}=0 \\
		\ldots \\
		a_{r1}x_1+\ldots+a_{rr}x_r + a_{rr+1}x{r+1}+\ldots+a_{rn}x_{n}=0
	\end{cases}
\]

Будем называть переменные, отвечающие базисны столбцам, главными (или базисными), а все остальные переменные свободными

В(*) $x_1,\ldots,x_r$ - главные (их $r=RgA$), $x_{r+1},\ldots,x_{n}$- свободные (их $n-r$ штук)

Перепишем (*) так чтобы слева остались только главные переменные, а справа свободные

**
\[
	\begin{cases}
		a_{11}x_1+\ldots+a_{1r}x_r = -a_{1r+1}x{r+1}-\ldots-a_{11}x_{n}=0 \\
		\ldots \\
		a_{r1}x_1+\ldots+a_{rr}x_r = -a_{rr+1}x{r+1}-\ldots-a_{rn}x_{n}=0
	\end{cases}
\]


Присвоим свободные переменные $x_{r+1},\ldots,x_n$ след. наборы значений

рисунок3

Для каждого набора решим СЛАУ (**) 

Она всегда имеет решение, т.к. это СЛАУ с квадратной невырожденной матрицей (её $\det = M \neq 0$, т.к. это БМ) (решения можно найти например по формулам Крамера)

Получим следующие решения:
рисунок4

$\Rightarrow$ столбцы рисунок5


являющиеся решениями СЛАУ (**) $\Leftrightarrow$  (*) $\Leftrightarrow$ исходной СЛАУ $Ax=0$ (здесь $k=n-r=n-RgA$)

Покажем, что $\Phi_1,\ldots,\Phi_k$ линейно независимы. Составим матрицу $\Phi = (\Phi_1, \ldots, \Phi_k)$

 \[
 	\Phi = 
	\begin{pmatrix}
		\phi_{11} & \phi_{12} & \ldots & \phi{1k} \\
		\vdots & \vdots & & \vdots \\
		\phi_{r1} & \phi_{r2} & \ldots & \phi_{rk} \\
       1 & 0 & \ldots & 0 \\ 
       0 & 1 & 0 & 0 \\
       \hdotsfor{4} \\
       0 & 0 & \ldots & 1
	\end{pmatrix}
\]

$M^{1,\ldots,k}_{r+1,\ldots,n}=|E| = 1 \neq 0 \Rightarrow$ это БМ в матрице $\Phi$ размера $n$x$k \Rightarrow$ по теореме о БМ столбцы $\phi_1,\ldots,\phi_k$ - линейно независимы $\Rightarrow$ по определению $k=n-r$ линейно независимых решений $\phi_1,\ldots,\phi_k$ образуют ФСР ОСЛАУ $Ax=0 \blacksquare$

{\bf Замечание 1: } Построенная в ходе доказательства ФСР называется нормальной (значения свободных переменных образуют столбцы единичной матрицы $E$ т.е. в каждом столбце ФСР 1 свободная переменная = 1, остальные свободные переменные = 0) 

{\bf Замечание 2: } Различных ФСР бесконечно много (при $k=n-r >= 1$) т.к. значения свободных переменных можно брать любыми с условием, чтобы на них образовался БМ в матрице $\Phi$ ("таблице ФСР"), и тоже получим ФСР ($n-r$ линейно независимых решений)

{\bf Следствие (из теоремы о существовании ФСР): } (Критерий существования ненулевого решения однородной квадратной СЛАУ)

пусть $A$ - квадратная матрица. Тогда ОСЛАУ $Ax = 0$ имеет ненулевое решение $\Leftrightarrow$ $\det A = 0$ 

$\square$ Необходимость ($\Rightarrow$):
Дано: $Ax = 0$ % дописать
Доказать: $\det A = 0$

Предположим противное

Предположим, что $\det A \neq 0 \Rightarrow$ по формулам Крамера СЛАУ имеет единственное решение, но всегда есть решение $x=0 \Rightarrow$ других нет - противоречие

Достаточность ($\Leftarrow$): Дано: $\det A = 0$

Доказать: $\exists$ ненулевое решение

Если $\det A=0 \Rightarrow RgA \le n$ Пусть $Rg A = r$ По теореме о существовании ФСР найдётся $n-r>0$ линейно независимых решений - они и будут ненулевыми решениями (т.к. $\forall$ система, содержащая нулевой столбец является линейно зависимой) $\blacksquare$

{\bf Теорема: } (о структуре общего решения ОСЛАУ)

Пусть $\phi_1,\ldots,\phi_k$ - ФСР однор. СЛАУ $Ax = 0$ (где $k=n-r$, $n$ - число переменных, $r=RgA$)

Тогда $\forall$ решение этой ОСЛАУ можно представить в виде линейной комбинации столбцов ФСР:

$x=c_1*\phi_1+\ldots+c_k*\phi_k$

$\square$ Пусть $x^{o}=(x_1^o,\ldots,x_n^o)^T$ - произвольное решение ОСЛАУ $Ax = 0$. Покажем, что $x^o$ линейно зависима через столбцы ФСР $\phi_1,\ldots,\phi_k$

Предположим, что БП располжоена в левом верхнем углу матрицы $A$.

Тогда исходная СЛАУ эквивалентна следующей СЛАУ:

(1)
\[
	\begin{cases}
		a_{11}x_1+\ldots+a_{1r}x_{r}=-a_{1r+1}x_{r+1}-\ldots-a_{1n}x_n \\
		\ldots \\
		a_{r1}x_1+\ldots+a_{rr}x_{r}=-a_{rr+1}x_{r+1}-\ldots-a_{rn}x_n
	\end{cases}
\]


Решим (1) относительно главных переменных $x_1,\ldots,x_r$ (как решение матричного уравнения или методом Крамера или методом Гаусса) - всегда существует решение, т.к. это квадратная невырожденная СЛАУ (в левом углу БМ) 

\[
	\begin{cases}
		x_1=\alpha_{1r+1}x_1 + \ldots + \alpha_{1n}x_{n} \\
		\ldots \\
		x_r=\alpha_{r+r+1}x_r + \ldots + \alpha_{rn}x_{n} \\

	\end{cases}
\]

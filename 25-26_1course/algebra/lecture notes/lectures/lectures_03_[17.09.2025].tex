\section{Перестановки (Подстановки)}
	{\bf Определение: } Перестановкой чисел $1, \ldots, n$ называется расположение их в определённом порядке.

	\[
		\alpha =
		\begin{pmatrix}
			\alpha_1, & \alpha_2, & \ldots, & \alpha_n
		\end{pmatrix}
	\]

	{\bf Пример: } 
	\[
		\alpha =
		\begin{pmatrix}
			5, & 3, & 4, & 1, & 2
		\end{pmatrix}
	\]

	{\bf Определение: } $\alpha_i~\text{и}~\alpha_j$ образуют инверсию в перестановке если $\alpha_i > \alpha_j$, но $i < j$ 

	{\bf Определение: } Знак перестановки это $(-1)^n$, где n - число инверсий в перестановке

	{\bf Обозначение: } $sgn(\alpha)$


	{\bf Пример: }
	\[
		\alpha =
		\underbrace{
			\begin{pmatrix}
				4 & 5 & 1 & 3 & 6 & 2
			\end{pmatrix}
		}_{3+3+0+1+1+0}
	\]

	$$sgn(\alpha)=1$$

	{\bf Определение: } Если $sgn(\alpha)=1$ то $\alpha$ - чётная перестановка, если $sqn(\alpha)=-1$ то $\alpha$ - нечётная перестановка

	{\bf Определение: } Транспозицией называется преобразование при котором в $\alpha$ меняются местами только $\alpha_i$ и $\alpha_j$, а остальные элементы не меняются

	{\bf Утверждение: } Каждая транспозиция меняет чётность перестановки

	$\square$ a) Транспозиция соседних элементов:

	$\alpha_1, \ldots, \alpha_i, \alpha_{i+1}, \ldots, \alpha_n$

	$\downarrow$

	$\alpha_1, \ldots, \alpha_{i+1}, \alpha_i, \ldots, \alpha_n$

	$\Rightarrow$ Число инверсий изменилось на 1 $\Rightarrow$ знак перестановки поменялся

	б) 

	$\alpha_1, \ldots, \alpha_i, \ldots, \alpha_{i+k}, \ldots, \alpha_n$

	Меняем $\alpha_{i} \leftrightarrow \alpha_{i+k}$ с $k-1$ соседних транспозиций

	$\alpha_1, \ldots, \alpha_{i+k}, \ldots, \alpha_{i}, \ldots, \alpha_n$

	$\Rightarrow k + k - 1 = 2k - 1$ шагов (соседних транспозиций) 

	$\Rightarrow$ Знак перестановки поменяется $\blacksquare$

	{\bf Определение: } Подстановка

	\[
		\delta=
		\begin{pmatrix}
			1 & 2 & \ldots & n \\
			\delta(1) & \delta(2) & \ldots & \delta(n)
		\end{pmatrix}
	\]

	-- отображение чисел в себя являющееся взаимо-однозначным

	Нижняя строка -- перестановка

	{\bf Пример: }
	\[
		\delta =
		\begin{pmatrix}
			1 & 2 & 3 & 4 \\
			4 & 2 & 1 & 3
		\end{pmatrix}
	\]

	$sgn(\delta)=1$
	$\delta(2)=2$ - стационарный элемент

	{\bf Определение: } Знаком подстановки называется знак перестановки в её нижней строке

	{\bf Обозначение: } $S_n$ - множество подстановок длины $n$

	$|S_n|=n!$ число элементов и мощность множества

	{\bf Замечание: } Транспозиция - нечётная подстановка (в которой $\alpha_i$ и $\alpha_j$ переходят друг в друга, а остальные элементы неподвижны)

	{\bf Замечание: } Часто используют запись подстановок <<в циклах>>, и каждый элементы выписывается справа

	\[
		\begin{pmatrix}
			1 & 2 & 3 & 4 \\
			4 & 2 & 1 & 3
		\end{pmatrix}
		=
		\begin{pmatrix}
			1 & 4 & 3
		\end{pmatrix}
		*
		\underbrace{
			\begin{pmatrix}
				2
			\end{pmatrix}
		}_{\text{можно не писать}}
		=
		\begin{pmatrix}
			1 & 4 & 3
		\end{pmatrix}
	\]

	Циклическая запись транспозиции: ($\alpha_i, \alpha_j$)

	{\bf Определение: }
	\[
		\delta =
		\begin{pmatrix}
			1 & 2 & \ldots & n \\
			1 & 2 & \ldots & n
		\end{pmatrix}
		= (1)(2)(3)\ldots(n)=Id
	\]
	называется тождественной подстановкой

	{\bf Обозначение: } $Id$ ($id$)

	{\bf Определение: } Умножением подстановок называется их последовательное применение (т.е. композиция отображений)

	{\bf Пример: }
	\[
		\begin{pmatrix}
			1 & 2 & 3 \\
			2 & 1 & 3
		\end{pmatrix}
		* 
		\begin{pmatrix}
			1 & 2 & 3 \\
			3 & 1 & 2
		\end{pmatrix}
		=
		\begin{pmatrix}
			1 & 2 & 3 \\
			1 & 3 & 2
		\end{pmatrix}
	\]

	(умножая слева направо)

	В циклах: $(12) * (132) = (1)(23) = (23)$

	{\bf Замечание: } Умножение подстановок некоммутативно

	{\bf Пример: } $(132)(12) = (13) \neq (23)$

	{\bf Замечание: } $Id$ -- нейтральный элемент по умножению

	{\bf Замечание: } Подстановку обратную к
	\[
		\delta = 
		\begin{pmatrix}
			1 & 2 & 3 & \ldots & n \\
			\delta(1) & \delta(2) & \delta(3) & \ldots & \delta(n)
		\end{pmatrix}
	\]
	можно получить как
	\[
		\delta^{-1} = 
		\begin{pmatrix}
			\delta(1) & \delta(2) & \delta(3) & \ldots & \delta(n) \\
			1 & 2 & 3 & \ldots & n \\
		\end{pmatrix}
	\]
	и отсортировав столбцы

	$\Rightarrow \delta * \delta^{-1} = \delta^{-1} * \delta = Id$

	{\bf Пример: }
	\[
		\delta = 
		\begin{pmatrix}
			1 & 2 & 3 \\
			2 & 1 & 3
		\end{pmatrix}
	\]
	\[
		\delta^{-1} = 
		\begin{pmatrix}
			1 & 2 & 3 \\
			2 & 1 & 3
		\end{pmatrix}
	\]

	{\bf Замечание: } $\forall$ Подстановку можно представить как произведение транспозиций

	$\square$ Запишем подстановку в циклах $(\alpha_1, \alpha_2, \ldots, \alpha_k)= (\alpha_k, \alpha_{k-1})(\alpha_{k-1}, \alpha_{k-2})\ldots(\alpha_2, \alpha_1)$

	- произведение $k-1$ транспозиций слева направо
	$\blacksquare$

	{\bf Замечание: } $sgn(\delta_1 * \delta_2)=sgn(\delta_1) * sgn(\delta_2)$

	\section{Определитель}
	
	{\bf Определение} Определителем или детерминантом квадратной матрицы $A$ порядка $n$ называется сумма $n!$ слагаемых следующего вида:

	$\det A = \sum_{\delta \in S_n}~sgn(\delta) * a_{1\delta(1)} * a_{2\delta(2)} * a_{3\delta(3)} * \ldots * a_{n\delta(n)}$, где 

	\[
		A = 
		\begin{pmatrix}
			a_{11} & a_{12} & a_{13} & \ldots & a_{1n} \\
			a_{21} & a_{22} & a_{23} & \ldots & a_{2n} \\
			\hdotsfor{5} \\
			a_{n1} & \hdotsfor{3} & a_{nn}
		\end{pmatrix}
	\]
	и
	\[
		\delta = 
		\begin{pmatrix}
			1 & 2 & 3 & \ldots & n \\
			\delta(1) & \delta(2) & \delta(3) & \ldots & \delta(n)
		\end{pmatrix}
	\]

	{\bf Обозначение: } $\det A$ или $|A|$

	{\bf Замечание: } По сути $\det A$ является суммой произведений элементов матрицы по одному из каждой строки и столбца (стоящие в разных строках и столбцах по всем способам так сделать (с учётом знака))

	{\bf Пример: } $n = 2 \Rightarrow n! = 2! = 2$

	\[
		\delta_1 = 
		\begin{pmatrix}
			1 & 2 \\
			1 & 2
		\end{pmatrix}
		= Id
	\]

	\[
		\delta_2 =
		\begin{pmatrix}
			1 & 2 \\
			2 & 1
		\end{pmatrix}
	\]

	$\Rightarrow$

	\[
		\begin{vmatrix}
			a_{11} & a_{12} \\
			a_{21} & a_{22}
		\end{vmatrix}
		= a_{11} * a_{22} - a_{21} * a_{12}
	\]

	% todo: пример для n = 3...

	{\bf Пример: } $n = 3 \Rightarrow n! = 3! = 6$

	\[
		A =
		\left[
			\begin{matrix}
				a_{11} & a_{12} & a_{13}, \\
				a_{21} & a_{22} & a_{23}, \\
				a_{31} & a_{32} & a_{33}
			\end{matrix}
			\left|
				\,
				\begin{matrix}
					a_{11} & a_{12} \\
					a_{21} & a_{22} \\
					a_{31} & a_{32}
				\end{matrix}
			\right.
		\right]
	\]


	$\det A = a_{11} a_{22} a_{33} + a_{12} a_{23} a_{31} + a_{13} a_{21} a_{32} - a_{31} a_{22} a_{13} - a_{22} a_{23} a_{11} - a_{32} a_{23} a_{11} - a_{33} a_{21} a_{12}$ - Правило Саррюса

	(Диагонали $\searrow$ идут с плюсом, диагонали с $\nearrow$ с минусом)

	\subsection{Свойства определителей}

	\textnumero1
	$\det A = \det A^T$
	($\Rightarrow$ Все свойства $\det$ верные для строк матрицы справедливы и для столбцов)

	$\square~B = A^T, b_{ij}=a_{ji}$
	Переставим $b_{i\delta_i}$ по возрастанию номеров столбцов

	$\det B = \sum_{\delta \in S_n} sgn(\delta) * b_{1\delta(1)} * b_{2\delta(2)} * \ldots * b_{n\delta(n)} = $

	Используем: 
	\[
		\delta = 
		\begin{pmatrix}
			1 & 2 & \ldots & n \\
			\delta(1) & \delta(2) & \ldots & \delta(n)
		\end{pmatrix}
		= 
		\begin{pmatrix}
			\delta(1)^{-1} & \delta(2)^{-1} & \ldots & \delta(n)^{-1} \\
			1 & 2 & \ldots & n
		\end{pmatrix}
	\]

	$sgn(\delta^{-1})=sgn(\delta)$ так как $sgn(\delta * \delta^{-1}) = sgn(\delta) * sgn(\delta^{-1}) = 1$

	$= \sum_{\delta \in S_n} sgn(\delta^{-1}) b_{\delta^{-1}_{}(1)1} b_{\delta^{-1}_{}(2)2} * \ldots * b_{\delta^{-1}_{}(n)n} = \sum_{\delta \in S_n} sgn(\delta^{-1}) * a_{1\delta^{-1}(1)} * a_{2\delta^{-1}(2)} * \ldots * a_{n\delta^{-1}(n)}$

	Переименуем $\tau = \delta^{-1}$

	$=\sum_{\tau \in S_n} sgn(\tau) * a_{1\tau_1} * \ldots * a_{n\tau_n} = \det A \,\blacksquare$


	\textnumero2
	Определитель линеен по строкам и столбцам т.е. пусть $A = (A_1, \ldots, A_n)$ -- матрица как набор строк

	Тогда:

	a) $\det(A_1, \ldots, A_i^\prime + A_i^{\prime\prime}, \ldots, A_n)$
	$=\det(A_1, \ldots, A_i^\prime, \ldots, A_n) + \det(A_1, \ldots, A_i^{\prime\prime}, \ldots, A_n)$

	б) $\det(A_1, \ldots, \alpha * A_i, \ldots, A_n) = \alpha * \det(A_1, \ldots, A_i, \ldots, A_n)$

	{\bf Пример: }
	\[
		\begin{vmatrix}
			1 & 3 + x \\
			2 & 7 + x
		\end{vmatrix}
		=
		\begin{vmatrix}
			1 & 3 \\
			2 & 7
		\end{vmatrix}
		+
		\begin{vmatrix}
			1 & x \\
			2 & x
		\end{vmatrix}
		=
		(1 * 7 - 3 * 2) + x * 
		\begin{vmatrix}
			1 & 1 \\
			2 & 1
		\end{vmatrix}
		= 1-x
	\]

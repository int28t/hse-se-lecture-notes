\section {Матрицы}
	Матрицей размера/типа/порядка $n$x$m$ называется упорядоченная таблица с $m$ строками и $n$ столбцами.

	{\bf Обозначение: }
	\[
		A = 
		\begin{pmatrix}
			a_{11} & \ldots & a_{1n} \\
			\vdots & \ddots & \vdots \\
			a_{m1} & \ldots & a_{mn} 
		\end{pmatrix}
		_{mxn}
	\]
	$i$ - строка, $j$ - столбец

	$M_{mn}(\mathbb{R})$ - множество всех матриц размера $m$x$n$ с вещественными элементами

	1) При $m = n$ матрица называется квадратной порядка $n$

	2) При $m = 1$ матрица $1$x$n$ - $n$-мерная строка

	При $n = 1$ матрица $m$x$1$ - $m$-мерный столбец

	Матрица $1$x$1$ - число

	3) Матрица состоящая из нулей (т.е. $a_{ij}=0~\forall i = \overline{1, m},~j = \overline{1, n}$) называется нулевой матрицей.
	{\bf Обозначение: $\mathbb{O}$ }

	4) Будем называть единичной квадратную матрицу порядка n, если:

	\[
		a_{ij}=\underbrace{\delta^i_j}_{\text{Символ Кронекера}}=
		\begin{cases}
			1, & \text{при  $i = j$ } \\
			0, & \text{при  $i \neq j$}
		\end{cases}
	\]

	{\bf Обозначение: }
	\[
		E = \begin{pmatrix}
			1 & 0 & 0 & \ldots & 0 \\
			0 & 1 & 0 & \ldots & 0 \\
			0 & 0 & 1 & \ldots & 0 \\
			\hdotsfor{5} \\
			0 & 0 & \ldots & 0 & 1
			\end{pmatrix}
	\]

	{\bf Определение: } Две матрицы $a$ и $b$ называются равными если они одного размера и соответствующие элементы равны (т.е. $a_{ij}=b_{ij} \forall i=\overline{1, m}, j=\overline{1, n}$)

	{\bf Определение: } Матрица $C$ называется суммой матриц A и B если все матрицы A, B и C одинакового размера и $c_{ij}=b_{ij}+a_{ij}, \forall i=\overline{1,m}, j=\overline{1,n}$

	{\bf Обозначение: } $C = A + B$

	{\bf Пример: }

	\[
		\begin{pmatrix} 
			1 & 0 & -1 \\
			0 & 0 & 2
		\end{pmatrix}
		+
		\begin{pmatrix} 
			0 & 0 & 1 \\
			0 & 0 & -2
		\end{pmatrix}
		=
		\begin{pmatrix} 
			1 & 0 & 0 \\
			0 & 0 & 0
		\end{pmatrix}
	\]

	\subsection{Свойства сложения матриц}

	1. Коммутативность ($A + B = B + A$)

	$\square$ Поэлементно: $[A+B]_{ij} \underbrace{=}_{\text{по определению сложения}} [A]_{ij}+[B]_{ij} \underbrace{=}_{\text{Вещественные числа коммутативны}} [B]_{ij}+[A]_{ij} = \underbrace{=}_{\text{по определению сложения}} [B+A]_{ij}$ $\blacksquare$

	2. Ассоциативность $(A + B) + C = A + (B + C)$

	3. $\exists$ нейтральный элемент по сложению т.е. $\exists$ матрица $\mathbb{O} \in M_{mn}(\mathbb R)~\forall A \in M_{mn}(\mathbb R)$ выполняется $\mathbb O + A = A + \mathbb O = A$

	$\square~[\mathbb O]_{ij} = 0$ -- нулевая матрица $\blacksquare$

	4. $\forall$ матрицы $A \in M_{mn}(\mathbb R)~\exists~B \in M_{mn}(\mathbb R): A + B = B + A = \mathbb O$ -- нулевая матрица
	
	$\square~B_{ij} = -A_{ij}~\blacksquare$

	{\bf Обозначение: } $B = -A$ -- Обратная по сложению к $A$ или противоположная

	{\bf Определение: } Матрица $C$ называется произведением числа $\lambda \in \mathbb R$ и матрицы $A$, если матрицы $C$ и $A$ одинакового размера и $[C]_{ij} = \lambda * [A]_{ij}~\forall i = \overline{1,m}, j = \overline{1,n}$

	{\bf Обозначение: } $C = \lambda * A$

	{\bf Пример: }
	\[
		2 * 
		\begin{pmatrix}
			1 & 3 \\
			0 & -1
		\end{pmatrix}
		=
		\begin{pmatrix}
			2 & 6 \\
			0 & -2
		\end{pmatrix}
	\]

	{\bf Определение: } Разностью матриц A и B называется сумма $A$ и $-B$

	\subsection {Свойства умножения на число}

	$\forall \lambda, \mu \in \mathbb R: (\lambda * \mu) * A = \lambda * (\mu * A)$ -- ассоциативность относительно умножения на число

	$(\lambda + \mu) * A = \lambda * A + \mu * A$ -- дистрибутивность относительно умножения чисел

	$\lambda * (A + B) = \lambda * A + \lambda * B$ -- дистрибутивность относительно сложения матриц

	{\bf Это место стоит перепроверить (!!!) } 

	$1 * A = A$ -- унарность

	\subsection{Умножение матриц}

	Рассмотрим $A_{mxn} \in M_{mn}(\mathbb R), B_{nxk} \in M_{nk}(\mathbb R)$:

	{\bf Определение: }
	Произведением матрицы $A_{mxn}$ и $B_{nxk}$ (число столбцов в $A$ равно числу строк в $B$) называется $C_{mk}$ где:
	\[
		C_{ij}=\sum_{q=1}^{n} a_{iq} * b_{qj}
	\]
	
	($C_{ij}=a_{i1} b_{1j}+a_{i2} b_{2j}+\ldots+a_{in} b_{nj}~\forall i=\overline{1,m},~j=\overline{1,k}$)

	{\bf Пример:}

	\[
		\begin{pmatrix}
			1 & 2 \\
			3 & 4 \\
			0 & -1
		\end{pmatrix}_{\text{3x2}}
		*
		\begin{pmatrix}
			1 \\
			-2
		\end{pmatrix}_{\text{2x1}}
		=
		\begin{pmatrix}
			-3 \\
			-5 \\
			2
		\end{pmatrix}_{\text{3x1}}
	\]

	\subsection{Свойства умножения матриц}

	1. Ассоциативность
	Пусть $A_{mxn}$, $B_{nxk}$, $C_{kxl}$. Тогда $(A * B) * C = A * (B * C)$

	$\square~[(A * B) * C]_{ij} = \sum_{r=1}^k~[A * B]_{ir} * [C]_{rj} = \sum_{r=1}^{k}~(\sum_{s=1}^{n}~[A]_{is} * [B]_{sr}) * [C]_{rj} = \sum_{r=1}^{k}~\sum_{s=1}^{n}~[A]_{is} * [B]_{sr} * [C]_{rj} \underbrace{=}_\text{перегруппируем слагаемые} \sum_{s=1}^{n}~\sum_{r=1}^{k}~[A]_{is} * [B]_{sr} * [C]_{rj}=\sum_{s=1}^{n}~[A]_{is} * (\sum_{r=1}^{k}~[B]_{sr} * [C]_{rj})=\sum_{s=1}^{n}~[A]_{is} * [B * C]_{sj} \underbrace{=}_\text{по определению умножения}=A * (B * C)~\blacksquare$

	2. $\exists$ Нейтрального элемента по умножению (для квадратных матриц)
	$\exists$ матрица $E \in M_{n}(\mathbb R): \forall A \in M_{n}(\mathbb R)~E * A = A * E = A$

	$\square$ Предъявим единичную матрицу

	$[E]_{ij} = \delta_j^i$

	$[A * E]_{ij} = \sum_{r=1}^{n}~[A]_{ir} * [E]_{rj} = \sum_{r=1}^{n}~[A]_{ir} * \delta_j^i = 0 + \ldots + [A]_{ij} + \ldots + 0 = [A]_{ij}~\forall i, j = \overline{1,n}$

	$E * A$ аналогично $\blacksquare$

	3. $A * \mathbb O = \mathbb O * A = \mathbb O$
	Для квадратных $A$ и $\mathbb O$ порядка $n$

	Рассмотрим: $(A + B) * C = A * C + B * C$ -- дистрибутивность умножения матриц
	{\bf Замечание: } Вообще говоря умножение матриц некоммутативно (даже для квадратных)
	{\bf Пример: }
	\[
		\begin{pmatrix}
			1 & 1 \\
			0 & 0
		\end{pmatrix}
		*
		\begin{pmatrix}
			0 & 0 \\
			1 & 1
		\end{pmatrix}
		=
		\begin{pmatrix}
			1 & 1 \\
			0 & 0
		\end{pmatrix}
	\]

	\[
		\begin{pmatrix}
			0 & 0 \\
			1 & 1
		\end{pmatrix}
		*
		\begin{pmatrix}
			1 & 1 \\
			0 & 0
		\end{pmatrix}
		=
		\begin{pmatrix}
			0 & 0 \\
			1 & 1
		\end{pmatrix}
	\]

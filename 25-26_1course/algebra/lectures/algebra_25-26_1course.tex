\documentclass[a4paper,12pt]{article}
\usepackage{header}
\usepackage{hyperref}
\usepackage{amssymb}

\title{Лекции по алгебре (1 курс, 25-26)}
\author{github.com/int28t/hse-se-lecture-notes}
\date{}

\begin{document}
    \pagestyle{empty}
	\maketitle
	\begin{flushright}
		Михайлец Екатерина Викторовна

		emihaylets@hse.ru

		ТГК: \url{t.me/alg_pi_25_26}
	\end{flushright}
	\tableofcontents{}
    \newpage
    \pagestyle{plain}
	% AUTO-LECTURES
	\section {Матрицы}
	Матрицей размера/типа/порядка $n$x$m$ называется упорядоченная таблица с $m$ строками и $n$ столбцами.

	{\bf Обозначение: }
	\[
		A = 
		\begin{pmatrix}
			a_{11} & \ldots & a_{1n} \\
			\vdots & \ddots & \vdots \\
			a_{m1} & \ldots & a_{mn} 
		\end{pmatrix}
		_{mxn}
	\]
	$i$ - строка, $j$ - столбец

	$M_{mn}(\mathbb{R})$ - множество всех матриц размера $m$x$n$ с вещественными элементами

	1) При $m = n$ матрица называется квадратной порядка $n$

	2) При $m = 1$ матрица $1$x$n$ - $n$-мерная строка

	При $n = 1$ матрица $m$x$1$ - $m$-мерный столбец

	Матрица $1$x$1$ - число

	3) Матрица состоящая из нулей (т.е. $a_{ij}=0~\forall i = \overline{1, m},~j = \overline{1, n}$) называется нулевой матрицей.
	{\bf Обозначение: $\mathbb{O}$ }

	4) Будем называть единичной квадратную матрицу порядка n, если:

	\[
		a_{ij}=\underbrace{\delta^i_j}_{\text{Символ Кронекера}}=
		\begin{cases}
			1, & \text{при  $i = j$ } \\
			0, & \text{при  $i \neq j$}
		\end{cases}
	\]

	{\bf Обозначение: }
	\[
		E = \begin{pmatrix}
			1 & 0 & 0 & \ldots & 0 \\
			0 & 1 & 0 & \ldots & 0 \\
			0 & 0 & 1 & \ldots & 0 \\
			\hdotsfor{5} \\
			0 & 0 & \ldots & 0 & 1
			\end{pmatrix}
	\]

	{\bf Определение: } Две матрицы $a$ и $b$ называются равными если они одного размера и соответствующие элементы равны (т.е. $a_{ij}=b_{ij} \forall i=\overline{1, m}, j=\overline{1, n}$)

	{\bf Определение: } Матрица $C$ называется суммой матриц A и B если все матрицы A, B и C одинакового размера и $c_{ij}=b_{ij}+a_{ij}, \forall i=\overline{1,m}, j=\overline{1,n}$

	{\bf Обозначение: } $C = A + B$

	{\bf Пример: }

	\[
		\begin{pmatrix} 
			1 & 0 & -1 \\
			0 & 0 & 2
		\end{pmatrix}
		+
		\begin{pmatrix} 
			0 & 0 & 1 \\
			0 & 0 & -2
		\end{pmatrix}
		=
		\begin{pmatrix} 
			1 & 0 & 0 \\
			0 & 0 & 0
		\end{pmatrix}
	\]

	\subsection{Свойства сложения матриц}

	1. Коммутативность ($A + B = B + A$)

	$\square$ Поэлементно: $[A+B]_{ij} \underbrace{=}_{\text{по определению сложения}} [A]_{ij}+[B]_{ij} \underbrace{=}_{\text{Вещественные числа коммутативны}} [B]_{ij}+[A]_{ij} = \underbrace{=}_{\text{по определению сложения}} [B+A]_{ij}$ $\blacksquare$

	2. Ассоциативность $(A + B) + C = A + (B + C)$

	3. $\exists$ нейтральный элемент по сложению т.е. $\exists$ матрица $\mathbb{O} \in M_{mn}(\mathbb R)~\forall A \in M_{mn}(\mathbb R)$ выполняется $\mathbb O + A = A + \mathbb O = A$

	$\square~[\mathbb O]_{ij} = 0$ -- нулевая матрица $\blacksquare$

	4. $\forall$ матрицы $A \in M_{mn}(\mathbb R)~\exists~B \in M_{mn}(\mathbb R): A + B = B + A = \mathbb O$ -- нулевая матрица
	
	$\square~B_{ij} = -A_{ij}~\blacksquare$

	{\bf Обозначение: } $B = -A$ -- Обратная по сложению к $A$ или противоположная

	{\bf Определение: } Матрица $C$ называется произведением числа $\lambda \in \mathbb R$ и матрицы $A$, если матрицы $C$ и $A$ одинакового размера и $[C]_{ij} = \lambda * [A]_{ij}~\forall i = \overline{1,m}, j = \overline{1,n}$

	{\bf Обозначение: } $C = \lambda * A$

	{\bf Пример: }
	\[
		2 * 
		\begin{pmatrix}
			1 & 3 \\
			0 & -1
		\end{pmatrix}
		=
		\begin{pmatrix}
			2 & 6 \\
			0 & -2
		\end{pmatrix}
	\]

	{\bf Определение: } Разностью матриц A и B называется сумма $A$ и $-B$

	\subsection {Свойства умножения на число}

	$\forall \lambda, \mu \in \mathbb R: (\lambda * \mu) * A = \lambda * (\mu * A)$ -- ассоциативность относительно умножения на число

	$(\lambda + \mu) * A = \lambda * A + \mu * A$ -- дистрибутивность относительно умножения чисел

	$\lambda * (A + B) = \lambda * A + \lambda * B$ -- дистрибутивность относительно сложения матриц

	{\bf Это место стоит перепроверить (!!!) } 

	$1 * A = A$ -- унарность

	\subsection{Умножение матриц}

	Рассмотрим $A_{mxn} \in M_{mn}(\mathbb R), B_{nxk} \in M_{nk}(\mathbb R)$:

	{\bf Определение: }
	Произведением матрицы $A_{mxn}$ и $B_{nxk}$ (число столбцов в $A$ равно числу строк в $B$) называется $C_{mk}$ где:
	\[
		C_{ij}=\sum_{q=1}^{n} a_{iq} * b_{qj}
	\]
	
	($C_{ij}=a_{i1} b_{1j}+a_{i2} b_{2j}+\ldots+a_{in} b_{nj}~\forall i=\overline{1,m},~j=\overline{1,k}$)

	{\bf Пример:}

	\[
		\begin{pmatrix}
			1 & 2 \\
			3 & 4 \\
			0 & -1
		\end{pmatrix}_{\text{3x2}}
		*
		\begin{pmatrix}
			1 \\
			-2
		\end{pmatrix}_{\text{2x1}}
		=
		\begin{pmatrix}
			-3 \\
			-5 \\
			2
		\end{pmatrix}_{\text{3x1}}
	\]

	\subsection{Свойства умножения матриц}

	1. Ассоциативность
	Пусть $A_{mxn}$, $B_{nxk}$, $C_{kxl}$. Тогда $(A * B) * C = A * (B * C)$

	$\square~[(A * B) * C]_{ij} = \sum_{r=1}^k~[A * B]_{ir} * [C]_{rj} = \sum_{r=1}^{k}~(\sum_{s=1}^{n}~[A]_{is} * [B]_{sr}) * [C]_{rj} = \sum_{r=1}^{k}~\sum_{s=1}^{n}~[A]_{is} * [B]_{sr} * [C]_{rj} \underbrace{=}_\text{перегруппируем слагаемые} \sum_{s=1}^{n}~\sum_{r=1}^{k}~[A]_{is} * [B]_{sr} * [C]_{rj}=\sum_{s=1}^{n}~[A]_{is} * (\sum_{r=1}^{k}~[B]_{sr} * [C]_{rj})=\sum_{s=1}^{n}~[A]_{is} * [B * C]_{sj} \underbrace{=}_\text{по определению умножения}=A * (B * C)~\blacksquare$

	2. $\exists$ Нейтрального элемента по умножению (для квадратных матриц)
	$\exists$ матрица $E \in M_{n}(\mathbb R): \forall A \in M_{n}(\mathbb R)~E * A = A * E = A$

	$\square$ Предъявим единичную матрицу

	$[E]_{ij} = \delta_j^i$

	$[A * E]_{ij} = \sum_{r=1}^{n}~[A]_{ir} * [E]_{rj} = \sum_{r=1}^{n}~[A]_{ir} * \delta_j^i = 0 + \ldots + [A]_{ij} + \ldots + 0 = [A]_{ij}~\forall i, j = \overline{1,n}$

	$E * A$ аналогично $\blacksquare$

	3. $A * \mathbb O = \mathbb O * A = \mathbb O$
	Для квадратных $A$ и $\mathbb O$ порядка $n$

	Рассмотрим: $(A + B) * C = A * C + B * C$ -- дистрибутивность умножения матриц
	{\bf Замечание: } Вообще говоря умножение матриц некоммутативно (даже для квадратных)
	{\bf Пример: }
	\[
		\begin{pmatrix}
			1 & 1 \\
			0 & 0
		\end{pmatrix}
		*
		\begin{pmatrix}
			0 & 0 \\
			1 & 1
		\end{pmatrix}
		=
		\begin{pmatrix}
			1 & 1 \\
			0 & 0
		\end{pmatrix}
	\]

	\[
		\begin{pmatrix}
			0 & 0 \\
			1 & 1
		\end{pmatrix}
		*
		\begin{pmatrix}
			1 & 1 \\
			0 & 0
		\end{pmatrix}
		=
		\begin{pmatrix}
			0 & 0 \\
			1 & 1
		\end{pmatrix}
	\]

	\section{Перестановки (Подстановки)}
	{\bf Определение: } Перестановкой чисел $1, \ldots, n$ называется расположение их в определённом порядке.

	\[
		\alpha =
		\begin{pmatrix}
			\alpha_1, & \alpha_2, & \ldots, & \alpha_n
		\end{pmatrix}
	\]

	{\bf Пример: } 
	\[
		\alpha =
		\begin{pmatrix}
			5, & 3, & 4, & 1, & 2
		\end{pmatrix}
	\]

	{\bf Определение: } $\alpha_i~\text{и}~\alpha_j$ образуют инверсию в перестановке если $\alpha_i > \alpha_j$, но $i < j$ 

	{\bf Определение: } Знак перестановки это $(-1)^n$, где n - число инверсий в перестановке

	{\bf Обозначение: } $sgn(\alpha)$


	{\bf Пример: }
	\[
		\alpha =
		\underbrace{
			\begin{pmatrix}
				4 & 5 & 1 & 3 & 6 & 2
			\end{pmatrix}
		}_{3+3+0+1+1+0}
	\]

	$$sgn(\alpha)=1$$

	{\bf Определение: } Если $sgn(\alpha)=1$ то $\alpha$ - чётная перестановка, если $sqn(\alpha)=-1$ то $\alpha$ - нечётная перестановка

	{\bf Определение: } Транспозицией называется преобразование при котором в $\alpha$ меняются местами только $\alpha_i$ и $\alpha_j$, а остальные элементы не меняются

	{\bf Утверждение: } Каждая транспозиция меняет чётность перестановки

	$\square$ a) Транспозиция соседних элементов:

	$\alpha_1, \ldots, \alpha_i, \alpha_{i+1}, \ldots, \alpha_n$

	$\downarrow$

	$\alpha_1, \ldots, \alpha_{i+1}, \alpha_i, \ldots, \alpha_n$

	$\Rightarrow$ Число инверсий изменилось на 1 $\Rightarrow$ знак перестановки поменялся

	б) 

	$\alpha_1, \ldots, \alpha_i, \ldots, \alpha_{i+k}, \ldots, \alpha_n$

	Меняем $\alpha_{i} \leftrightarrow \alpha_{i+k}$ с $k-1$ соседних транспозиций

	$\alpha_1, \ldots, \alpha_{i+k}, \ldots, \alpha_{i}, \ldots, \alpha_n$

	$\Rightarrow k + k - 1 = 2k - 1$ шагов (соседних транспозиций) 

	$\Rightarrow$ Знак перестановки поменяется $\blacksquare$

	{\bf Определение: } Подстановка

	\[
		\delta=
		\begin{pmatrix}
			1 & 2 & \ldots & n \\
			\delta(1) & \delta(2) & \ldots & \delta(n)
		\end{pmatrix}
	\]

	-- отображение чисел в себя являющееся взаимо-однозначным

	Нижняя строка -- перестановка

	{\bf Пример: }
	\[
		\delta =
		\begin{pmatrix}
			1 & 2 & 3 & 4 \\
			4 & 2 & 1 & 3
		\end{pmatrix}
	\]

	$sgn(\delta)=1$
	$\delta(2)=2$ - стационарный элемент

	{\bf Определение: } Знаком подстановки называется знак перестановки в её нижней строке

	{\bf Обозначение: } $S_n$ - множество подстановок длины $n$

	$|S_n|=n!$ число элементов и мощность множества

	{\bf Замечание: } Транспозиция - нечётная подстановка (в которой $\alpha_i$ и $\alpha_j$ переходят друг в друга, а остальные элементы неподвижны)

	{\bf Замечание: } Часто используют запись подстановок <<в циклах>>, и каждый элементы выписывается справа

	\[
		\begin{pmatrix}
			1 & 2 & 3 & 4 \\
			4 & 2 & 1 & 3
		\end{pmatrix}
		=
		\begin{pmatrix}
			1 & 4 & 3
		\end{pmatrix}
		*
		\underbrace{
			\begin{pmatrix}
				2
			\end{pmatrix}
		}_{\text{можно не писать}}
		=
		\begin{pmatrix}
			1 & 4 & 3
		\end{pmatrix}
	\]

	Циклическая запись транспозиции: ($\alpha_i, \alpha_j$)

	{\bf Определение: }
	\[
		\delta =
		\begin{pmatrix}
			1 & 2 & \ldots & n \\
			1 & 2 & \ldots & n
		\end{pmatrix}
		= (1)(2)(3)\ldots(n)=Id
	\]
	называется тождественной подстановкой

	{\bf Обозначение: } $Id$ ($id$)

	{\bf Определение: } Умножением подстановок называется их последовательное применение (т.е. композиция отображений)

	{\bf Пример: }
	\[
		\begin{pmatrix}
			1 & 2 & 3 \\
			2 & 1 & 3
		\end{pmatrix}
		* 
		\begin{pmatrix}
			1 & 2 & 3 \\
			3 & 1 & 2
		\end{pmatrix}
		=
		\begin{pmatrix}
			1 & 2 & 3 \\
			1 & 3 & 2
		\end{pmatrix}
	\]

	(умножая слева направо)

	В циклах: $(12) * (132) = (1)(23) = (23)$

	{\bf Замечание: } Умножение подстановок некоммутативно

	{\bf Пример: } $(132)(12) = (13) \neq (23)$

	{\bf Замечание: } $Id$ -- нейтральный элемент по умножению

	{\bf Замечание: } Подстановку обратную к
	\[
		\delta = 
		\begin{pmatrix}
			1 & 2 & 3 & \ldots & n \\
			\delta(1) & \delta(2) & \delta(3) & \ldots & \delta(n)
		\end{pmatrix}
	\]
	можно получить как
	\[
		\delta^{-1} = 
		\begin{pmatrix}
			\delta(1) & \delta(2) & \delta(3) & \ldots & \delta(n) \\
			1 & 2 & 3 & \ldots & n \\
		\end{pmatrix}
	\]
	и отсортировав столбцы

	$\Rightarrow \delta * \delta^{-1} = \delta^{-1} * \delta = Id$

	{\bf Пример: }
	\[
		\delta = 
		\begin{pmatrix}
			1 & 2 & 3 \\
			2 & 1 & 3
		\end{pmatrix}
	\]
	\[
		\delta^{-1} = 
		\begin{pmatrix}
			1 & 2 & 3 \\
			2 & 1 & 3
		\end{pmatrix}
	\]

	{\bf Замечание: } $\forall$ Подстановку можно представить как произведение транспозиций

	$\square$ Запишем подстановку в циклах $(\alpha_1, \alpha_2, \ldots, \alpha_k)= (\alpha_k, \alpha_{k-1})(\alpha_{k-1}, \alpha_{k-2})\ldots(\alpha_2, \alpha_1)$

	- произведение $k-1$ транспозиций слева направо
	$\blacksquare$

	{\bf Замечание: } $sgn(\delta_1 * \delta_2)=sgn(\delta_1) * sgn(\delta_2)$

\end{document}

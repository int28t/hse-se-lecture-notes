\section{Листок 4}

\begin{theorem}[Малая теорема Ферма]
    \[
        a \not \vdots p, p \text{ --- простое}
    \]
    \[
        \Downarrow
    \]
    \[
        a^{p-1} \equiv 1 \pmod{p} 
    \]
\end{theorem}

\begin{lemma}[Утверждение]
    \[
        ma \equiv mb \pmod{n} \Rightarrow a \equiv b \pmod{n},\ (m, n) = 1
    \]
\end{lemma}

\begin{theorem}[]
    \[
        ma \equiv mb \pmod{n}, (m, n) = 1
    \]
    \[
        \Downarrow
    \]
    \[
        (ma - mb) \vdots n,\ (m, n) = 1
    \]
    \[
        \Downarrow
    \]
    \[
        (a - b) \vdots n
    \]
    \[
        \Downarrow
    \]
    \[
        a \equiv b \pmod{n}
    \]
\end{theorem}

\subsection{Задача 14}

\begin{example}[Задача]
    Найдите остаток от деления числа $\underbrace{111 \ldots 111}_{105}$ на 107
\end{example}

\begin{remark}[Решение]
    \[
        \underbrace{11 \ldots 1}_{107} \equiv x \pmod{107} | \cdot 9
    \]
    \[
        \underbrace{99 \ldots 9}_{105} \equiv 9x \pmod{107} 
    \]
    \[
        1 \underbrace{0 \ldots 0}_{105}  \equiv 9x + 1 \pmod{107} | \cdot 10
    \]
    \[
        10^{106} \equiv 90x + 10 \pmod{107} 
    \]
    По МТФ: 
    \[
        90x + 10 \equiv 1 \pmod{107}
    \]
    \[
        90x \equiv -9 \pmod{107} | : 9
    \]
    \[
        10x \equiv -1 \pmod{107}
    \]
    \[
        10x \equiv 106 \pmod{107} | : 2
    \]
    \[
        5x \equiv 53 \pmod{107}
    \]
    \[
        5x \equiv 160 \pmod{107} | : 5
    \]
    \[
        x \equiv 32 \pmod{107}
    \]
\end{remark}

\begin{remark}[Ответ]
    32
\end{remark}

\subsection{Задача 15}

\begin{remark}[Решение]
    \[
        1)\ 41^{41^{41}} \equiv 2^{41^{41} } \pmod{13}
    \]
    Построим табличку $[n, 2^n mod 13]$. Заметим что $2^6 \equiv 12 \equiv -1 \pmod{13}$, значит $2^{12} \equiv (2^6)^2 \equiv 1 $
    \[
        2)\ 41^{41} \equiv 5^{41} \pmod{12}  
    \]
    \[
        5^2 \equiv 1 \pmod{12}
    \]
    \[
        3)\ 41 \equiv 1 \pmod{2}
    \]
    \[
        5^{41} \equiv 5^1 \pmod{12} 
    \]
    \[
        2^{41^{41}} \equiv 2^5 \equiv 6 \pmod{13} 
    \]
\end{remark}

\begin{remark}[Ответ]
    6
\end{remark}

\begin{lemma}
    Если $b \equiv c \pmod{\phi (n)}$, то $a^b \equiv a^c \pmod{n}$
\end{lemma} 

\begin{theorem}[Теорема Эйлера]
    \[
        (a, n) = 1 \Rightarrow a^{\phi (n)} \equiv 1 \pmod{n}
    \]
\end{theorem}

\subsection{Задача 16}
\begin{example}[Задача о теореме Вильсона]
    Число $p > 1$ простое тогда и только тогда, когда $(p - 1)! \equiv -1 \pmod{p}$
\end{example}
\begin{Proof}
{\bf $\Leftarrow$}

о/п $p$ --- составное

{\bf Случай 1: } $ p = a \cdot b,\ 1 < a < b < p $ (то есть $p$ --- не квадрат простого числа) 
\[
    \Rightarrow (p-1)! = 1 \cdot 2 \cdot \ldots \cdot a \cdot \ldots \cdot b \ldots (p - 1) \vdots p \Rightarrow (p - 1)! \equiv 0 \pmod{p}\ \bot
\]
{\bf Случай 2: } $ p = q^2,\ q$ --- простое
\[
    2q < p \Leftrightarrow 2q < q^2 \Leftrightarrow q > 2
\]
\[
    \Rightarrow (p - 1)! = 1 \cdot 2 \cdot \ldots q \ldots 2q \ldots (p - 1) \vdots p \Rightarrow (p-1)! \equiv 0 \pmod{p}\ \bot
\]
{\bf Случай 3: } $p = 4 \Rightarrow (4 - 1)! \equiv 6 \equiv 2 \not\equiv -1 \pmod{4}\ \bot$

Во всех случаях получили противоречие

\end{Proof}

\begin{lemma}
    \[
        a \in \Z,\ (a, n) = 1 \Rightarrow \exists x \in [1, n-1] : ax \equiv 1 \pmod{n}
    \]
\end{lemma}

\begin{Proof}[]
{\bf $\Rightarrow$}

1) Пусть $x \in [1; p - 1] \land x^2 \equiv 1 \pmod{p}$

\[
    x^2 \equiv 1 \pmod{p} \Leftrightarrow (x^2 - 1) \vdots p \Leftrightarrow (x-1)(x+1) \vdots p \Leftrightarrow (x - 1) \vdots p \lor (x + 1) \vdots p \Leftrightarrow
\]
\[
    \Leftrightarrow x = 1 \lor x = p - 1
\]

2) $(p - 1)! = 1 \cdot 2 \cdot 3 \cdot \ldots \cdot (p-1)$

Пусть $a, b \in [1; p - 1],\ ax \equiv 1 \pmod{1},\ bx \equiv 1 \pmod{p} \Rightarrow ax \equiv bx \pmod{p} (ax - bx) \vdots p \Rightarrow (a - b) x \vdots p $, но $x \in [1; p-1] \Rightarrow (a-b) \vdots p$, но $a, b \in [1;p -1] \Rightarrow a = b \Rightarrow $ числа $2, 3, \ldots, p - 2$ разбиваются на пары чисел $(a, b)$ тч $a \cdot b \equiv 1 \pmod{p}$

Тогда:
\[
    (p-1)! \equiv 1 \cdot 2 \cdot 3 \cdot \ldots \cdot (p - 2) \cdot (p - 1) \equiv 1 (p - 1) \pmod{p} \equiv -1 \pmod{p}
\]

Случай $p = 2$ на упражнение читателю
\end{Proof}

\subsection{Задача 17}
\begin{example}[Задача]
    \[
        \forall n\ \exists a, d \in \N : a, a+d, a + 2d, \ldots, a + (n - 1)d \text{ --- попарно взаимно просты }
    \]
\end{example}
\begin{Proof}

\end{Proof}

\section{Листок 5}

\subsection{КТО}

\[
    \begin{dcases}
        x \equiv a_1 \pmod{m_1}\\
        \vdots \\
        x \equiv a_k \pmod{m_k} .
    \end{dcases}
    \qquad
\text{
\begin{tabular}{@{}l}
$(m_i, m_j) = 1$ при $i \neq j$ \\
$M = m_1 \cdot \ldots \cdot m_k, \quad M_i = \frac{M}{m_i}$\\
$b_i$ --- числа, такие что $M_{i}b_i \equiv a_i \pmod{m_i} $
\end{tabular}
}
\]

Тогда $\exists !$ решение $(*): x \equiv M_{1}b_1 + \ldots + M_{k}b_k \pmod{M}$

\subsection{Задача 4}
\begin{example}[Задача 4]
    \[
        \begin{dcases}
            x \equiv 12 \pmod{15}\\
            x \equiv 8 \pmod{17}\\
            x \equiv 3 \pmod{8}
        \end{dcases}
    \]
\end{example}
\begin{remark}[Решение]
    \[
        M = 15 \cdot 8 \cdot 17 = 2040
    \]
    \[
        M_1 = 136
    \]
    \[
        136b_1 \equiv 12 \pmod{15}
    \]
    \[
        b_1 \equiv 12 \pmod{15}
    \]
    \[
        M_2 = 120
    \]
    \[
        120 b_2 \equiv 8 \pmod{17}
    \]
    \[
        b_2 \equiv 8 \pmod{17}
    \]
    \[
        M_3 = 255
    \]
    \[
        255b_3 \equiv 3 \pmod{8}
    \]
    \[
        -b_3 \equiv 3 \pmod{8}
    \]
    \[
        b_3 \equiv -3 \pmod{8}
    \]
    \[
        x \equiv 136 \cdot 12 + 120 \cdot 8 - 255 \cdot 3 = 1827 \pmod{2040}
    \]
\end{remark}
\begin{remark}[Ответ]
    1827
\end{remark}

\begin{example}
    \[
        \begin{dcases}
            x \equiv 7 \pmod{9}\\
            x \equiv 11 \pmod{13}\\
            x \equiv 6 \pmod{8}
        \end{dcases}
        \Leftrightarrow
        \begin{dcases}
            x \equiv -2 \pmod{9}\\
            x \equiv -2 \pmod{13}\\
            x \equiv -2 \pmod{8}
        \end{dcases}
        \Leftrightarrow
        \begin{dcases}
            (x + 2) \vdots 9 \\
            (x + 2) \vdots 13\
            (x + 2) \vdots 8
        \end{dcases}
        \Leftrightarrow
    \]
    \[
        \Leftrightarrow
        (x + 2)\vdots 936 \Leftrightarrow x \equiv -2 \pmod{936} \equiv -2 \equiv 934
    \]
\end{example} 

\subsection{Задача 5}
\begin{example}[Задача 5]
    \[
        f(x) = x^4 + 2x^3 + 8x + 9 \equiv 0 \pmod{35}
    \]
    Сколько решений и найти все решения
\end{example}
\begin{remark}[Решение]
    \[
        f(x) \equiv 0 \pmod{35}
        \Leftrightarrow
        \begin{dcases}
            f(x) \equiv 0 \pmod{5}\\
            f(x) \equiv 0 \pmod{7}
        \end{dcases}
    \]
    1) Решим $f(x) \equiv 0 \pmod{5}$

    \[
        x \equiv 0 : f(x) \equiv 9 \not\equiv 0 \pmod{5} \ \bot
    \]
    \[
        x \equiv 1 : f(x) \equiv 20 \equiv 0 \pmod{5}\ \checkmark
    \]
    \[
        x \equiv 2 : f(x) \equiv 57 \not\equiv 0 \pmod {5}\ \bot
    \]
    \[
        x \equiv 3 : f(x) \equiv 168 \not\equiv 0 \pmod{5}\ \bot
    \]
    \[
        x \equiv 4 : f(x) \equiv f(-1) \equiv 0 \pmod{5}\ \checkmark
    \]
    2) Решим $f(x) \equiv 0 \pmod{7}$
    \[
        x \equiv 0: f(x) \equiv 9 \not\equiv 0 \pmod{7}\ \bot
    \]
    \[
        x \equiv 1: f(x) \equiv 20 \not\equiv 0 \pmod{7}\ \bot
    \]
    \[
        x \equiv 2: f(x) \equiv 57 \not\equiv 0 \pmod{7}\ \bot
    \]
    \[
        x \equiv 3: f(x) \equiv 168 \equiv 0 \pmod{7}\ \checkmark
    \]
    \[
        x \equiv 4: f(x) \equiv f(-3) \equiv 12 \not\equiv 0 \pmod{7}\ \bot
    \]
    \[
        x \equiv 5: f(x) \equiv f(-2) \equiv -7 \equiv 0 \pmod{7}\ \checkmark
    \]
    \[
        x \equiv 6: f(x) \equiv f(-1) \equiv 0 \pmod{7}\ \checkmark
    \]
    \[
        \Leftrightarrow
        \begin{dcases}
            \left[
            \begin{array}{c}
                x \equiv 1 \pmod{5}\\
                x \equiv 4 \pmod{5}
            \end{array}
            \right.\\
            \left[
            \begin{array}{c}
                x \equiv 3 \pmod{7}\\
                x \equiv 5 \pmod{7}\\
                x \equiv 6 \pmod{7}
            \end{array}
            \right.
        \end{dcases}
        \Leftrightarrow
        \left[
        \begin{array}{c}
            \begin{dcases}
                x \equiv 1 \pmod{5}\\
                x \equiv 3 \pmod{7}
            \end{dcases}\\
            \begin{dcases}
                x \equiv 1 \pmod{5}\\
                x \equiv 5 \pmod{7}
            \end{dcases}\\
            \begin{dcases}
                x \equiv 1 \pmod{5}\\
                x \equiv 6 \pmod{7}
            \end{dcases}\\
            \begin{dcases}
                x \equiv 4 \pmod{5}\\
                x \equiv 3 \pmod{7}
            \end{dcases}\\
            \begin{dcases}
                x \equiv 4 \pmod{5}\\
                x \equiv 5 \pmod{7}
            \end{dcases}\\
            \begin{dcases}
                x \equiv 4 \pmod{5}\\
                x \equiv 6 \pmod{7}
            \end{dcases}
        \end{array}
        \right.
        \Leftrightarrow
        \underset{\text{ответ}}{\left[
        \begin{array}{c}
            x \equiv 31 \pmod{35}\\
            x \equiv 26 \pmod{35}\\
            x \equiv 6 \pmod{35}\\
            x \equiv 24 \pmod{35}\\
            x \equiv 19 \pmod{35}\\
            x \equiv 34 \pmod{35}
        \end{array}
        \right.}
    \]
\end{remark}

\begin{remark}
    \[
        ax \equiv b \pmod{n} \Rightarrow ax + by = c 
    \]
    \[
        ax \equiv b \pmod{n} \Leftrightarrow |ax - b| \vdots n \Leftrightarrow ax - b = ny \Leftrightarrow ax - ny = b
    \]
\end{remark}

\subsubsection{Арифметика предела функции}

Если $f(x) \xrightarrow[x \to x_0]{} A,\ g(x) \xrightarrow[x \to x_0]{} B$, тогда:

\begin{enumerate}
    \item $f(x) \pm g(x) \xrightarrow[x \to x_0]{} A \pm B$
    \item $f(x) \cdot g(x) \xrightarrow[x \to x_0]{} A \cdot B$
    \item $\frac{f(x)}{g(x)} \xrightarrow[x \to x_0]{} \frac{A}{B}, B \neq 0$
    \item $\sqrt[k]{f(x)} \xrightarrow[x \to x_0]{} \sqrt[k]{A} $
\end{enumerate}

\begin{Proof}[Доказательство 1]
    Возьмем произвольную последовательность \fbox{$\{x_n\}_{n \in \N}: x_n \xrightarrow[n \to \infty ]{} x_0,\ x_n \neq x_0$}. Тогда по Гейне:
    \[
        f(x_n) \xrightarrow[n \to \infty ]{} A,\ g(x_n) \xrightarrow[n \to \infty ]{} B
    \]
    По арифметике предела последовательности:
    \[
        \fbox{$f(x_n) - g(x_n) \xrightarrow[n \to \infty ]{} A - B$}
    \]
    Тогда по Гейне:
    \[
        \Lim{x}{x_0} (f(x) - g(x)) = A - B
    \]
\end{Proof}

\begin{remark}
    Остальные свойства доказываются аналогично
\end{remark}

\subsubsection{Теорема о зажатой функции}
\begin{theorem}
    \[
        \exists \delta_0\ \forall x \in \overset{\circ}{U}_{\delta_0}(x_0)\ : f(x) \leq h(x) \leq g(x)
    \]
    \[
        \Lim{x}{x_0} f(x) = \Lim{x}{x_0} g(x) = A, \text{ тогда } \Lim{x}{x_0} h(x) = A
    \]
\end{theorem}

\begin{Proof}
    Возьмем \fbox{произвольную $\{x_n\}_{n \in \N} : x_n \xrightarrow[n \to \infty ]{} x_0,\ x_n \neq x_0$}

    \[
        \exists N_0\ \forall n > N_0\ x_0 \in \overset{\circ}{U}_{\delta_0}(x_0)
    \]
    \[
        \forall n > N_0\ f(x_n) \leq h(x_n) \leq g(x_n)
    \]
    \[
        \underbrace{f(x_n)}_{\xrightarrow{n \to \infty} A} 
        \leq h(x_n) \leq
        \underbrace{g(x_n)}_{\xrightarrow{n \to \infty} A} \Rightarrow \fbox{$h(x_n) \xrightarrow[n \to \infty ]{} A$} \Rightarrow h(x) \xrightarrow[x \to x_0]{} A
    \]
\end{Proof}

\subsubsection{Предел сложной функции}
Вернемся к арифметике предела функции. Хотим:

\begin{enumerate}
    \item[5.] \sout{Если $\Lim{x}{x_0} f(x) = A, \Lim{y}{A} h(y) = A$, то $\Lim{x}{x_0} h(f(x)) = A$}. Это неправда
\end{enumerate}

Контрпример:
\[
    f(x) \equiv 0, x_n \to 0
\]
\[
    \Lim{x}{0} f(x) \to 0
\]
\[
    h(y) = |sign\ y|,\ sign(y) = \\
    \begin{dcases}
        1, &y > 0\\
        0, &y = 0\\
        -1, &y < 0
    \end{dcases}
\]
\[
    \Lim{y}{0} h(y) = 1
\]
Хотим: 
\[
    h(f(x)) \xrightarrow[x \to 0]{} 1
\]
Но: 
\[
    h(f(x)) \implies h(0) \implies 0 \text{. Противоречие}
\]
Мы всегда смотрели на аргумент из проколотой окрестности, а $f(x)$ попал в центр окрестности

\begin{theorem}[Теорема о пределе сложной функции]
    Если $f(x) \xrightarrow[x \to x_0]{} A,\ h(y) \xrightarrow[y \to A]{} h(A)$, тогда: 
    \[
        h(f(x)) \xrightarrow[x \to x_0]{} h(A)
    \]
\end{theorem}
\begin{Proof}
\[
    \forall \omega > 0\ \exists \delta = \delta (\omega )\ \forall x \in \overset{\circ}{U}_{\delta }(x_0)\ |f(x) - A| < \omega
\]
\[
    \forall \eps > 0\ \exists \eta = \eta (\eps )\ \forall y \in \overset{\circ}{U}_{\eta }(A)\ |h(y) - h(A)| < \eps 
\]
Добавляем в окрестность $\eta_0$, так как неравенство выполнено $(0 < \eps )$
\[
    \forall \eps > 0\ \exists \eta = \eta (\eps )\ \forall y \in {U}_{\eta }(A)\ |h(y) - h(A)| < \eps 
\]
Нужно доказать: 
\[
    \forall \eps > 0\ \exists \delta_1 = \delta_1(\eps )\ \forall x \in \overset{\circ}{U}_{\delta_1}(x_0)\ |h(f(x)) - h(A)| < \eps 
\]
Верно, если $f(x) \in U_{\eta (\eps )}(A)$
\[
    \exists \delta = \delta (\eta (\eps ))\ \forall x \in \overset{\circ}{U}_{\delta }(x_0)\ |f(x) - A| < \eta (\eps )
\]
Возьмем $\delta_1(\eps ) = \delta (\eta (\eps ))$
\end{Proof}

\begin{example}
    \[
        \Lim{x}{0} \frac{\sin x}{x} = 1
    \]
    \[
        \Lim{x}{0} \frac{\sin 5x}{x} =\ ?
    \]
    \[
        \Lim{x}{0} \frac{\sin 5x}{x} = \Lim{x}{0} \frac{\sin 5x}{5x} \cdot 5 = \Lim{x}{0} \frac{\sin t}{t} \cdot 5 = 1 \cdot 5 = 5
    \]
    Рассмотрим обоснование: 
    \[
        h(y) = \frac{\sin y}{y},\ f(x) = 5x
    \]
    Кажется нужно искать $h(f(x))$ и тогда: 
    \[
        \Lim{x}{0} h(f(x)) \sim h(0) \sim 1
    \]
    В чем кроется замена переменной: если найдем предел $h(y)$, то $h(f(x))$ будет такой же. Но эта теорема неверна (см контрпример выше). Тогда требуются доп ограничения (спойлер: теорема ниже)
\end{example}

\begin{theorem}[Вторая теорема о пределе сложной функции]
    $f(x) \xrightarrow[x \to x_0]{} f(x_0)$, $f(x)$ --- строго монотонна в какой-то окрестности $x_0$, $h(y) \xrightarrow[y \to f(x_0)]{} A$

    Тогда:
    \[
        h(f(x)) \xrightarrow[x \to x_0]{} A
    \]
\end{theorem}

\begin{Proof}
    \[
        \forall \omega > 0\ \exists \delta = \delta (\omega )\ \forall x \in \overset{\circ}{U}_{\delta }(x_0)\ |f(x) - f(x_0)| < \omega
    \]
    \[
        \forall \omega > 0\ \exists \delta = \delta (\omega )\ \forall x \in \overset{\circ}{U}_{\delta }(x_0)\ 0 < |f(x) - f(x_0)| < \omega
    \]
    \[
        \forall \eps > 0\ \exists \eta = \eta (\eps )\ \forall y \in \overset{\circ}{U}_{\eta }(f(x_0))\ |h(y) - A| < \eps 
    \]
    Нужно: 
    \[
        \forall \eps > 0\ \exists \delta_1 = \delta_1 (\eps )\ \forall y \in \overset{\circ}{U}_{\delta_1 }(x_0)\ |h(f(x)) - A| < \eps 
    \]
    Верно при $f(x) \in \overset{\circ}{U}_{\eta(\eps ) }(f(x_0)) \Rightarrow 0 < |f(x) - f(x_0)| < \eta(\eps )$. Это верно при $x \in \overset{\circ}{U}_{\delta(\eta (\eps )) }(x_0) $

    Возьмем $\delta_1(\eps ) = \delta (\eta (\eps ))$
\end{Proof}

\begin{definition}
    Функция $f(x)$ называется \textbf{непрерывной в точке} $x_0$, если: 
    \[
        \Lim{x}{x_0} f(x) = f(x_0)
    \]
    \[
        \forall \eps > 0\ \exists \delta > 0\ \forall x \in U_{\delta }(x_0)\ |f(x) - f(x_0)| < \eps  
    \]
\end{definition}

\begin{example}[]
    \[
        \fbox{$x \in \text{I} \Rightarrow \sin x < x$}
    \]
\end{example}
\begin{Proof}
$\ldots$
\end{Proof}

\begin{example}
    $\sin x$ непрерывен на $\R$, т.е $\Lim{x}{x_0} \sin x = \sin x_0$

    Хотим: $(\sin x - \sin x_0)$ --- б.м. Будем рассматривать по модулю, так как неважно, само выражение б.м. или модуль от него. Используя \fbox{$x \in \text{I} \Rightarrow \sin x < x$}:
    \[
        |\sin x - \sin x_0| = \l|2 \cdot \sin \frac{x - x_0}{2} \cdot \cos \frac{x + x_0}{2} \r| \leq 2 \cdot \sin \frac{|x - x_0|}{2} < 2 \cdot \frac{|x - x_0|}{2} = |x - x_0|
    \]
    \[
        |x - x_0| \xrightarrow[x \to x_0]{} 0
    \]
\end{example}

\begin{example}
    $\cos x$ непрерывен на $\R$, т.е $\Lim{x}{x_0} \cos x = \Lim{x}{x_0} \sin \underbrace{\l(\frac{\pi }{2} - x\r)}_{\frac{\pi}{2} - x_0} = \sin \left( \frac{\pi }{2} -x_0 \right) = \cos x_0$
\end{example}

\begin{example}
    $\tg x$ непрерывен на $D_f$, т.е $\underset{x_0 \neq \frac{\pi }{2} + \pi n, n \in \Z}{\Lim{x}{x_0}} \tg x = \frac{\overset{\to \sin x_0}{\sin x}}{\underset{\to \cos x_0 \neq 0}{\cos x}} \xrightarrow[]{} \frac{\sin x_0}{\cos x_0} = \tg x_0$
\end{example}

\begin{definition}
    {\bf Односторонним пределом} $\Lim{x}{x_{0-}^+} f(x) = A$ называется такое число, что: 
    \begin{itemize}
        \item $\Lim{x}{x_{0}^+} f(x) = A$
        \[
            \forall \eps > 0\ \exists \delta = \delta (\eps )\ \forall x : 0 < x - x_0 < \delta (\eps )\ |f(x) - A| < \eps \text{ --- \textbf{правосторонний предел}}
        \]
        \item $\Lim{x}{x_{0-}} f(x) = A$
        \[
            \forall \eps > 0\ \exists \delta = \delta (\eps )\ \forall x : 0 < x_0 - x < \delta (\eps )\ |f(x) - A| < \eps \text{ --- \textbf{левосторонний предел}}
        \]
    \end{itemize}
\end{definition}

\begin{remark}[Замечание]
    \[
        \exists \Lim{x}{x_0} f(x) \Leftrightarrow \exists \Lim{x}{x_{0}^+} f(x) = \exists \Lim{x}{x_{0-}} f(x)
    \]
    Доказательство тривиальное (в обратную сторону нужно взять минимум из двух $\delta $)
\end{remark}

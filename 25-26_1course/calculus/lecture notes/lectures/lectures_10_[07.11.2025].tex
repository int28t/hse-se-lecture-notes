\subsection{Классификация разрывов}

{\Large \text{\bf 1 рода: } $\exists \Lim{x}{x_{0}\pm} f(x)$}

{\bf 1.1 --- Устранимый} (устранимый, значит вы можете переопределить/доопределить функцию в точке, и она станет непрерывной в этой точке)
\[
    \Lim{x}{x_{0}^-} f(x) = \Lim{x}{x_{0}^+} f(x) \neq f(x_0)
\]

\begin{example}
    \[
        f(x) = \frac{\sin x}{x}
    \]
    Функция неопределена в 0, можем доопределить единицей и тогда она станет непрерывной в 0
\end{example}
\begin{example}
    \[
        f(x) = |sign(x)|
    \]
    Функция в 0 разрывна, так как предел есть и равен 1. Можем переопределить функцию в 0 и она станет непрерывной
\end{example}

{\bf 1.2 --- Скачок}
\[
    \Lim{x}{x_{0}^-} f(x) \neq \Lim{x}{x_{0}^+} f(x)
\]
\begin{figure}[h]
  \centering
  \includegraphics[width=0.3\textwidth]{lectures/files/1ctur124_10_07.11.2025-22-15-24.png}
  \caption{Скачок на примере функции распределения}
  \label{fig:1ctur124_10_07.11.2025-22-15-24.png}
\end{figure}
\begin{remark}
    Доказать, что если функция монотонная, то у нее есть разрывы только типа скачок
\end{remark}

{\Large \text{\bf 2 рода: } $\neg \exists \Lim{x}{x_{0}\pm} f(x)$}

\begin{remark}
    У таких функций нет предела или они бесконечно большие
\end{remark}

{\bf 2.1 --- Полюс} 
\[
    \Lim{x}{x_{0}^-} f(x) = \pm \infty,\ \fbox{$x_0$ - полюс}
\]

{\bf 2.2 --- Никак((}

\[
    \Lim{x}{2^+} f(x) = -\infty
\]
\[
    \forall M > 0\ \exists \delta > 0\ \forall x \in (2;2+\delta)\ f(x) < -M
\]

\begin{definition}
    $f(x)$ огр. при $x \to x_0$, если 
    \[
        \exists C\ \exists \delta_0\ \forall x \in \overset{\circ}{U}_{\delta_0 } (x_0)\ |f(x)| \leq C
    \]
\end{definition}

\begin{definition}
    $f(x)$ отделима от нуля при $x \to x_0$, если:
    \[
        \exists c\ \exists \delta_0\ \forall x \in \overset{\circ}{{U}}_{\delta_0(x_0)} |f(x)| \geq c
    \]
\end{definition}

\subsection{Асимптоты}
\subsubsection{Вертикальная}
\begin{definition}
    Прямая $x = a$ называется \fbox{вертикальной асимптотой}, если $\Lim{x}{a^\infty} f(x) = \pm \infty$ или $\Lim{x}{a^+} f(x) = \pm \infty$
\end{definition}

\begin{example}
    \[
        f(x) = \frac{x+3}{x-2}
    \]
    \fbox{$x = 2$ --- вертикальная асимптота} $D_f = \R^{\setminus \{2\}} $
    \[
        \Lim{x}{2^+} \frac{x + 3}{x - 2} = \frac{5}{0^+} = +\infty
    \]
    \[
        \Lim{x}{2^-} \frac{x + 3}{x - 2} = \frac{5}{0^-} = -\infty
    \]
    $x_0 \neq 2$ не годится для верт асимптоты
    \[
        \Lim{x}{x_0} f(x) = f(x_0) \in \R
    \]
\end{example}

\begin{example}[Пример (бесконечное число верт асимптот)]
    \[
        f(x) = \frac{1}{1 - \{x\}}
    \]
\end{example}
\begin{figure}[h]
  \centering
  \includegraphics[width=0.35\textwidth]{lectures/files/image201817.png}
  \label{fig:20265ctur545_10_07.11.2025]-18-16-45.png}
\end{figure}
\begin{example}[]
    $\{x\} = x - [x]$ --- дробная часть

    $[x] - \max \{k \in \Z : k \leq x\}$ --- целая часть
\end{example}

\subsubsection{Горизонтальная}

\begin{definition}
    $y = b$ --- горизонтальная асимптота при $x \to \pm \infty$, если
    \[
        \Lim{x}{\pm \infty} f(x) = b
    \]
\end{definition}

\subsubsection{Наклонная}

\begin{definition}
    $y = kx + b$ --- наклонная асимптота при $x \to \pm \infty$, если 
    \[
        \Lim{x}{\pm \infty} (f(x) - (kx + b)) = 0
    \]
\end{definition}

\begin{remark}
    Горизонтальная --- частный случай наклонной
\end{remark}

\begin{theorem}
    $y = kx + b$ --- наклонная асимптота при $x \to \pm \infty$

    \[
        \Updownarrow
    \]

    \[
        \begin{dcases}
            \exists \Lim{x}{\pm \infty} \frac{f(x)}{x} = k\\
            \exists \Lim{x}{\pm \infty} (f(x) - kx) = b
        \end{dcases}
    \]
\end{theorem}

\begin{Proof}
    {\bf $\Rightarrow$}
    \[
        f(x) - (kx + b) = \alpha(x),\ \alpha(x) \to 0
    \]
    \[
        \frac{f(x)}{x} = k + \frac{b}{x} + \frac{\alpha(x)}{x} \xrightarrow[x \to \pm \infty ]{} k
    \]
    \[
        f(x) - kx = b + \alpha(x) \xrightarrow[x \to \pm \infty ]{}  b
    \]
    {\bf $\Leftarrow$}
    \[
        f(x) - kx = b + \beta (x), \beta (x) \to 0
    \]
    \[
        f(x) - (kx + b) = \beta (x)
    \]
    \[
        \Lim{x}{\pm \infty} (f(x) - (kx + b)) = 0
    \]
\end{Proof}

\begin{example}
    \[
        y = \sqrt{x^2 + 7x + 14} 
    \]
    Вертикальных нет, так как $\forall x_0 \Lim{x}{x_0} f(x) = f(x_0)$
    Наклонная:
    \[
        \Lim{x}{\pm \infty} \frac{\sqrt{x^2 + 7x + 14}}{x}
    \]
    \begin{itemize}
        \item[1)] 
        $
            \Lim{x}{+\infty } \sqrt{1 + \frac{7}{x} + \frac{14}{x^2}} = 1
        $
        \item[2)]
        $
            \Lim{x}{-\infty } \sqrt{1 + \frac{7}{x} + \frac{14}{x^2}} = -1
        $
    \end{itemize}
    \[
        \Lim{x}{\pm \infty} (\sqrt{x^2 + 7x + 14} \mp x)
    \]
    \begin{itemize} 
        \item[1)]
        $
            \Lim{x}{+\infty} \sqrt{x^2 + 7x + 14} - x = \frac{7}{2} 
        $
        \item[2)]
        $
            \Lim{x}{-\infty} \sqrt{x^2 + 7x + 14} + x = - \frac{7}{2} 
        $
    \end{itemize}
\end{example}

\subsection{O --- символика}

\subsubsection{O малое}

\begin{definition}
    $f(x) = \overline{o}(g(x))$ при $x \to x_0\ (x_0 \in \overline{\R} = \R \cup \{\pm \infty \})$, если

    \[
        \frac{f(x)}{g(x)} = \text{бесконечно малая при } x \to x_0
    \]
\end{definition}

\subsubsection{O большое}

\begin{definition}
    $f(x) = \underline{O}(g(x))$ при $x \to x_0\ (x_0 \in \overline{\R})$, если

    \[
        \frac{f(x)}{g(x)} = \text{ограниченная при } x \to x_0
    \]
\end{definition}

\begin{example}
    \[
        x^2 = \overline{o}(x^3)\ \text{при } x \to 0\ \text{ :( }
    \]
    \[
        \frac{x^2}{x^3} = \frac{1}{x} \to \infty
    \]
\end{example}

\begin{remark}
    1. Впредь бесконечно малые будем обозначать $\overline{o}(1)$

    \[
        \overline{o}(1) + \overline{o}(1) = \overline{o}(1)
    \]

    2. Впредь ограниченные будем обозначать $\underline{O}(1)$
    \[
        \overline{o}(1) + \underline{O}(1) = \underline{O}(1)
    \]
\end{remark}

\subsection{Замечательные пределы}

\begin{theorem}
    $\cos x$ непрерывен на $\R$
\end{theorem}

\begin{Proof}
    \[
        \Lim{x}{x_0} \cos x = \cos x_0
    \]
    \[
        \Updownarrow
    \]
    \[
        \cos x - \cos x_0 = \overline{o}(1) \text{ при }  x \to x_0
    \]
    \[
        \l| -2 \cdot \sin \frac{x + x_0}{2} \cdot \sin \frac{x - x_0}{2} \r| \leq 2 \cdot 1 \cdot \l|\frac{x - x_0}{2}\r| \to 0
    \]
\end{Proof}

\fbox{1} $\Lim{x}{0} \frac{\sin x}{x} = 1$

\begin{Proof}
    \[
        S_{\triangle OAB} < S_{\sector OAB} < S_{\triangle OAC}
    \]
    \[
        \frac{\sin x}{2} < \pi \cdot \frac{x}{2 \pi} < \frac{tg\ x}{2}
    \]
\end{Proof}

\begin{figure}[h]
  \centering
  \includegraphics[width=0.4\textwidth]{lectures/files/lec_10_14-13-30.png}
  \caption{К доказательству}
  \label{fig:lec_10_14-13-30.png}
\end{figure}

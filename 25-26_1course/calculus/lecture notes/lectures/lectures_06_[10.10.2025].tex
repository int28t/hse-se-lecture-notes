\subsubsection{Постоянная Эйлера}

\subsection{Предел рекуррентно заданных последовательностей}
Для того чтобы найти пример рекуррентно заданных последовательностей

\textbf{Шаг 1:} Проверить, что последовательность сходится (можно пытаться делать через теорему Вейерштрасса или критерий Коши)

\textbf{Шаг 2:} Найти предел используя арифметику пределов

\subsection{Подпоследовательности}

\begin{definition}
    \textbf{Подпоследовательностью} последовательности $\{a_n\}$ называется последовательность $\{b_k\}$, такая что $b_k = a_{n_k}$, где $n_k$ - строго возрастающая последовательность номеров ($\N$)
\end{definition}

\begin{remark}[Замечание]
    $$ n_k \geq k $$
\end{remark}

\begin{example}
    $ a_n = \sin \frac{\pi n}{2} $

    $ b_k = a_{4k} = \sin 2\pi k \equiv 0$

    $ c_k = a_{4k - 1} = \sin \frac{\pi (4k - 1)}{2} \equiv -1$

    $ d_k = a_{2k + 1} = \sin \frac{\pi (2k + 1)}{2} = (-1)^k  $
\end{example}

\subsubsection{Частичные пределы}
\begin{definition}
    \textbf{Частичный предел -} конечный или бесконечный предел подпоследовательности
\end{definition}

\begin{remark}
У какой (ограниченной) последовательности бесконечное число частичных пределов?
\end{remark}
\subsubsection{Предельные точки} 

\begin{definition}
    Предельная точка последовательности $\{a_n\}_{n \in \N}$ --- число $A \in \R$, такое что:
    \[
        \forall \eps > 0 \text{ в } U_{\eps}(A) \text{ находится } \infty  \text{ число членов } \{a_n\}
    \]
\end{definition}

\begin{theorem}
    Конечный частичный предел $\Leftrightarrow$ предельная точка
\end{theorem}

\begin{Proof}
    {\bf ($\Rightarrow$):}

    $\Lim{k}{\infty} a_{n_k} = A$
    \[
        \forall \eps > 0\ \exists K(\eps)\ \forall k > K(\eps)\ a_{n_k} \in U_{\eps}(A) 
    \]

    {\bf ($\Leftarrow$):}

    $a$ --- предельная точка

    Возьмем $\eps = 1$

    $U_1(A)$ Возьмем 1 член, его номер объявим $n_1$

    Возьмем $\eps = \frac{1}{2}$

    $U_{\frac{1}{2}}(A)$ Возьмем член с номером $> n_1$. Его номер объявим $n_2$

    $\ldots$

    Возьмем $\eps = \frac{1}{k}$

    $U_{\frac{1}{k}}(A)$ Возьмем член с номером $> n_{k-1}$. Его номер объявим $n_k$

    $\{a_{n_k}\}_{k \in \N} $

    Рассмотрим $\{a_{n_k}\}$:
    \[
        \underbrace{A - \frac{1}{k}}_{\xrightarrow{k \to \infty} A} 
        < \{a_{n_k}\} < 
        \underbrace{A + \frac{1}{k}}_{\xrightarrow{k \to \infty} A}
    \]
\end{Proof}

\subsubsection{Свойства частичных пределов}
\newpage
\subsubsection{Теорема Больцано-Вейерштрасса}

\begin{theorem}
Если $a_n \underset{n \to \infty}{\to} A, \text{ то}\ \forall n_k\ b_k = a_{n_k} \underset{n \to \infty}{\to} A $ 
\end{theorem}

\begin{Proof}
    $$ \forall \eps > 0\ \exists N(\eps)\ \forall n > N(\eps)\ |a_n - A| < \eps $$
    Хотим
    $$ \forall \eps > 0\ \exists K(\eps)\ \forall k > k(\eps)\ |a_{n_k} - A| < \eps $$
    Возьмем $K(\eps) = N(\eps)$
    \[\begin{array}{l}
        \l.
        \begin{array}{l}
            \forall k > N(\eps)\\
            \text{т.к } n_k \geq k
        \end{array}
        \r]
        \l. \Rightarrow \r.
        \l.
        n_k > N(\eps). \text{ Тогда } |a_{n_k} - A| < \eps \text{ истина}
        \r.
    \end{array}\]
\end{Proof}

\begin{figure}[h]
  \centering
  \includegraphics[width=0.8\textwidth]{lectures/files/lec06-11-25-35.png}
  \label{fig:lec06-11-25-35.png}
\end{figure}

\begin{theorem}[Теорема Больцано-Вейерштрасса]
    Если последовательность ограничена, то у нее есть сходящаяся подпоследовательность
\end{theorem}

\begin{Proof}
    $c_n$ -- огр. Докажем что у нее есть сходящаяся подпоследовательность

    $A = inf\{c_n\}$
    
    $B = sup\{c_n\}$

    $A, B \in \R$

    Рассмотрим отрезок $[a_1,b_1] = [A,B]$

    \textbf{Шаг 1:} разделим отрезок $[a_1,b_1]$ пополам. В какой-то половине (одной или обеих) лежит бесконечное число членов $c_n$. Выберем в качестве $[a_2,b_2]$ эту половинку (если в обеих бесконечное число, то любую)

    \textbf{Шаг 2:} $\ldots$

    Получаем некоторую последовательность подотрезков $\{[a_k,b_k]\}_{k \in \N}$

    На первом шаге выберем какой-то член $c_n \in [a_1,b_1]$. Его номер возьмем в качестве~$n_1$

    $\exists$ член последовательности $\in [a_2,b_2]$ такой, что его номер $> n_1$ (Это следует из того, что на каждом шаге мы выбираем отрезок, содержащий бесконечное число членов). Его возьмем в качестве $n_2$

    $\ldots$

    Таким образом, параллельно строя последовательность отрезков, мы построили подпоследовательность $\{C_{n_k}\}_{k\in\N}$

    Рассмотрим $\{a_k\}_{k\in\N}$. Она неубывает, ограничена сверху $B$. \\По т. Вейерштрасса: $\exists \Lim{k}{\infty} a_k = A'$

    $\{b_k\}_{k\in\N}$. Она невозрастает, ограничена снизу $A$. \\По т. Вейерштрасса: $\exists \Lim{k}{\infty} b_k = B'$

    На каждом шаге мы вдвое уменьшаем длину отрезка. Выпишем общую формулу:
    
    $$b_k - a_k = \frac{B-A}{2^{k-1}} \text{, для строгости необходимо доказать по индукции}$$
    Также по арифметике пределов сходящихся последовательностей $b_k,\ a_k\ (b_k \to~B',\ a_k \to~A')$, их разность стремится к $B' - A'$ при $k \to \infty$

    $\frac{B-A}{2^{k-1}} \to 0,\ k \to \infty \Rightarrow B' - A' = 0 \Rightarrow B' = A'$

    Получаем, что у $a_k$ и $b_k$ один и тот же предел

    Заметим, что $a_k \leq c_{n_k} \leq b_k\ \forall k \in \N$. По теореме о сходящейся последовательности:

    $a_k \to A', b_k \to B', A' = B'$ при $k \to \infty$. Следовательно: $\lim_{k \to \infty} c_{n_k} = A' = B'$
\end{Proof}

\begin{example}
    $c_n = (-1)^n$
\end{example}

\begin{figure}[h]
  \centering
  \includegraphics[width=0.9\textwidth]{lectures/files/lec6-11-28-20.png}
  \label{fig:lec6-11-28-20.png}
\end{figure}

\subsubsection{Критерий Коши}

\begin{definition}
    Последовательность $a_n$ называется фундаментальной, если:
    \[
        \forall \varepsilon > 0\ \exists N(\varepsilon) \in \N\ \forall n, m > N(\varepsilon)\ |a_n - a_m| < \varepsilon 
    \]
    \[
        \Updownarrow
    \]
    \[
        \forall \varepsilon > 0\ \exists N(\varepsilon) \in \N\ \forall n > N(\varepsilon)\ \forall k \in \N\ |a_{n+k} - a_n| < \varepsilon 
    \]
\end{definition}

\begin{theorem}[Теорема о критерии Коши]
    Последовательность $a_n$ сходится $\Leftrightarrow$ последовательность $a_n$ фундаментальна
\end{theorem}

\begin{Proof}
{\bf ($\Rightarrow$):}

$\Lim{n}{\infty} a_n = A$
\[
    \forall \eps > 0\ \exists N_1(\eps) \in \N\ \forall n > N_1(\eps)\ |a_n - A| < \eps 
\]
Хотим:
\[
    \forall \varepsilon > 0\ \exists N_2(\varepsilon) \in \N\ \forall n, m > N_2(\varepsilon)\ |a_n - a_m| < \varepsilon
\]
\[
    \forall \varepsilon > 0\ \exists N_2(\varepsilon) \in \N\ \forall n, m > N_2(\varepsilon)\ |(a_n - A) + (A - a_m)| < \varepsilon
\]
\[
    \forall \varepsilon > 0\ \exists N_2(\varepsilon) \in \N\ \forall n, m > N_2(\varepsilon)\ \underbrace{|a_n - A|}_{< \frac{\eps}{2}} + \underbrace{|a_m - A|}_{< \frac{\eps}{2}} < \varepsilon
\]
Это выполняется $\forall n > N_1\l(\frac{\eps}{2}\r)$ и $\forall m > N_1\l(\frac{\eps}{2}\r)$

Возьмем $N_2(\eps) = N_1\l(\frac{\eps}{2}\r)$

{\bf ($\Leftarrow$):}

1) Фундаментальные последовательности ограниченны

Возьмем $\eps_0 = 1, N(1)$
\[
    \forall n, m > N(1)\ |a_n - a_m| < 1
\]
Возьмем $m_0 = N(1) + 1$
\[
    \forall n > N(1), a_n \in U_1(a_{m_0})
\]
\end{Proof}

\begin{figure}[h]
  \centering
  \includegraphics[width=0.55\textwidth]{lectures/files/638_06_10.10.2025-23-21-38.png}
  \label{fig:638_06_10.10.2025-23-21-38.png}
\end{figure}

\begin{Proof}[]
2) По теореме Больцано-Вейерштрасса:
\[
    \exists a_{n_k} \xrightarrow[k \to \infty]{} A
\]
Доказать:
\[
    a_n \xrightarrow[n \to \infty]{} A
\]
Имеем:
\[
    \forall \varepsilon > 0\ \exists N_1(\varepsilon) \ \forall n, m > N_1(\varepsilon)\ |a_n - a_m| < \varepsilon 
\]
\[
    \forall \eps > 0\ \exists K(\eps)\ \forall k > K(\eps)\ |a_{n_k} - A| < \eps
\]
Хотим:
\[
    \forall \eps > 0\ \exists N_2(\eps)\ \forall n > N_2(\eps)\ |a_n - A| < \eps 
\]
\[
    \forall \eps > 0\ \exists N_2(\eps)\ \forall n > N_2(\eps)\ \underbrace{|a_n - a_{n_k}|}_{< \frac{\eps}{2}} + \underbrace{|a_{n_k} - A|}_{< \frac{\eps}{2}} < \eps 
\]
Это выполнено $\forall k > K\l(\frac{\eps}{2}\r)$ и $n, n_k > N_1\l(\frac{\eps}{2}\r)$

Так как $n_k \geq k$, потребуем $n, k > N_1\l(\frac{\eps}{2}\r)$

Изначально мы прибавили и вычли $a_{n_k}, k > \max \{N_1\l(\frac{\eps}{2}\r), K\l(\frac{\eps}{2}\r)\}$

$k$ --- вспомогательный аргумент, которого изначально не было (требовалось только показать существование). Поэтому при предъявлении $N_2(\eps)$ мы не будем использовать вспомогательное $K(\eps)$

Возьмем $N_2(\eps) = N_1\l(\frac{\eps}{2}\r)$
\end{Proof}

\newpage
\section{Функции}
\subsection{Понятия функции и ее предела}
\subsubsection{Функция. График функции}
$f(x) : \R \to \R$

\begin{definition}
    \textbf{Функцией} $f$ отображающей множество $X$ во множество $Y$ $(f : X \to Y)$ называется множество упорядоченных пар $\{(x, y)\}_{x \in X, y \in Y}$, причем каждый элемент $X$ встречается в какой-то паре на первой позиции и только один раз.

    В частности, $X$ называется областью определения функции и обозначается $D_f$

    Множество вторых элементов пар обозначается $E_f$, множество значений функции. $E_f \subset Y$
\end{definition}

\begin{definition}
    \textbf{График функции} --- множество точек на декартовой (координатной) плоскости, координаты которых --- пары функции
\end{definition}
\subsubsection{Инъекция, сюрьекция, биекция} 
\begin{definition}
    Функция называется \textbf{инъекцией}, если вторые элементы пар не повторяются
\end{definition}
\begin{definition}
    Функция называется \textbf{сюрьекцией}, если $E_f = Y$
    \[
        \forall y \in Y\ \exists x \in X\ : f(x) = y
    \]
\end{definition}
\begin{definition}
    Функция называется \textbf{биекцией}, если она инъекция и сюрьекция
\end{definition}
\subsubsection{Обратимость функции}
\begin{definition}
    Функция $f^{-1} : Y \to X$ называется \textbf{обратной} к $f : X \to Y$, если пары $f^{-1}$ -- пары $f$, где элементы поменяны местами 
\end{definition}

\begin{remark}
    Чтобы $f : X \to Y$ была обратима нужно, чтобы она была биекцией
\end{remark}

\begin{example}
    \[
        f(x) = \sin(x),\ \R \to \R\ \text{--- не обратима}
    \]
    \[
        f(x) = \sin(x),\ \l[-\frac{\pi}{2};\frac{\pi}{2} \r] \to [-1; 1]
    \]
    Тогда обратима
    \[
        f^{-1}(x) = \arcsin (y) \in \l[-\frac{\pi}{2};\frac{\pi}{2} \r]
    \]
\end{example}

Далее работаем с числовыми функциями ($X, Y \subset \R$)
\subsubsection{Предел функции по Коши}
\begin{definition}[Определение по Коши]
    $\Lim{x}{x_0} f(x) = A$, если
    \[
        \forall \eps > 0\ \exists \delta = \delta (\eps )\ \forall x \in \overset{\circ}{U}_{\delta } (x_0)\ |f(x) - A| < \eps
    \]
    \[
        \Updownarrow
    \]
    \[
        \forall \eps > 0\ \exists \delta = \delta (\eps )\ \forall x : 0 < |x - x_0| < \delta\ |f(x) - A| < \eps
    \]
\end{definition}

\begin{figure}[h]
  \centering
  \includegraphics[width=0.5\textwidth]{lectures/files/614_07_17.10.2025-23-17-14.png}
  
  \label{fig:614_07_17.10.2025-23-17-14.png}
\end{figure}

\begin{example}
    $\Lim{x}{3} (2x - 7) = -1$
    \[
        \forall \eps > 0\ \exists \delta = \delta (\eps )\ \forall x : 0 < |x - 3| < \delta\ |(2x - 7) - (-1)| < \eps
    \]
    \[
        |2x - 6| < \eps
    \]
    \[
        2 \cdot |x - 3| < \eps
    \]
    \[
        |x - 3| < \frac{\eps}{2}
    \]
    Возьмем $\delta (\eps) = \frac{\eps }{3}$
\end{example}

\subsubsection{Предел функции по Гейне}

\begin{definition}[Определение по Гейне]
    $\Lim{x}{x_0} f(x) = A$, если истинна импликация

    \[
    \underset{
        \substack{
            x_n \xrightarrow[n \to \infty]{} x_0 \\ 
            x_n \neq x_0
        }
    }{\forall \{x_n\}_{n \in \N}} 
    \Rightarrow f(x_n) \xrightarrow[n \to \infty]{} A
    \]
\end{definition}

\begin{example}
    $\Lim{x}{3} (2x - 7) = -1$

    Возьмем $\{x_n\}_{n \in \N} : x_n \xrightarrow[n \to \infty]{} 3, x_n \neq 3$

    Посмотрим $f(x_n) = \underbrace{\underbrace{2 \underbrace{x_n}_3}_6 - 7}_{-1} \xrightarrow[n \to \infty]{} -1 $
\end{example}

\begin{theorem}
    Определения предела функции по Коши и Гейне равносильны
\end{theorem}

\begin{Proof}
    {\bf $(\Rightarrow):$}

    Имеем:
    \[
        \forall \eps > 0\ \exists \delta = \delta (\eps )\ \forall x : 0 < |x - x_0| < \delta\ |f(x) - A| < \eps
    \]
    Предположим, что посылка истинна. Тогда также имеем:
    \[
        x_n \xrightarrow[n \to \infty ]{} x_0,\ x_n \neq x_0, \text{ то есть } \forall \omega > 0\ \exists N_1(\omega )\ \forall n > N_1(\omega )\ 0 < |x_n - x_0| < \omega 
    \]
    Хотим:
    \[
        f(x_n) \xrightarrow[n \to \infty ]{} A,\ \forall \eps > 0\ \exists N_2(\eps )\ \forall n > N_2(\eps )\ |f(x_n) - A| < \eps 
    \]

    Чтобы $|f(x_n) - A| < \eps$ нужно $0 < |x_n - x_0| < \delta(\eps )$ (Подставили в определение Коши $f(x_n)$ вместо $f(x)$). Поэтому возьмем $\omega = \delta (\eps )$. 
    
    Получаем $N_1(\delta (\eps ))$ и $\forall n > N_1(\delta (\eps ))\ 0 < |x_n - x_0| < \delta (\eps )$ и тогда $|f(x_n) - A| < \eps $

    Возьмем $N_2(\eps ) = N_1(\delta (\eps ))$

    {\bf $(\Leftarrow):$}

    Имеем:
    \[
    \underset{
        \substack{
            x_n \xrightarrow[n \to \infty]{} x_0 \\ 
            x_n \neq x_0
        }
    }{\forall \{x_n\}_{n \in \N}} 
    \Rightarrow f(x_n) \xrightarrow[n \to \infty]{} A
    \]

    Пп т.е верно:
    \[
        \exists \eps > 0\ \forall \delta = \delta (\eps )\ \exists x(\delta ) \in \overset{\circ}{U}_{\delta } (x_0)\ |f(x) - A| \geq \eps 
    \]
    Для каждого $\delta = \frac{1}{n} \text{ найдем } x_n \in \overset{\circ}{U}_{\delta } (x_0)$

    Получим
    \[
        \underbrace{x_0 - \frac{1}{n}}_{\xrightarrow{n \to \infty} x_0} 
        < x_n < 
        \underbrace{x_0 + \frac{1}{n}}_{\xrightarrow{n \to \infty} x_0}
    \]
    \[
        \{x_n \}_{n \in \N}: x_n \xrightarrow[n \to \infty ]{} x_0, x_n \neq x_0 
    \]
    Значит $f(x_n) \xrightarrow[n \to \infty ]{} A$

    Но $|f(x) - A| \geq \eps$. Получили противоречие (расписать определение и в качестве $\eps$ взять $\eps_0$)
\end{Proof}

\begin{example}
    Рассмотрим функцию Дирихле:
    \[
        D(x) = 
        \begin{dcases}
            1, &x \in \Q\\
            0, &x \notin \Q\\
        \end{dcases}
    \]
    Утверждается, что у данной функции нет предела ни в какой точке. Рассмотрим~$x_0 = 0$

    $\Lim{x}{0} D(x)\ \lnot \exists $

    $x_n = \frac{1}{n} \xrightarrow[n \to \infty ]{} 0,\ x_n \neq 0$; $f(x_n) \equiv 1 \xrightarrow[n \to \infty ]{} 1$

    $x_n' = \frac{\sqrt{2}}{n} \xrightarrow[n \to \infty ]{} 0, x_n' \neq 0$; $f(x_n') \equiv 0 \xrightarrow[n \to \infty ]{} 0$
\end{example}

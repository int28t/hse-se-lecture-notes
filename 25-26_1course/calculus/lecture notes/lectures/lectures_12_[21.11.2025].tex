\subsection{Обратные функции}

$f(x) : X \to Y\ D_f = X,\ E_f \subset Y$ 

$f^{-1} (y) : E_f \to X $

$f^{-1} : \{(y;x)\}_{x \in X,\ y \in E_f} $

$f(.) $ инъекция

$X, Y \subset \R$

\begin{theorem}[Теорема о достаточном условии обратимости]
    Если $f(x) $ --- строго монотонна, то она обратима
\end{theorem}

\begin{Proof}
Пусть $f(x)$ возрастает

о/п (от противного):
\[
    \exists x_1 \text{ и } x_2 : x_1 \neq x_2 \Rightarrow (x_1 > x_2) \lor (x_1 < x_2) \Rightarrow (y_1 > y_2) \lor (y_1 < y_2)
\]
Но: $y_1 = y_2, \bot$
\end{Proof}

\begin{theorem}[Теорема о критерии обратимости непрерывной функции]
    Если $f(x) $ непрерывна на $[a;b]$, то:
    \[
        f(x) \text{ обратима } \Leftrightarrow f(x) \text{ строго монотонна }
    \]
\end{theorem}

\begin{Proof}
{\bf $\Leftarrow$} Доказано ранее

{\bf $\Rightarrow$}

о/п (от противного)
\[
    1)\ \exists x_1 < x_2 < x_3 : y_1 > y_2 < y_3
\]
\[
    \text{ИЛИ}
\]
\[
    2)\ \exists x_1 < x_2 < x_3 : y_1 < y_2 > y_3
\]
\[
    3)\ \text{ ИЛИ когда-то равны } \ \Rightarrow \text{ тогда функция не инъективна } \Rightarrow \text{ она не обратима } \bot 
\]
//TODO: рисунок
\[
    1)\ \exists c: c \in (y_2; y_1),\ c \in (y_2; y_3)
\]
По теореме о промежуточных значениях:
\[
    \exists z_1 \in (x_1,x_2) : f(z_1) = c
\]
\[
    \exists z_2 \in (x_2,x_3) : f(z_2) = c
\]
\[
    z_1 \neq z_2
\]
\[
    \bot
\]
\end{Proof}

\begin{theorem}
    $$f(x)$$
    \begin{enumerate}
        \item[1)] $D_f = [a;b]$ 
        \item[2)] $f(x)$ непрерывна на $[a;b]$
        \item[3)] $f(x)$ строго монотонна на $[a;b]$, то
    \end{enumerate}
    $$f^{-1}(y)$$
    \begin{enumerate}
        \item[1)] область определения --- отрезок с концами $f(a)$ и $f(b)$
        \item[2)] непрерывна на этом отрезке 
        \item[3)] также строго монотонна на этом отрезке
    \end{enumerate}
\end{theorem}
\begin{Proof}
    \begin{enumerate}
        \item[\bf 1)] $\Rightarrow$ из теоремы о промежуточных значениях
        \item[\bf 3)] Пусть $f(.)$ возрастает. о/п:
        \[
            \exists y_1 < y_2 \in [f(a); f(b)] : f^{-1} (y_1) \geq f^{-1} (y_2)  
        \]
        \[
            x_1 \geq x_2 \Rightarrow y_1 \geq y_2 \text{ но } y_1 < y_2\ \bot
        \]
    \end{enumerate}
\end{Proof}
\begin{remark}[Словарик]
    Без ограничений общности --- кажется что мы накладываем ограничение превращая в частный случай, но на самом деле не влияя на справедливость доказательства в целом, так как остальные случаи рассматриваются похожим образом
\end{remark}
\begin{example}[Упражнение]
    \[
        \forall y_1, y_2 \in E_f = D_{f^{-1} }\ (y_1 < y_2 \Rightarrow f^{-1}(y_1) < f^{-1} (y_2)  )
    \]
    \[
        \ldots \overline{y_1 < y_2} \lor f^{-1}(y_1) < f^{-1}(y_2)  
    \]
    \[
        \exists y_1,y_2 \in E_f\ (y_1 < y_2 \land f^{-1}(y_1) \geq f^{-1}(y_2)  )
    \]
\end{example}
\begin{Proof}[]
    \begin{enumerate}
        \item[\bf 2)] $f^{-1}(y) $
        
        //TODO: рисунок 

        Предположим, что она не непрерывна на отрезке, тогда у нее разрыв типа скачок. Но тогда в множестве значений появилась дыра (или две), чего не может быть (теория из доп листка)

        Докажем другим способом:

        \[
            y_0 \in (f(a); f(b))
        \]
        \[
            \Lim{y}{y_0} f^{-1}(y) = f^{-1}(y_0)  
        \]
        \[
            \forall \eps > 0\ \exists \delta = \delta (\eps )\ \forall y \in U_{\delta }(y_0)\ |f^{-1}(y) - \underbrace{f^{-1}(y_0)}_{x_0 \in (a;b)}| < \eps   
        \]
        Без ограничений обзности считаем, что:
        \[
            U_{\eps }(x_0) \subset (a;b) 
        \]
        //TODO
        \[
            x_0 + \eps,\ x_0 - \eps \in (a;b)
        \]
        \[
            \Downarrow
        \]
        \[
            y_1 = f(x_0 - \eps ),\ y_2 = f(x_0 + \eps )
        \]
        \[
            y_1 < y_0 < y_2 \text{ тк $f$ возрастает}
        \]
        Возьмем $\delta = \min \{y_0 - y_1; y_2 - y_0 \}$
        \[
            \forall y \in U_{\delta }(y_0) \Rightarrow y_1 < y < y_2 \text{ тк $f^{-1} $ возрастает}
        \]
        \[
            \underset{= x_0 - \eps }{f^{-1}(y_1)} < f^{-1}(y) < \underset{= x_0 + \eps }{f^{-1}(y_2)}\ |f^{-1}(y) - \underset{\text{todo}}{x_0}| < \eps
        \]
    \end{enumerate}
\end{Proof}

\begin{example}[Следствие]
    Если $f(x)$ определена, непрерывна и строго монотонна на промежутке 
    \[
        (a; b) a,b \in \overline{\R} (a;b] a \in \overline{\R}
    \]
    TODO
\end{example}
Без доказательства

TODO!!!

$\Lim{n}{\infty} \l(1 + \frac{1}{n}\r)^n = e$,\ $\Lim{n}{\infty} \l(1 - \frac{1}{n}\r)^{-n} = e$

\begin{remark}
    Неопределенность $\frac{0}{0}: \Lim{x}{x_0} \frac{f(x)}{g(x)}; \Lim{x}{x_0} f(x) = 0,\ \Lim{x}{x_0} g(x) = 0 $, а также $\frac{\infty}{\infty}, \infty - \infty$

    $x_0 \in \overline{\R} = \R \cup \{\pm \infty\}$
\end{remark}

\begin{Proof}
    Хотим:
    
    $\Lim{n}{+\infty} \l(1 + \frac{1}{t}\r)^t = e$,\ $\Lim{n}{-\infty} \l(1 + \frac{1}{t}\r)^{t} = e$

    \[
        \l(1 + \frac{1}{[t] + 1} \r)^{[t]} \leq \l(1 + \frac{1}{t} \r)^{[t]} \leq \l(1 + \frac{1}{[t]} \r)^{[t] + 1},\ \fbox{$t \geq 1, t \in \R$}
    \]

    \[
        \forall \eps > 0\ \exists N(\eps) \forall n > N(\eps) \l| \l(1 + \frac{1}{n}\r)^{n+1} - e \r| < \eps 
    \]
    $h(t)$ принимают те же значения, что и $a_n$

    {\bf Утверждение:} $\Lim{t}{+\infty} h(t) = e$

    \[
        \forall \eps > 0\ \exists M(\eps)\ \forall t > M(\eps) \l|\l(1 + \frac{1}{[t]} \r)^{[t] + 1} - e\r| < \eps
    \]

    $M(\eps) = N(\eps) + 1$

    \[
        \forall t > M(\eps)\ \text{верно}\ [t] > N(\eps)
    \]

    Аналогично рассмотрим $\Lim{n}{\infty} \l(1 + \frac{1}{n + 1} \r)^{n} = e$

    Аналогично доказывается $\Lim{n}{-\infty} \l(1 + \frac{1}{t}\r)^{t} = e$
\end{Proof}

\fbox{2} $\Lim{x}{0} (1 + x)^{\frac{1}{x}} = e \Leftrightarrow \Lim{x}{0^{\pm}} (1 + x)^{\frac{1}{x}} = e$

\begin{Proof}
Замена $t = \frac{1}{x}$
\[
    \Lim{x}{0^+} (1 + x)^{\frac{1}{x}} = \Lim{t}{+\infty} \l(1 + \frac{1}{t}\r)^{t} = e
\]
\[
    \Lim{x}{0^-} (1 + x)^{\frac{1}{x}} = \Lim{t}{-\infty} \l(1 + \frac{1}{t}\r)^{t} = e
\]
\end{Proof}

{\bf Следствие 1} 
\[
    \Lim{x}{0} \frac{\ln (1 + x)}{x} = 1
\]
\[
    \frac{1}{x} \cdot \ln (1 + x) = \ln (1 + x)^{\frac{1}{x}} \rightarrow \ln e = 1
\]

Непрерывность $f(x)$ в т $x_0$
\[
    \Lim{x}{x_0} f(x) = f(x_0) = f(\Lim{x}{x_0} x)
\]
\[
    \Lim{x}{x_0} \ln (x) = \ln (x_0)
\]

{\bf Следствие 2}
\[
    \Lim{x}{x_0} \frac{e^x - 1}{x} = 1
\]
Замена $t = e^x - 1$, при $x \to 0,\ t = e^x - 1 \to 0$,\
$t + 1 = e^x, x = \ln (t + 1)$
\[
    \frac{t}{\ln (t + 1)} = \frac{1}{\frac{\ln(t + 1)}{t} \to 1} \to 1
\]

$\Lim{x}{0} \frac{e^x - e^0}{x - 0} = (e^x)^{'} |_{x=0}$

\[
    f'(x_0) = \Lim{x}{x_0} \frac{f(x) - f(x_0)}{x - x_0}
\]

\section{Непрерывность функции на отрезке}

\begin{definition}
    Функция непрерывна на $[a; b]$, если

    \begin{itemize}
        \item[1)] $[a;b] \subset D_f$
        \item[2)] $f(x)$ непрерывна в т. $x_0 \in (a;b)$
        \item[3)] $\Lim{x}{a^+} f(x) = f(a)\ \Lim{x}{b^-} f(x) = f(b)$
    \end{itemize}
\end{definition}

TODO: рисунок

\begin{theorem}
    Если функция непрерывна на отрезке $[a;b]$, то 
    \begin{itemize}
        \item[1)] она ограничена на $[a;b]$
        \item[2)] достигает $\sup f(x) x \in [a;b] = \sup E_f$ и $\inf f(x) x \in [a;b] = \inf E_f$
    \end{itemize}
\end{theorem}

\[
    M = \sup f(x) x \in [a;b] \in \overline{\R}
\]
\[
    m = \inf f(x) x \in [a;b] \in \overline{\R}
\]

\begin{itemize}
    \item[1)] $M \in \R$
    \item[2)] $\exists x_0 \in [a;b] M = f(x_0)$
\end{itemize}

Построим $a_n \uparrow M$
\[
    \forall n \exists x_n f(x_n) \geq a_n
\]

TODO: рисунок

\[
    \text{Пп} \exists n_0 \forall x \in [a;b] f(x) < a_{n_0}\ \text{$a_{n_0}$ --- верхняя грань для $E_+$} \text{противоречие}
\]

Построили $\{x_n\}$: 
\[
    a_n \leq f(x_n) \leq M
\]
\[
    \Lim{n}{\infty} f(x_n) = M
\]
$x_n \in [a;b] \Rightarrow \text{ограничена}$
\[
    \exists x_{n_k} : x_{n_k} \underset{k \to \infty}{\rightarrow} x_0
\]
\[
    \text{Предельный переход в нер-вах}\ a \leq x_{n_k} \leq b \Rightarrow x_0 \in [a;b]
\]
\[
    \text{Определение непрерывности по Гейне}\ f(x_{n_k}) \underset{k \to \infty}{\rightarrow} f(x_0)
\]
\[
    f(x_{n_k}) \underset{k \to \infty}{\rightarrow} M
\]

\begin{theorem}
    Непрерывная на отрезке функция принимает все промежуточные значения, т.е $f(x), [a;b]$

    $A = f(x_1), B = f(x_2), A < B\ x_1, x_2 \in [a;b]$

    тогда $\forall C \in (A;B) \exists x_0 \in [a;b] : f(x_0) = C$
\end{theorem}

{\bf Следствие} $f(x)\ \text{непрерывна на}\ [a;b] = D_f, \text{то}\ E_f = [m; M]$

\begin{Proof}
Построим последовательность влож отрезков $\l\{[a_k; b_k] \r\}_{k \in \N}$, будем считать $x_1 < x_2$

$[a_1; b_1] = [x_1; x_2]$

Возьмем середину $x_3 = \frac{x1 + x2}{2}$

\begin{itemize}
    \item[1.] $f(x_3) = C$ чтд
    \item[2.] $f(x_3) < C$ $[a_2; b_2] = [x_3; b_1]$
    \item[3.] $f(x_3) > C$ $[a_2; b_2] = [a_1; x_3]$
\end{itemize}
\end{Proof}

TODO: рисунок

\begin{Proof}[]
$\{a_k\}_k$ --- неубывающая и огр сверху b

$\{b_k\}_k$ --- невозрастает и огр снизу a

При $k \to \infty$: $a_k \to \alpha$, $b_k \to \beta$

$b_k - a_k = \frac{1}{2^{k - 1}} (x_2 - x_1) \underset{k \to \infty}{\to} 0$

Следовательно по теореме Вейерштрасса: $\alpha = \beta$

\[
    f(a_k) < C \to f(\alpha) \leq C
\]
Непрерывность $f$ по Гейне в т. $x_0 = \alpha$
\[
    f(b_k) > C \to f(\beta) \geq C
\]
\[
    f(\alpha) = f(\beta) = C
\]
\end{Proof}

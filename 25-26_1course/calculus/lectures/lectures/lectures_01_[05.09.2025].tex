\section{Информация}
\subsection{Оценка}
Тут когда-нибудь появится оценка

\section{Формальная логика}
\begin{definition}
    \textbf{Высказывание} - словестное утверждение, про которое можно сказать, истинное оно или ложное.

    Обозначение: заглавные латинские буквы: $A, B, C \ldots$
\end{definition}

\begin{definition}
    \textbf{Предикат} - высказывание, зависящее от переменной (при этом не являющееся высказыванием)
\end{definition}

\begin{example}
    $$ B(x) : x + 5 = 10 $$
\end{example}

\subsection{Кванторы}
\begin{itemize}
    \item $\forall$ - всеобщности
    \item $\exists$ - существования
\end{itemize}

\subsection{Метод математической индукции}
$$\fbox{
    $\forall n \in \N\ P(n)$
} \text{ --- истинно, если:} $$

\begin{itemize}
    \item[1)] $P(1)$ - истинно (база)
    \item[2)] $\forall n \in \N\ (P(n) \to P(n + 1))$ - истинно (шаг)
\end{itemize}

\documentclass[a4paper,12pt]{article}
\usepackage{header}

\title{Математический анализ. Коллок I\\1 курс, 25-26}
\author{github.com/int28t/hse-se-lecture-notes}
\date{}

\begin{document}
    \pagestyle{empty}
	\maketitle
	\tableofcontents{}
    \newpage
    \pagestyle{plain}
    \section{Понятие высказывания и n-местного предиката. Логические операции. Кванторы. Построение отрицания к высказыванию с кванторами.}
    \subsection{Понятие высказывания и n-местного предиката}
    \subsection{Логические операции}
    \subsection{Кванторы}
    \subsection{Построение отрицания к высказыванию с кванторами}

    \newpage
    \section{Доказательства методами математической индукции и от противного. Неравенство Бернулли.}
    \subsection{Метод математической индукции}
    $$\fbox{
        $\forall n \in \N\ P(n)$
    } \text{ --- истинно, если:} $$

    \begin{itemize}
        \item[1)] $P(1)$ - истинно (база)
        \item[2)] $\forall n \in \N\ (P(n) \to P(n + 1))$ - истинно (шаг)
    \end{itemize}
    \subsection{Доказательство от противного. Пример}
    Доказать, что количество простых чисел бесконечно
    
    Пп (предположим противное). Тогда количество простых чисел конечное число:
    $$ n_1,\ldots,n_k$$
    Рассмотрим следующее число:
    $$ m = n_1\cdot \ldots \cdot n_k+1,\ m \in \N,\ m > n_i\ \forall i = \overline{1, k}\ \text{то есть}\ m \neq n_i\ \forall i = \overline{1, k}$$
    Следовательно, $m$ - составное. Тогда:
    $$ m = n^{\alpha_1}_1 \cdot \ldots \cdot n^{\alpha_k}_k $$
    $$ \exists n_j : m\ \vdots\ n_j $$
    Но $m = n_1\cdot \ldots \cdot n_k+1$ и при делении на $n_i\ \forall i = \overline{1, k}$ дает остаток 1 ($\bot$)
    \subsection{Неравенство Бернулли}
    $$ \fbox{
        $ \forall n \in \N\ \forall x \geq -1: (1 + x)^n \geq 1 + xn $
    } \text{ --- неравенство Бернулли} $$

    Докажем с помощью ММИ
    $$ \forall n \in \N\ \underbrace{\forall x \geq -1\ \underbrace{(1 + x)^n \geq 1 + xn}_{Q(n)}}_{P(n)} $$

    \begin{itemize}
        \item[1)] $ \forall x \geq -1\ (1 + x) \geq 1 + x$ - истина
        \item[2)] Предположим $(1 + x)^{n_0} \geq 1 + xn_0$ - истина. Докажем, что $(1 + x)^{n_0 + 1} \geq 1 + x(n_0 + 1)$:
        
        $$ (1 + x)^{n_0 + 1} \geq 1 + x(n_0 + 1) $$
        $$ (1 + x)^{n_0 + 1} = (1 + x)^{n_0}\underbrace{(1 + x)}_{\geq 0} \geq (1 + x)(1 + xn_0) = $$
        $$ = 1 + x + xn_0 + \underbrace{x^2n_0}_{\geq 0} \geq 1 + x + xn_0 = 1 + x(n_0 + 1) $$
    \end{itemize}

    \newpage
    \section{Перестановки, размещения и сочетания. Бином Ньютона.}
    \subsection{}
    \subsection{}

    \newpage
    \section{Понятие последовательности. Предел последовательности. Единственность предела. Ограниченные, бесконечно малые, бесконечно большие  и отделимые от нуля последовательности. Связь между ними. Ограниченность сходящейся последовательности. Отделимость от нуля последовательности, сходящейся не к нулю.}
    \subsection{}
    \subsection{}
    \subsection{}
    \subsection{}
    \subsection{}

    \newpage
    \section{Арифметические свойства предела последовательности.}

    \newpage
    \section{Предельный переход в неравенствах. Теорема о зажатой последовательности.}
    \subsection{}
    \subsection{}

    \newpage
    \section{Ограниченные подмножества действительных чисел. Аксиома непрерывности действительных чисел. Верхняя и нижняя грань. Точная верхняя и точная нижняя грань. Теорема о существовании точной верхней и нижней грани.}
    \subsection{}
    \subsection{}
    \subsection{}
    \subsection{}
    \subsection{}

    \newpage
    \section{Теорема Вейерштрасса.}

    \newpage
    \section{Число e. Постоянная Эйлера.}

    \newpage
    \section{Подпоследовательность. Предельная точка последовательности. Частичный предел. Эквивалентность понятий частичного предела и предельной точки.}
    \subsection{}
    \subsection{}
    \subsection{}
    \subsection{}

    \newpage
    \section{Теорема Больцано-Вейерштрасса.}

    \newpage
    \section{Фундаментальные последовательности. Критерий Коши.}
    \subsection{}
    \subsection{}

    \newpage
    \section{Понятие функции, числовой функции. График числовой функции. Инъекция, сюръекция, биекция.}
    \subsection{}
    \subsection{}
    \subsection{}

    \newpage
    \section{Предел функции в точке: определения по Коши и по Гейне. Эквивалентность двух определений. Арифметика предела функции. Теорема о зажатой функции.}
    \subsection{}
    \subsection{}
    \subsection{}
    \subsection{}

    \newpage
    \section{Сходимость стандартных последовательностей.}
\end{document}

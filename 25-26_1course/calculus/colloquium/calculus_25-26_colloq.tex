\documentclass[a4paper,12pt]{article}
\usepackage{header}

\title{Математический анализ. Коллок I\\1 курс, 25-26}
\author{github.com/int28t/hse-se-lecture-notes}
\date{}

\begin{document}
    \pagestyle{empty}
	\maketitle
	\tableofcontents{}
    \newpage
    \pagestyle{plain}
    \flushleft
    \section{Понятие высказывания и n-местного предиката. Логические операции. Кванторы. Построение отрицания к высказыванию с кванторами.}
    \subsection{Понятие высказывания и n-местного предиката}
    \textbf{Высказывание} - словестное утверждение, про которое можно сказать, истинное оно или ложное.

    Обозначение: заглавные латинские буквы: $A, B, C \ldots$
    
    /TODO: N-местный предикат -
    \subsection{Логические операции}
    \subsection{Кванторы}
    \begin{itemize}
        \item $\forall$ - всеобщности
        \item $\exists$ - существования
    \end{itemize}
    \subsection{Построение отрицания к высказыванию с кванторами}

    \newpage
    \section{Доказательства методами математической индукции и от противного. Неравенство Бернулли.}
    \subsection{Метод математической индукции}
    $$\fbox{
        $\forall n \in \N\ P(n)$
    } \text{ --- истинно, если:} $$

    \begin{itemize}
        \item[1)] $P(1)$ - истинно (база)
        \item[2)] $\forall n \in \N\ (P(n) \to P(n + 1))$ - истинно (шаг)
    \end{itemize}
    \subsection{Доказательство от противного. Пример}
    Доказать, что количество простых чисел бесконечно
    
    Пп (предположим противное). Тогда количество простых чисел конечное число:
    $$ n_1,\ldots,n_k$$
    Рассмотрим следующее число:
    $$ m = n_1\cdot \ldots \cdot n_k+1,\ m \in \N,\ m > n_i\ \forall i = \overline{1, k}\ \text{то есть}\ m \neq n_i\ \forall i = \overline{1, k}$$
    Следовательно, $m$ - составное. Тогда:
    $$ m = n^{\alpha_1}_1 \cdot \ldots \cdot n^{\alpha_k}_k $$
    $$ \exists n_j : m\ \vdots\ n_j $$
    Но $m = n_1\cdot \ldots \cdot n_k+1$ и при делении на $n_i\ \forall i = \overline{1, k}$ дает остаток 1 ($\bot$)
    \subsection{Неравенство Бернулли}
    $$ \fbox{
        $ \forall n \in \N\ \forall x \geq -1: (1 + x)^n \geq 1 + xn $
    } \text{ --- неравенство Бернулли} $$

    Докажем с помощью ММИ
    $$ \forall n \in \N\ \underbrace{\forall x \geq -1\ \underbrace{(1 + x)^n \geq 1 + xn}_{Q(n)}}_{P(n)} $$

    \begin{itemize}
        \item[1)] $ \forall x \geq -1\ (1 + x) \geq 1 + x$ - истина
        \item[2)] Предположим $(1 + x)^{n_0} \geq 1 + xn_0$ - истина. Докажем, что $(1 + x)^{n_0 + 1} \geq 1 + x(n_0 + 1)$:
        
        $$ (1 + x)^{n_0 + 1} \geq 1 + x(n_0 + 1) $$
        $$ (1 + x)^{n_0 + 1} = (1 + x)^{n_0}\underbrace{(1 + x)}_{\geq 0} \geq (1 + x)(1 + xn_0) = $$
        $$ = 1 + x + xn_0 + \underbrace{x^2n_0}_{\geq 0} \geq 1 + x + xn_0 = 1 + x(n_0 + 1) $$
    \end{itemize}

    \newpage
    \section{Перестановки, размещения и сочетания. Бином Ньютона.}
    \subsection{Перестановки, размещения и сочетания}
    \textbf{Перестановка} - упорядоченное множество размера $n$ 

    \# перестановок $= n!$

    \textbf{Размещения} - упорядоченное подмножество размера $k$ множества размера $n$

    \# размещений $= \frac{n!}{(n - k)!} = A^k_n$

    \textbf{Сочетания} - неупорядоченное подмножество размера $k$ множества размера $n$

    Одному сочетанию соответствуют $k!$ размещений

    $ \frac{A^k_n}{k!} = \frac{n!}{k!(n-k)!} = C^k_n $
    \subsection{Бином Ньютона}
    $$ \fbox{
    $ (a + b)^n = \Sum{k = 0}{n} C^k_n a^k b^{n-k} $
    } $$

    $$ C^0_n a^0 b^{n - 0} + C^1_n a^1 b^{n - 1} + \ldots + C^n_n a^n b^{n - n} $$

    где $C^0_n, C^1_n, \ldots $ - биномиальные коэффициенты
    \newpage
    \section{Понятие последовательности. Предел последовательности. Единственность предела. Ограниченные, бесконечно малые, бесконечно большие  и отделимые от нуля последовательности. Связь между ними. Ограниченность сходящейся последовательности. Отделимость от нуля последовательности, сходящейся не к нулю.}
    \subsection{Понятие последовательности}
    \textbf{Последовательность} - индексированный набор чисел

    $ \{a_n\}_{n \in \N} $
    
    \subsubsection{Способы задания последовательности:}
    \begin{itemize}
        \item[1)] Формульный $a_n = n^2 + n - 7$
        \item[2)] Рекуррентный $a_1 = 1, a_2 = 1, a_n = a_{n-1} + a_{n-2} $
    \end{itemize}
    \subsection{Предел последовательности}
    $\Lim{n}{\infty} a_n = A$, если 
    $$\fbox{$ \forall \eps > 0\ \exists N \in \N\ \forall n > N\ |a_n - A| < \eps $}$$
    $$ \Updownarrow $$
    $$ \forall \eps > 0\ \exists N \in \N\ \forall n > N\ -\eps < a_n - A < \eps $$
    $$ \Updownarrow $$
    $$ \forall \eps > 0\ \exists N \in \N\ \forall n > N\ A-\eps < a_n < A+\eps $$
    $$ \Updownarrow $$
    $$ \forall \eps > 0\ \exists N \in \N\ \forall n > N\ a_n \in U_\eps(A) $$
    \subsection{Единственность предела}
    \textbf{Теорема:} У последовательности может быть только 1 предел

    \begin{proof} Пп $\exists$ хотя бы 2 lim : $A$ и $B$, $A \neq B$
    $$ \forall \eps > 0\ \exists N_1(\eps) \in \N\ \forall n > N_1(\eps)\ a_n \in U_{\eps}(A) $$
    $$ \forall \eps > 0\ \exists N_2(\eps) \in \N\ \forall n > N_2(\eps)\ a_n \in U_{\eps}(B) $$
    Возьмем $\eps_0 = \frac{|A - B|}{3}$ и $n_0 = N_1(\eps_0) + N_2(\eps_0) $

    Получаем
    $$ a_{n_0} \in U_{\eps_0}(A),\ a_{n_0} \in U_{\eps_0}(B) $$
    Но 
    $$ U_{\eps_0}(A) \cap U_{\eps_0}(B) = \varnothing $$
    Противоречие
    \end{proof}
    \subsection{Ограниченные, бесконечно малые, бесконечно большие  и отделимые от нуля последовательности. Связь между ними}
    Последовательность называется \textbf{ограниченной}, если
    $$ \fbox{
        $ \exists c\ \forall n\ |a_n| \leq c $
    } = P(\{a_n\}) $$
    И \textbf{неограниченной}, если
    $$ \fbox{
        $ \forall c\ \exists n(c)\ |a_{n(c)}| > c $
    } $$
    \textbf{Бесконечно малой} (б.м.) последовательностью называют последовательность $\{a_n\}_{n\in\N}$ такую что, $ \Lim{n}{\infty} a_n = 0 $

    \textbf{Бесконечно большой} (б.б.) последовательностью называют последовательность $\{b_n\}_{n\in\N}$ такую что, $ \Lim{n}{\infty} b_n = \infty $

    $$ \forall M > 0\ \exists N(M) \in \N\ \forall n > N(M)\ |b_n| > M $$
    $$ b_n > M\ (+ \infty) $$
    $$ b_n < M\ (- \infty) $$

    Назовем последовательность $d_n$ \textbf{отделимой от нуля}, если:
    $$ \exists \delta > 0\ \forall n \in \N\ |d_n| \geq \delta $$

    Пример: $(-1)^n$
    \subsection{Ограниченность сходящейся последовательности}
    \subsection{Отделимость от нуля последовательности, сходящейся не к нулю}

    \newpage
    \section{Арифметические свойства предела последовательности.}

    \newpage
    \section{Предельный переход в неравенствах. Теорема о зажатой последовательности.}
    \subsection{}
    \subsection{}

    \newpage
    \section{Ограниченные подмножества действительных чисел. Аксиома непрерывности действительных чисел. Верхняя и нижняя грань. Точная верхняя и точная нижняя грань. Теорема о существовании точной верхней и нижней грани.}
    \subsection{}
    \subsection{}
    \subsection{}
    \subsection{}
    \subsection{}

    \newpage
    \section{Теорема Вейерштрасса.}

    \newpage
    \section{Число e. Постоянная Эйлера.}

    \newpage
    \section{Подпоследовательность. Предельная точка последовательности. Частичный предел. Эквивалентность понятий частичного предела и предельной точки.}
    \subsection{}
    \subsection{}
    \subsection{}
    \subsection{}

    \newpage
    \section{Теорема Больцано-Вейерштрасса.}

    \newpage
    \section{Фундаментальные последовательности. Критерий Коши.}
    \subsection{}
    \subsection{}

    \newpage
    \section{Понятие функции, числовой функции. График числовой функции. Инъекция, сюръекция, биекция.}
    \subsection{}
    \subsection{}
    \subsection{}

    \newpage
    \section{Предел функции в точке: определения по Коши и по Гейне. Эквивалентность двух определений. Арифметика предела функции. Теорема о зажатой функции.}
    \subsection{}
    \subsection{}
    \subsection{}
    \subsection{}

    \newpage
    \section{Сходимость стандартных последовательностей.}
\end{document}
